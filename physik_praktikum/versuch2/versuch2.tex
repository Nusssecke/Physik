\documentclass{article}

\usepackage[a4paper, top=2cm, bottom=3cm]{geometry}

\usepackage[ngerman]{babel}
\usepackage{csquotes}

\usepackage{booktabs}
\usepackage{mathtools}
\usepackage{amssymb}
\usepackage{enumitem}
\usepackage{amsmath}

\usepackage{hyperref}

\usepackage{graphicx}
\usepackage{wrapfig}

\usepackage{pgfplots}

\usepackage{siunitx}

\usepackage{biblatex}
\DefineBibliographyStrings{ngerman}{
  urlseen = {Abruf vom}
}
\addbibresource{quellen.bib}

\newcommand{\proofeq}{\overset{!}{=}}
\newcommand{\proofeqv}{\overset{!}{\Leftrightarrow}}
\newcommand{\equivalent}{\overset{\scriptscriptstyle\wedge}{=}}
\DeclarePairedDelimiter\ceil{\lceil}{\rceil}
\DeclarePairedDelimiter\floor{\lfloor}{\rfloor}

\date{7.03.2022}
\title{Physikalisches Grundpraktikum Teil I \\ (Mechanik und Thermodynamik) \\ Versuch 2 Drehbewegung}
\author{Finn Wagner}

\begin{document}
    \maketitle

    TODO: Messwert Zettel vorbereiten.
    \section{Versuchsziel und Versuchsmethode}
      In diesem Experiment soll das Trägheitsmoment eines Kreisels über drei verschiedene Methoden bestimmt werden.
      

      Was ist ein Trägheitsmoment? aus Lehrbuch

      Durchführung mit Bild woher Variblen an Bild
      Ergebnisse(Messergebnisse), da nochmal ordentlich
      Auswertung
      Formeln aus der Anleitung nicht nochmal herleiten


      1. Teilversuch Graphische Auswertung daraus S
      Fehler auch Graphische
      \( \Delta I = \left|\right| \) <- maximaler Fehler
      Bei wenigen Messungen eher maximaler Fehler da schlechte Statistik für Standardabweichung nicht gut.


      z.B. I = 0.0728359
      \( \Delta I = 0.00017 \) <- nur zwei signifikante stellen
      I = 0,7284 +- 0.0017

      Länge ca.5 Seiten.

    Vergleich der Ergebnisse
    Hoffentlich ähnliche Ergebnisse.
    Diskutiere z.B Reibungseffekte, Kleinwinkelnäherung, etc.
    Fazit beste Methode, z.B. kleinester Fehler, Durchführung am wenigsten Aufpassen.

    Anhang Messwerte

    \section{Grundlagen}

    \section{Durchführung}

    \section{Formeln}

\end{document}