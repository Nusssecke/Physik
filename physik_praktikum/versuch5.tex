\documentclass{article}

\usepackage[a4paper, top=2cm, bottom=3cm]{geometry}
\usepackage{babel}[german]
\usepackage{booktabs}
\usepackage{mathtools}
\usepackage{amssymb}
\usepackage{enumitem}
\usepackage{amsmath}
\usepackage{hyperref}
\newcommand{\proofeq}{\overset{!}{=}}
\newcommand{\proofeqv}{\overset{!}{\Leftrightarrow}}
\newcommand{\equivalent}{\overset{\scriptscriptstyle\wedge}{=}}
\DeclarePairedDelimiter\ceil{\lceil}{\rceil}
\DeclarePairedDelimiter\floor{\lfloor}{\rfloor}

\date{7.03.2022}
\title{Physikalisches Grundpraktikum Teil I (Mechanik und Thermodynamik) Versuch 5 Lufballon mit \(CO_2\)}
\author{Finn Wagner}

\begin{document}
    \maketitle

    \section{Versuchsziel und Versuchsmethode}
    In diesem Versuch wird die Dichte von \(CO_2\) über die Fallzeit eines mit \(CO_2\) gefüllten Luftballons bestimmt.
    Der Fall des Luftballons wird als Bewegung eines kugelförmigen Körpers in einer Flüssigkeit (Luft) mit laminarer Strömung approximiert.

    \section{Grundlagen}
    Auf jeden Körper wirkt im Schwerefeld der Erde eine Zentralkraft Richtung Erdmittelpunkt, die von seiner Masse unabhängig ist.
    Weiterhin wirkt auf Körper mit echter Ausdehnung (keine Punktmasse) in Flüssigkeiten/Gasen wie Luft eine Auftriebskraft, 
    durch Verdrengung der eigentlichen Mediums.
    Für Runde Körper wirkt das Gesetz von Stokes.

    Wir betrachten folgende Kräfte die auf die Luftballons wirken

    Ein kugelförmiger Körper mit Dichte \(\rho_K\) in einer Flüssigkeit mit Dichte \(\rho_{Fl}\) fällt in eben jener Flüssigkeit nach unten

    1. Die Schwerkraft beschleunigt den Luftballon nach unten mit
    \[ F_G = mg = \rho_K V g \]
    Wo wir hier die Masse des Luftballons durch seine Dichte mal sein Volumen ersetzen \( m = \rho_K V \)
    TODO COMPLETE

    \section{Versuchsaufbau}
        \subsection{Material}
        \begin{itemize}
            \item 2 Luftballons
            \item 1 Flasche Mineralwasser Classic
            \item Smartphone zum Aufnehmen eines Videos
            \item 1 Maßband
            \item 1 Stoppuhr
        \end{itemize}

        


    \section{Quellen}
    \url{https://de.wikipedia.org/wiki/Gesetz_von_Stokes}
    Anleitung zum Physikalischen Grundpraktikum Teil I (Mechanik und Thermodynamik) JLU Gießen II. Physikalisches Institut Version 1.3, 28.02.2022 von Jens Sören Lange
    https://texample.net/tikz/examples/bernoulli/

\end{document}