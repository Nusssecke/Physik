\documentclass{article}

\usepackage[a4paper, top=2cm, bottom=3cm]{geometry}
\usepackage{babel}[german]
\usepackage{booktabs}
\usepackage{mathtools}
\usepackage{amssymb}
\usepackage{enumitem}
\usepackage{amsmath}
\usepackage{hyperref}

\usepackage{siunitx}

\newcommand{\proofeq}{\overset{!}{=}}
\newcommand{\proofeqv}{\overset{!}{\Leftrightarrow}}
\newcommand{\equivalent}{\overset{\scriptscriptstyle\wedge}{=}}
\DeclarePairedDelimiter\ceil{\lceil}{\rceil}
\DeclarePairedDelimiter\floor{\lfloor}{\rfloor}

\date{7.03.2022}
\title{Physikalisches Grundpraktikum Teil I (Mechanik und Thermodynamik) Versuch 5 Lufballon mit \(CO_2\)}
\author{Finn Wagner}

\begin{document}
    \maketitle

    \section{Versuchsziel und Versuchsmethode}
    In diesem Versuch wird die Dichte von \(CO_2\) über die Fallzeit eines mit \(CO_2\) gefüllten Luftballons bestimmt.
    Der Fall des Luftballons wird als Bewegung eines kugelförmigen Körpers in einer Flüssigkeit (Luft) mit laminarer Strömung approximiert.

    \section{Grundlagen}
    Auf jeden Körper wirkt im Schwerefeld der Erde eine Zentralkraft Richtung Erdmittelpunkt, die von seiner Masse unabhängig ist.
    Weiterhin wirkt auf Körper mit echter Ausdehnung (keine Punktmasse) in Flüssigkeiten/Gasen wie Luft eine Auftriebskraft, 
    durch Verdrengung der eigentlichen Mediums.
    Für Runde Körper wirkt das Gesetz von Stokes.

    Wir betrachten folgende Kräfte die auf die Luftballons wirken

    Ein kugelförmiger Körper mit Dichte \(\rho_K\) in einer Flüssigkeit mit Dichte \(\rho_{Fl}\) fällt in eben jener Flüssigkeit nach unten

    1. Die Schwerkraft beschleunigt den Luftballon nach unten mit
    \[ F_G = mg = \rho_K V g \]
    Wo wir hier die Masse des Luftballons durch seine Dichte mal sein Volumen ersetzen \( m = \rho_K V \)
    TODO COMPLETE

    TODO: Video aufgenommen ausgewertet mit Media Player Classic ausgewertet
    TODO: Beachten Vidoe nur mit 30fps aufgenommen Fehlerrechnung

    \section{Versuchsaufbau}
        \subsection{Material}
        \begin{itemize}
            \item 2 Luftballons
            \item 1 Flasche Mineralwasser Classic
            \item Smartphone zum Aufnehmen eines Videos
            \item 1 Maßband
            \item 1 Stoppuhr
        \end{itemize}

    TODO: SCHEMATIK EINFÜGEN PFEILLÄNGEN ÄNDERN

    \section{Durchführung}

    Luftballon auf Flasche gestüllpt und dann Flasche geschütelt.
    Wiederholtes schütteln. (Wasser in Ballon?)
    Zweiten Luftballon aufgepustet, sodass er gleich groß
    Gekennzeichnet
    Fallen gelassen von Unterkante des Türrahmens
    
    \section{Auswertung}
    Umfänge mitteln zu von \(a_1 = \SI{0.475}{m} und a_2 = \SI{0.477}{m} \)
    \(a = \SI{0.476}{m} \) und \(b = \SI{0.45}{m} \)
    Umfänge mitteln \( \frac{ \SI{0.476}{m} + \SI{0.45}{m} }{2} = \SI{0.463}{m} \)

    Volumen ausrechnen:
    \begin{equation} \label{eq:volumen}
        V = \frac{4}{3} \pi {\left( \frac{U}{2 \pi} \right) }^3 = \frac{4}{3} \pi \frac{U^3}{8 \pi^3} = \frac{1}{6} \frac{U^3}{\pi^2}
    \end{equation} % kopiert aus 6
    Eingesetzt und ausgerechnet:
    \( \SI{1.676 * 10^{-3} }{m^3} \)

    Dann Zeiten einsetzen:

    rho luft aus 5.4
    \begin{equation}
        \rho_{CO_2} = \SI{1.2041}{\frac{kg}{m^3}} + (\frac{t_{Luft}}{t_{CO_2}} - 1)+\frac{m_H}{V}
    \end{equation}
    Berechnen Sie aus den drei Messungen die Mittelwerte und setzen sie ein.

    Formel mit Volumenformel eingesetzt
    Ergebnis \(\)

    TODO: Prozent abweichung vom Literaturwert

    TODO: Fehler Luftballon wiegt 2g
    TODO: Wie viel Co2 in Waserflasche gelöst.
    TODO: Fehler Luft im Co2 ballon
    TODO: Schwer zu sehen wann Boden berührt
    TODO:Literaturwert Quelle angeben

    \section{Quellen}
    \url{https://de.wikipedia.org/wiki/Gesetz_von_Stokes}
    Anleitung zum Physikalischen Grundpraktikum Teil I (Mechanik und Thermodynamik) JLU Gießen II. Physikalisches Institut Version 1.3, 28.02.2022 von Jens Sören Lange
    https://texample.net/tikz/examples/bernoulli/
    Luftballon Bild

\end{document}