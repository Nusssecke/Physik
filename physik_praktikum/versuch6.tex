\documentclass{article}

\usepackage[a4paper, top=2cm, bottom=3cm]{geometry}
\usepackage{babel}[german]
\usepackage{booktabs}
\usepackage{mathtools}
\usepackage{amssymb}
\usepackage{enumitem}
\usepackage{amsmath}
\usepackage{hyperref}
\newcommand{\proofeq}{\overset{!}{=}}
\newcommand{\proofeqv}{\overset{!}{\Leftrightarrow}}
\newcommand{\equivalent}{\overset{\scriptscriptstyle\wedge}{=}}
\DeclarePairedDelimiter\ceil{\lceil}{\rceil}
\DeclarePairedDelimiter\floor{\lfloor}{\rfloor}

\date{7.03.2022}
\title{Physikalisches Grundpraktikum Teil I (Mechanik und Thermodynamik) Versuch 6 Innendruck eines Luftballons}
\author{Finn Wagner}

\begin{document}
    \maketitle

    \section{Versuchsziel und Versuchsmethode}

    \section{Auswertung}
    Der Umfang einer Kugel in 2 Dimensionen ist der eines Kreises mit \(U = 2 \pi R\). \\
    Umgeformt nach \(R\) ergibt sich \(\frac{U}{2 \pi} = r\) \\
    Das Volumen einer Kugel betrögt \( V = \frac{4}{3} \pi r^3 \) \\
    Es wurde zweimal der Umfang des Luftballons gemessen, einmal auf der Lange und Kurzen Seite.
    Einmal vom Munstück bis oben über den Ballon und einmal die Tallie \\
    Wir berechnen den Durchschnitt um das Volumen besser als Kugel zu approximieren \\
    Um das Volumen aus dem Durchmesser zu berechnen setzten wir den Umfang ein \\
    \[ V = \frac{4}{3} \pi {\left( \frac{U}{2 \pi} \right) }^3 = \frac{4}{3} \pi \frac{U^3}{8 \pi^3} = \frac{1}{6} \frac{U^3}{\pi^2}\]

    Die Fläche des Luftballonhalses berechnet sich aus dem Durchmesser mit \\
    \( A = \pi {\left(\frac{D}{2}\right)}^2 \) \\
    Wir stellen die im Skript gegebene Gleichung \( V = A \cdot v_a \cdot t_a \)
    nach \(v_a\) um. \\
    \( v_a = \frac{V}{A \cdot t_a} \) \\
    Jetzt setzen wir die Werte ein und vereinfachen:
    \[ v_a = \frac{V}{A \cdot t_a} = \frac{\frac{1}{6} \frac{U^3}{\pi^2} }{ \pi {\left(\frac{D}{2}\right)}^2 \cdot t_a} = \frac{2}{3} \frac{U^3}{\pi^3 D^2 t_a} \]
    Einheitenrechnung \\
    \[ \frac{m^3}{m^3 \cdot s} = \frac{m}{s}\]

    Wir berechnen aus den Versuchen jeweils die Ausströmgeschwindigkeiten \(v_1\) bis \(v_3\) \\
    Zuerst berechnen wir den Innendurchmesser der Mundstücks aus den drei gemessenen Werten
    \(1.15cm, 1.1cm, 1.2cm\). Das ergibt \( (1.15cm + 1.1cm + 1.2cm)/3 = 1.15cm = 0.0115m \) \\
    Die Fläche \(A\) des Mundstücks beträgt \( \pi {\left(\frac{0.0115m}{2}\right)}^2 = 1,03869m^2\)

    \subsection{Durchführung 1}
    \(d_1 = 0.584 m\) \(d_2 = 0.511 m\) Durchschnitt der Luftballonumfänge \(0.5475m\) \\
    Die beiden durchgeführten Zeitmessungen weichen stark voneinander ab: \\
    \( (1.34s + 1.04s)/2 = 1.19s \) \\
    Wir setzen in die oben berechnete Formel ein: \\
    \( v_a = \frac{2}{3} \frac{{(0.5475m)}^3}{\pi^3 {(0.0115m )}^2 \cdot 1.19s} = 22.42 \frac{m}{s}\)

    \subsection{Durchführung 3}
    \(d_1 = 0.615 m\) \(d_2 = 0.564 m\) Durchschnitt der Luftballonumfänge \(0.5895m\) \\
    Die beiden durchgeführten Zeitmessungen weichen stark voneinander ab: \\
    \( (1.45s + 1.61s + 1.64s )/3 = 1,5\bar{6}s \) \\
    Wir setzen in die oben berechnete Formel ein: \\
    \( v_a = \frac{2}{3} \frac{{(0.5895m)}^3}{\pi^3 {(0.0115m )}^2 \cdot 1,5\bar{6}s} = 21,25877656 \frac{m}{s}\)

    \subsection{Mittlere Geschwindigkeit}
    \( 22.4217162 \frac{m}{s}, 24.8927916 \frac{m}{s}, 21,25877656 \frac{m}{s} \) \\
    Wir berechnen den Mittelwert der Ausströmgeschwindigkeiten:
    \(( 22.4217162 \frac{m}{s} + 24.8927916 \frac{m}{s} + 21,25877656 \frac{m}{s} )/3 = 22.85776145 \frac{m}{s}\)
    -umrechnen mit Bernoulliformel

    TODO Berücksichtigen Luftballon ausgeleiert über Versuche
    TODO Wann ist der Luftballon eigentlich leer?
    TODO Abhängig von Ballonbeschaffenheit
    TODO Abhägig wie doll aufgepustet?

    \subsection{Umrechnen zum Überdruck}
    Die Bernoulligleichung verbindet Drücke und Strömungsgeschwindigkeiten. 
    Wobei \(p_i\) der Innendruck des Luftballons und \(p_a\) der Außen/Umgebungsdruck\\
    \(p_i = p_a + \frac{1}{2} \rho v_a^2\)
    \(p_i - p_a\) ist die Druckdifferenz zwischen dem Inneren und Äußeren des Ballons. Also \\
    \( (p_i - p_a) = \frac{1}{2} \rho v_a^2 \) \\
    Gegeben in Aufgabe für \( P_a = 101 325 Pa\) und \(\rho = 1.2041 \frac{kg}{m^3} \) bei 20° Celsius auf Meereshöhe \\
    Eingesetzt
    \[ (p_i - p_a) = \frac{1}{2} \cdot 1.2041 \frac{kg}{m^3} \cdot {(22.85776145 \frac{m}{s})}^2 = 314.557 Pa \]
    TODO Kommastellen wegmachen \\
    TODO Beachten Vidoe nur mit 30fps aufgenommen Fehlerrechnung
    \subsection{Versuch mit verkleinerter Öffnung}

    \subsection{Weltall}
    Umformungen
    FAKTOR!!!

    \section{Fehlerrechnung}
    Die Längenmessungen wurden mit einenm handelsüblichen Maßband mit einer
    Genauigkeit von 1mm gemacht. Abgelesene Werte wurden kaufmännisch auf den
    nächsten Milimeter gerundet.

    Die Zeitmessungen von unterschiedlichen Personen mit Handystoppuhren gemacht

    Video aufgenommen ausgewertet mit Media Player Classic ausgewertet

    mit 0.2.1.14 Fehlerfortpflanzung \\
    \[ \Delta v_a = \sqrt{ {\left( \frac{ \partial v_a }{ \partial t } \cdot \Delta t \right)}^2 + {\left( \frac{ \partial v_a }{ \partial U } \cdot \Delta U \right)}^2 } \]
    \(\Delta t \approx 0.1s\) aus der Aufgabenstellung, wohl eher \(0.25s\) Fehlerabweichung
    \(\Delta U \approx 0.1mm\) aus der Aufgabenstellung, wohl eher \(0.5cm\) Fehlerabweichung durch ungenaues ansetzen des Maßbandes am Luftballon
    Geschwindigkeitsfehler \( \frac{m}{s}\)
    Die durchströmte Fläche A wird als fehlerlos angenommen

    Funktion mit \(v_a\) in Fehlerfortplanzungsformel einsetzen und Messfehler einsetzen

    Das Volumen des Luftballons ist sehr ungenau und weicht sehr stark vom reelen Wert ab, da mit Kugelvolumen berechnet wurde.




\end{document}