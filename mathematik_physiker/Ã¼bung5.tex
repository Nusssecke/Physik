\documentclass{article}

%\usepackage[a4paper, total={170mm,257mm}]{geometry}
\usepackage[a4paper, top=2cm, bottom=3cm]{geometry}
\usepackage{babel}[german]
\usepackage{booktabs}
\usepackage{mathtools}
\usepackage{amssymb}
\usepackage{enumitem}
\usepackage{amsmath}

\date{22.11.2021}
\title{Aufgabenblatt 5, Mathematik für Physiker 1}
\author{Florian Adamczyk, Finn Wagner}

\begin{document}
    \maketitle

    \section*{A 5.1}
    \begin{enumerate}[ label = (\alph*)]
        \item Sei \(c_n := \sqrt[n]{n} - 1\), zu zeigen:
        \[ n = {(1+c_n)}^n \geq^! 1 + \frac{n (n-1)}{2} {c_n}^2 \]
        Es gilt \( n = {(1 + (\sqrt[n]{n}) -1)}^n \) \\
        Einsetzen in die allgemeine binomischen Formel:
        \[ \Leftrightarrow {(1+c_n)}^n = \sum_{k=0}^{n} \binom{n}{k} 1^{n-k} {c_n}^k \]
        Ausklammern der ersten drei Terme:
        \[ \sum_{k=0}^{n} \binom{n}{k} 1^{n-k} {c_n}^k = \binom{n}{0} 1^n {c_n}^0 + \binom{n}{1} 1^{n-1} {c_n}^1 + \binom{n}{2} 1^{n-2} {c_n}^2 + \sum_{k=3}^{n} \binom{n}{k} 1^{n-k} {c_n}^k \]
        Es gilt \(n \in \mathbb{N}\) und damit \(n \geq 1\). Die n.te Wurzel von n ist größer gleich 1. Somit \(c_n \geq 0\).
        Ein Binomialkoeffitient kann ebenfalls nicht negativ werden. Alle Terme auf der rechten Seite der Gleichung sind also positiv, als Produkt von 1 und \(c_n\).
        Wir schätzen nun nach unten ab, indem wir Terme weglassen.
        \[ \binom{n}{0} 1^n {c_n}^0 + \binom{n}{1} 1^{n-1} {c_n}^1 + \binom{n}{2} 1^{n-2} {c_n}^2 + \sum_{k=3}^{n} \binom{n}{k} 1^{n-k} {c_n}^k \geq \binom{n}{0} 1^n {c_n}^0 + \binom{n}{2} 1^{n-2} {c_n}^2 \]
        Vereinfachen:
        \[ \binom{n}{0} 1^n {c_n}^0 + \binom{n}{2} 1^{n-2} {c_n}^2 = 1 + \frac{n!}{(n-2)! * 2!} {c_n}^2 = 1 + \frac{n (n-1)}{2} {c_n}^2 \]
        Die Aussage ist bewiesen. \\ \\
        Nun ist zu zeigen, dass \(\sqrt[n]{n} \to 1\) für \(n \to \infty \) \\
        Es gilt wie oben bewiesen: 
        \begin{gather*}
            n \geq 1 + \frac{n (n-1)}{2} {c_n}^2 \ | \ -1 \ | \ : n(n-1) > 0 \ \forall \ n \in \mathbb{N} \\
            \Leftrightarrow \frac{(n-1)}{n(n-1)} \geq \frac{{c_n}^2}{2} \\
            \Leftrightarrow \frac{1}{n} \geq \frac{{c_n}^2}{2}
        \end{gather*}
        Da \(c_n\) immer größer ist als 0, ist auch \(\frac{{c_n}^2}{2}\) immer größer als 0. Es gilt insgesamt:
        \[ \frac{1}{n} \geq \frac{{c_n}^2}{2} \geq 0 \]
        Mit dem Sandwichtheorem folgt:
        \[ \lim_{n \to \infty} \frac{1}{n} \geq \lim_{n \to \infty} \frac{{c_n}^2}{2} \geq \lim_{n \to \infty} 0 \]
        \[ 0 \geq \lim_{n \to \infty} \frac{{c_n}^2}{2} \geq 0 \]
        Es folgt, dass der Grenzwert von \(c_n\) 0 ist. \(\sqrt[n]{n}\) konvergiert damit gegen 1.

        \item Zu zeigen: Für \(a > 0\) gilt \( \sqrt[n]{a} \to 1 \) für \( n \to \infty \) \\
        Ab einem n groß genug, \(N \in \mathbb{N}\) \(n \geq N\) gilt: \\
        \(a \leq n\) für alle \(n\), daraus folgt \(\sqrt[n]{a} \leq \sqrt[n]{n}\) \\
        \begin{enumerate}[ label = \arabic*. Fall ]
            \item a = 1 \\
            Ist a = 1, so konvergiert \(\sqrt[n]{1}\) trivialler Weise gegen 1.
            
            \item \(1 < a\) \\
            Da die n.te Wurzel einer Zahl a größer als 1 auch immer größer gleich 1 sein muss gilt:
            \[1 \leq \sqrt[n]{a} < \sqrt[n]{n} \]
            Mit dem Sandwichtheorem folgt:
            \[\lim_{n \to \infty} 1 \leq \sqrt[n]{a} < \lim_{n \to \infty} \sqrt[n]{n} \]
            \[1 \leq \sqrt[n]{a} \leq 1\]
            Die Folge \(\sqrt[n]{a}\) konvergiert gegen 1.
    
            \item \(1 > a > 0 \) \\
            Wie von Aufgabenblatt 4 bekannt, lässt sich eine reele Zahl \(a \in \mathbb{R}^+\) beliebig nah
            durch eine rationale Zahl q mit \(x, y \in \mathbb{N} \) nähern.
            Für alle unendlich viele, immer besser werdende Näherungen von a gilt:
            \[q = \frac{x}{y} \Leftrightarrow \sqrt[n]{q} = \sqrt[n]{\frac{x}{y}} = \frac{\sqrt[n]{x}}{\sqrt[n]{y}}\]
            Der Grenzwert aller Folgen \({(q_n)_n}\) ist gleich 1, als Bruch aus zwei konvergenten Folgen, die beide nicht Null werden.
            \[\lim_{n \to \infty} \frac{\sqrt[n]{x}}{\sqrt[n]{y}} = \frac{1}{1} \]
            Da der Grenzwert der Folge \(\sqrt[n]{a}\) für alle Näherungen von a gleich 1 ist, konvergiert die Folge gegen 1.
        \end{enumerate}
    
        Es gilt für \(a > 0\) konvergiert \(\sqrt[n]{a} \to 1\) wenn \(n \to \infty\) \\
    \end{enumerate}

    \section*{A 5.2}
    Wir definieren die Folgen
    \[ a_n := { \left( 1 + \frac{1}{n} \right) }^n \ b_n := { \left( 1 + \frac{1}{n} \right) }^{n+1} \]
    Zu zeigen: \( {(a_n)}_n \) ist monoton wachsend:
    Es muss gelten:
    \[ \frac{ a_{n+1} }{ a_n } \geq 1 \]
    Definition einsetzen:
    \[ \frac{ a_{n+1} }{ a_n  }  = \frac{\left(1+\frac{1}{n+1}\right)^{n+1}}{\left(1+\frac{1}{n}\right)^n}  \quad = \quad \left(\frac{1+\frac{1}{n+1}}{1+\frac{1}{n}}\right)^n \cdot \; \left(1+\frac{1}{n+1}\right) \]//
    \[ = \quad \left(\frac{\frac{n+1+1}{n+1}}{\frac{n+1}{n}}\right)^n \cdot  \; \left(1+\frac{1}{n+1}\right) \]\\
    \[ = \quad \left( \frac{n^2 + 2n}{\left(n+1\right)^2}\right)^n \cdot  \; \left(1+\frac{1}{n+1}\right)  \]\\
    \[ = \quad \left(\frac{(n+1)^2}{(n+1)^2}+\frac{-1}{(n+1)^2}\right)^n \cdot  \; \left(1+\frac{1}{n+1}\right) \]\\
    
    \[\left(1+\frac{-1}{(n+1)^2}\right)^n  \cdot  \; \left(1+\frac{1}{n+1}\right) \geq \left(1+\frac{-n}{(n+1)^2}\right)\cdot\left(1+\frac{1}{n+1}\right)      \]\\
    Mit der Bernoulli-Ungleichung nach unten abschätzen, da gilt \( \frac{-1}{ {(n+1)}^2 } \geq -1 \):
    \[ \left(\frac{n^2 +n+1}{n^2+2n+}\right)\cdot\left(\frac{n+2}{n+1}\right) \quad = \quad \left(\frac{n^3 + 3n^2 +3n+2}{n^3+3n^2+3n+1}  \right) > 1    \]\\
    
    \( {(a_n)}_n \) ist damit monoton wachsend \\

    Zu zeigen: \( {(b_n)}_n \) ist monoton fallend:
    Es muss gelten:
    \[ \frac{ b_{n+1} }{ b_{n} } \leq 1 \]\\
    Definition einsetzen und umformen:
    \[   \frac{ b_{n+1} }{ b_{n} } \quad = \quad \frac{\left(1+\frac{1}{n+1}\right)^{n+2}}{\left(1+\frac{1}{n}\right)^{n+1}}   \]\\
    \[   = \quad \frac{\left(\frac{n+2}{n+1}\right)^{n+2}}{\left(\frac{n+1}{n}\right)^{n+1}}   \]\\
    \[   = \quad \left(\frac{\frac{n+2}{n+1}}{\frac{n+1}{n}}\right)^{n+1} \cdot \frac{n+2}{n+1}   \]\\
    \[   = \quad \left(\frac{n^2+2n}{n^2+2n+1}\right)^{n+1} \cdot \frac{n+2}{n+1}   \]\\
    \[   = \quad \left(\frac{n^2+2n}{n^2+2n+1}\right)^{n+1}  \cdot  \left(1+\frac{1}{n+1}\right) \]\\
    Nach oben abschätzen mit der Bernoulli-Ungleichung \(\frac{1}{n+1} > 0\)
    \[   \leq \quad \left(\frac{n^2+2n}{n^2+2n+1}\right)^{n+1}  \cdot  \left(1+\frac{1}{(n+1)^2}\right)^{n+1} \]\\
    \[  = \quad \left(\frac{n^2+2n}{n^2+2n+1}  \cdot  \frac{n^2+2n+2}{n^2+2n+1}\right)^{n+1} \]\\
    \[ a = (n+1)^2 \]\\
    \[\Rightarrow \frac{(a-1)\cdot (a+1)}{a^2} = \frac{a^2+a-a-1}{a^2} = \frac{a^2 -1}{a^2} \leq 1     \]
    
    \( {(b_n)}_n \) ist damit monoton fallend \\

    Zu zeigen \({(b_n)}_n\) ist nach unten beschränkt:
    Abschätzen nach unten mit Bernoulli-Ungleichung \({(1+x)}^n \geq 1 + nx\)
    \[ 
        b_n := { \left( 1 + \frac{1}{n} \right) }^n >
        1 + (n+1) \frac{1}{n} = 
        1 + \frac{n}{n} + \frac{1}{n} =
        2 + \frac{1}{n} > 2 > 0
    \]
    Damit ist \({(b_n)}_n\) nach unten beschränkt und da auch monoton fallend, konvergent. \\
    
    Zu zeigen \({(a_n)}_n\) ist nach unten beschränkt: \\
    Dafür \(a_n < b_n\)
    \[ { \left( 1 + \frac{1}{n} \right) }^n < { \left( 1 + \frac{1}{n} \right) }^{n+1} \]
    Da \({(b_n)}_n\) nach unten beschränkt und monoton fallend ist, ist \(a_n < b_0\).
    Damit ist \({(a_n)}_n\) nach oben beschränkt und da auch auch monoton wachsend, konvergent. \\

    Zu zeigen \({(a_n)}_n\) und \({(b_n)}_n\) konvergieren gegen den gleichen Grenzwert. \\
    Equivalent ist: Zu zeigen die Folge \({(c_n)}_n\) \(c_n = a_n - b_n\) ist eine Nullfolge.
    \begin{align*}
        c_n &= { \left( 1 + \frac{1}{n} \right) }^n - { \left( 1 + \frac{1}{n} \right) }^{n+1} \\
        &= { \left( 1 + \frac{1}{n} \right) }^n \left( 1 - \left( 1 + \frac{1}{n} \right) \right) \\
        &= { \left( 1 + \frac{1}{n} \right) }^n \left( \frac{1}{n} \right) \\
        &= a_n \frac{1}{n}
    \end{align*}
    Da wir wissen, dass \({(a_n)}_n\) konvergiert und \(\frac{1}{n}\) gegen 0 konvergiert, folgt, dass \({(c_n)_n}\)
    eine Nullfolge ist. \({(a_n)}_n\) und \({(b_n)}_n\) konvergieren also gegen den gleichen Wert.
    
    \section*{A 5.3}
    \begin{enumerate}[ label= (\roman*) ]
        \item \( \sum_{n=1}^{\infty} {\left( \frac{n}{n+1} \right)}^{n^2} \) \\
        Sei \(q \in \mathbb{R}\), \(0 < q < 1\) \\
        \[ \text{Zu zeigen: } \sqrt[n]{|x_n|} < q \Leftrightarrow \sqrt[n]{ \left| {\left( \frac{n}{n+1} \right)}^{n^2} \right| } < q \]
        Wurzel umformen:
        \[ 
            \lim_{n \to \infty} \sqrt[n]{ {\left( \frac{n}{n+1} \right)}^{n^2} } =
            \lim_{n \to \infty} \sqrt[n]{ {\left( \frac{n}{n+1} \right)}^{n * n} } = 
            \lim_{n \to \infty} {\left( \frac{n}{n+1} \right)}^{n}
        \]
        Bruch umformen:
        \[
            \lim_{n \to \infty} {\left( \frac{n}{n+1} \right)}^{n} =
            \lim_{n \to \infty} {\left( \frac{n}{n} \frac{1}{1+ \frac{1}{n}} \right)}^{n} =
            \lim_{n \to \infty} {\left( \frac{1}{1+ \frac{1}{n}} \right)}^{n} = 
            \lim_{n \to \infty} \frac{1}{ { 1 + \frac{1}{n} }^{n} } = \\
            \lim_{n \to \infty} \frac{1}{ a_n } =
            \frac{1}{e} = q < 1
        \]
        Weil \(\sqrt[n]{|x_n|} < q\) gegen \(\frac{1}{e}\) konvergiert,
        folgt daraus, dass es ein \(N \in \mathbb{N}\) gibt, sodass für alle \(n \geq \mathbb{N} \) \(\sqrt[n]{|x_n|} < q\) gilt.
        Die Vorraussetzungen für das Wurzelkriterium sind also erfüllt.
        Die Reihe \( \sum_{n=1}^{\infty} {\left( \frac{n}{n+1} \right)}^{n^2} \) konvergiert.

        \item ?

        \item \(\sum_{n=1}^{\infty} \frac{2^n}{n^n}\)
        Sei \(q \in \mathbb{R}\), \(0 < q < 1\) \\
        \[ \text{Zu zeigen: } \sqrt[n]{|x_n|} < q \Leftrightarrow \sqrt[n]{ \left| \frac{2^n}{n^n} \right| } < q \]
        Wurzel umformen, \(x_n\) ist immer positiv:
        \[
            \lim_{n \to \infty} \sqrt[n]{ \left| \frac{2^n}{n^n} \right| } 
            = \lim_{n \to \infty} \sqrt[n]{ \frac{2^n}{n^n} } 
            = \lim_{n \to \infty} \frac{2}{n} = 0 < q
        \]
        Weil \(\sqrt[n]{|x_n|} < q\) gegen \(0\) konvergiert,
        folgt daraus, dass es ein \(N \in \mathbb{N}\) gibt, sodass für alle \(n \geq \mathbb{N} \) \(\sqrt[n]{|x_n|} < q\) gilt.
        Die Vorraussetzungen für das Wurzelkriterium sind also erfüllt.
        Die Reihe \( \sum_{n=1}^{\infty} \frac{2^n}{n^n} \) konvergiert.

        \item \( \sum_{n=1}^{\infty} \frac{n!}{n^n} \) \\
        Sei \(q \in \mathbb{R}\), \(0 < q < 1\) \\
        \[ \text{Zu zeigen: } \frac{|x_{n+1}|}{|x_{n}|} < q \Leftrightarrow  \frac{  \left| \frac{(n+1)!}{{(n+1)}^{n+1}} \right| }{ \left| \frac{n!}{n^n} \right| } < q \]
        Betragsstriche weggelassen, da alle Terme immer positiv sind und vereinfachen:
        \begin{gather*}
            \lim_{n \to \infty} \frac{|\frac{(n+1)!}{{(n+1)}^{n+1}}|}{|\frac{n!}{n^n}|}
            = \lim_{n \to \infty} \frac{\frac{(n+1)!}{{(n+1)}^{n+1}}}{\frac{n!}{n^n}}
            = \lim_{n \to \infty} \frac{(n+1)! \: n^n}{ {(n+1)}^{n+1} n! }
            = \lim_{n \to \infty} \frac{(n+1)!}{n!} \frac{n^n}{{(n+1)}^{n+1}} \\
            = \lim_{n \to \infty} \frac{(n+1)!}{n!} \frac{n^n}{ {(n+1)}^{n} (n+1)}
            = \lim_{n \to \infty} \frac{(n+1)}{(n+1)} \frac{n^n}{{(n+1)}^{n}} 
            = \lim_{n \to \infty} {\left( \frac{n}{n+1} \right)}^n 
        \end{gather*}
        Bruch umformen
        \[
            \lim_{n \to \infty} {\left( \frac{n}{n+1} \right)}^{n} =
            \lim_{n \to \infty} {\left( \frac{n}{n} \frac{1}{1+ \frac{1}{n}} \right)}^{n} =
            \lim_{n \to \infty} {\left( \frac{1}{1+ \frac{1}{n}} \right)}^{n} = 
            \lim_{n \to \infty} \frac{1}{ { 1 + \frac{1}{n} }^{n} } =
            \lim_{n \to \infty} \frac{1}{ a_n } =
            \frac{1}{e} = q < 1
        \]
        Weil \(\frac{|x_{n+1}|}{|x_{n}|}\) gegen \(\frac{1}{e}\) konvergiert,
        folgt daraus, dass es ein \(N \in \mathbb{N}\) gibt, sodass für alle \(n \geq \mathbb{N} \) \(\frac{|x_{n+1}|}{|x_{n}|} < q\) gilt.
        Die Vorraussetzungen für das Quotientenkriterium sind also erfüllt.
        Die Reihe \( \sum_{n=1}^{\infty} \frac{n!}{n^n} \) konvergiert.

        \item \( \sum_{n=1}^{\infty} \frac{ 2^{2n+1} }{ 9^{n-2}} \) \\
        Umformen:
        \[
            \sum_{n=1}^{\infty} \frac{ 2^{2n+1} }{ 9^{n-2}}
            = \sum_{n=1}^{\infty} \frac{ 2^{2n} 2 }{ 9^{n} \frac{1}{9^2}}
            = \sum_{n=1}^{\infty} \frac{ 4^{n} 2 }{ 9^{n} \frac{1}{9^2}}+
            = \sum_{n=1}^{\infty} 162 \frac{ 4^{n} }{ 9^{n} }
            = 162 \sum_{n=1}^{\infty} \frac{ 4^{n} }{ 9^{n} }
         \]
         Hierbei handelt es sich um die geometrische Reihe, diese Reihe konvergiert für hier \(q = \frac{4}{9}\) gegen \(\frac{1}{1-q}\) = 291,6.
         Da diese Reihe aber nicht bei 0, sondern bei 1 startet müssen wir noch den ersten Term abziehen \(291,6 - 162*1 = 129,6\)
         Die Reihe \( \sum_{n=1}^{\infty} \frac{ 2^{2n+1} }{ 9^{n-2}} \) konvergiert also gegen \(129,6\)
    \end{enumerate}

\end{document}