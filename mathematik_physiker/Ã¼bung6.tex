\documentclass{article}

\usepackage[a4paper, top=2cm, bottom=3cm]{geometry}
\usepackage{babel}[german]
\usepackage{booktabs}
\usepackage{mathtools}
\usepackage{amssymb}
\usepackage{enumitem}
\usepackage{amsmath}

\date{26.11.2021}
\title{Aufgabenblatt 6, Mathematik für Physiker 1}
\author{Florian Adamczyk, Finn Wagner}

\begin{document}
    \maketitle

    \section*{A6.1 Berechnen Sie die folgenden Reihenwerte}
    \textbf{Hinweis:} Sie dürfen benutzen, dass \( \sum_{n=1}^{\infty} \frac{1}{n^2} = \frac{\pi^2}{6} \)

    \begin{enumerate}[ label = (\roman*) ]
        
        \item \[\sum_{n=1}^{\infty} \frac{1}{ {(2n)}^2 } \] 
        \[ \sum_{n=1}^{\infty} \frac{1}{ {(2n)}^2 } = \sum_{n=1}^{\infty} \frac{1}{ 4 n^2 } = \frac{1}{4} \sum_{n=1}^{\infty} \frac{1}{ n^2 } \]
        Ersetzen mit dem Hinweis:
        \[\frac{1}{4} \sum_{n=1}^{\infty} \frac{1}{ n^2 } = \frac{1}{4} \frac{\pi^2}{6} = \frac{\pi^2}{32} \]

        \item \[ \sum_{n=1}^{\infty} \frac{1}{ {(2n + 1)}^2 } \]
        Die Folgenglieder der Reihe \( \sum_{n=1}^{n} \frac{1}{n^2} \) sind die Kehrwerte der Quadrate aller natürlichen Zahlen.
        \( \frac{1}{1^2}, \frac{1}{2^2}, \frac{1}{3^2}, \frac{1}{4^2}, \ldots \) \\
        Die Folgenglieder der Reihe \( \sum_{n=1}^{n} \frac{1}{ {(2n)}^2 } \) sind die Kehrwerte der Quadrate aller geraden natürlichen Zahlen.
        \( \frac{1}{2^2}, \frac{1}{4^2}, \frac{1}{6^2}, \frac{1}{8^2}, \ldots \) \\
        Die Folgenglieder der Reihe \( \sum_{n=1}^{n} \frac{1}{ {(2n + 1)}^2 } \) sind die Kehrwerte der Quadrate aller ungeraden natürlichen Zahlen ab 3.
        \( \frac{1}{3^2}, \frac{1}{5^2}, \frac{1}{7^2}, \frac{1}{9^2}, \ldots \) \\
        Addiert man die beiden Reihen \( \sum_{n=1}^{n} \frac{1}{ {(2n)}^2 } \) und \( \sum_{n=1}^{n} \frac{1}{ {(2n + 1)}^2 } \),
        so erhält eine Summe aller ihrer Folgenglieder,
        \( \frac{1}{2^2}, \frac{1}{3^2}, \frac{1}{4^2}, \frac{1}{5^2}, \frac{1}{6^2}, \frac{1}{7^2}, \frac{1}{8^2}, \ldots, \frac{1}{ {( 2n+1 )}^2 } \) \\
        Diese Summe ist äquivalent zu \( \sum_{n=1}^{ 2n+1 } \frac{1}{n^2} - 1 \), da das erste Element, die 1 fehlt.
        \[ \sum_{n=1}^{ 2n+1 } \frac{1}{n^2} - 1 = \sum_{n=1}^{n} \frac{1}{ {(2n)}^2 } + \sum_{n=1}^{n} \frac{1}{ {(2n + 1)}^2 } \]
        Bildet man nun den Grenzübergang \( n \to \infty \), so folgt:
        \[ \sum_{n=1}^{ \infty } \frac{1}{n^2} - 1 = \sum_{n=1}^{ \infty } \frac{1}{ {(2n)}^2 } + \sum_{n=1}^{ \infty } \frac{1}{ {(2n + 1)}^2 } \]
        Ersetzen wir nun die bereits bekannten Grenzwerte:
        \[ \frac{\pi^2}{6} - 1 =  \frac{\pi^2}{32} + \sum_{n=1}^{ \infty } \frac{1}{ {(2n + 1)}^2 } \ | \ -\frac{\pi^2}{32}  \]
        \[ \Leftrightarrow \sum_{n=1}^{ \infty } \frac{1}{ {(2n + 1)}^2 } = \frac{ \pi^2 }{8} - 1 \]

    \end{enumerate}
   
    \section*{A6.3 Beweisen Sie das \( \mathbb{C} \) ein Körper ist}
    Sei eine komplexe Zahl gegeben als \( (x, y) \in \mathbb{R}^2 \) \\
    Die Addition (+) mit \( (x_1, y_1), (x_2, y_2) \in \mathbb{R}^2 \) sei definiert als: \\
    \[ (x_1, y_1) + (x_2, y_2) := (x_1 + x_2, \ y_1 + y_2) \]
    Die Mulitplikation (\( \cdot \)) mit \( (x_1, y_1), (x_2, y_2) \in \mathbb{R}^2 \) sei definiert als: \\
    \[ (x_1, y_1) \cdot (x_2, y_2) := (x_1 x_2 - y_1 y_2, \ x_1 y_2 + x_2 y_1) \]
    \textbf{Zu zeigen: \( (\mathbb{R}^2, +) \) ist eine abelsche Gruppe } \\
    Wir weisen nun alle Eigenschaften einer Gruppe für \( (\mathbb{R}^2, +) \) nach:

    \begin{enumerate}[ label = (\alph*) ]

        \item Zu zeigen: Es gilt das Assoziativgesetz: \\
        Seien \( (x_1, y_1), (x_2, y_2), (x_3, y_3) \in \mathbb{R}^2 \) \\
        Es soll also gelten:
        \[ (x_1, y_1) + ((x_2, y_2) + (x_3, y_3)) =^! ((x_1, y_1) +  (x_2, y_2)) + (x_3, y_3) \]
        Wir setzen die Definition der Addition für die hinteren beiden Terme ein:
        \[ (x_1, y_1) + ((x_2, y_2) + (x_3, y_3)) = (x_1, y_1) + (x_2 + x_3, y_2 + y_3) \]
        Wir setzen erneut die die Definition der Addition ein:
        \[ (x_1, y_1) + (x_2 + x_3, y_2 + y_3) = ( x_1 + ( x_2 + x_3 ) , y_1 + ( y_2 + y_3) ) \]
        Wir klammern die Terme um, da die reelen Zahlen ein Körper sind und für sie das Assoziativgesetz gilt:
        \[ ( x_1 + ( x_2 + x_3 ) , y_1 + ( y_2 + y_3) ) = ( ( x_1 + x_2 ) + x_3, (y_1 + y_2) + y_3 ) \]
        Wir setzen die Definition der Addition rückwärts ein:
        \[ ( ( x_1 + x_2 ) + x_3, (y_1 + y_2) + y_3 ) = ( x_1 + x_2, y_1 + y_2 ) + (x_3, y_3) \]
        Wir setzen die Definition der Addition erneut rückwärts ein:
        \[ ( x_1 + x_2, y_1 + y_2 ) + (x_3, y_3) = ((x_1, y_1) +  (x_2, y_2)) + (x_3, y_3) \]
        Damit gilt für die Gruppe \( (\mathbb{R}^2, +) \) das Assoziativgesetz

        \item Zu zeigen: Es existiert ein neutrales Element
        \[ \exists \ n \in \mathbb{R}^2 : \forall \ a \in \mathbb{R}^2: a + n = n + a = a \]
        Seien \( (0, 0), \ (x, y) \in \mathbb{R}^2 \) \\
        \[ (0, 0) + (x, y) = (x + 0, y + 0) = (x , y) \]
        \[ (x, y) + (0, 0) = (0 + x, 0 + y) = (x , y) \]
        Wir haben gezeigt, dass ein neutrales Element existiert. Dieses Element ist \( (0, 0) \)

        \item Zu zeigen: Es existiert ein inverses Element
        \[ \forall \ a \in \mathbb{R}^2 : \exists \ b \in \mathbb{R}^2 : a + b = b + a = n \]
        Seien \( (x, y), (-x, -y) \in \mathbb{R}^2 \)
        \[ (x, y) + (-x, -y) = (x - x, y - y) = (0, 0) \]
        \[ (-x, -y) + (x, y) = (-x + x, -y + y) = (0, 0) \]
        Wir haben gezeigt, dass zu jedem Element \((x , y)\) ein inverses Element \((-x, -y)\) existiert.

        \item Zu zeigen: Die Gruppe ist abelsch: \\
        \( \forall a, b \in \mathbb{R}^2: a + b = b + a\) \\
        Seien \( (x_1, y_1), \ (x_2, y_2) \in \mathbb{R}^2 \) \\
        Wir setzen die Definition der Addition ein:
        \[ (x_1, y_1) + (x_2, y_2)  = (x_1 + x_2 , y_1 + y_2) \]
        Wir ändern die Reihenfolge der Variablen, da die reelen Zahlen ein Körper sind und für sie das Kommutativgesetz gilt:
        \[ (x_1 + x_2 , y_1 + y_2) = (x_2 + x_1 , y_2 + y_1) \]
        Wir setzen die Definition der Addition rückwärts ein:
        \[ (x_2 + x_1 , y_2 + y_1) = (x_2, y_2) + (x_1, y_1) \]
        Damit ist die Gruppe abelsch. Es gilt das Kommutativgesetz.

    \end{enumerate}

    \textbf{Zu zeigen: \( (\mathbb{R}^2 \backslash \{(0, 0)\}, \cdot) \) ist eine abelsche Gruppe }
    Wir weisen nun alle Eigenschaften einer Gruppe für \( (\mathbb{R}^2 \backslash \{(0, 0)\}, \cdot) \) nach:

    \begin{enumerate}[ label = (\alph*) ]

        \item Zu zeigen: Es gilt das Assoziativgesetz: \\
        Seien \( (x_1, y_1), (x_2, y_2), (x_3, y_3) \in \mathbb{R}^2 \) \\
        Es soll also gelten:
        \[ (x_1, y_1) \cdot ((x_2, y_2) \cdot (x_3, y_3)) =^! ((x_1, y_1) \cdot  (x_2, y_2)) \cdot (x_3, y_3) \]
        Wir setzen die Definition der Mulitplikation für die hinteren beiden Terme ein:
        \[ (x_1, y_1) \cdot ((x_2, y_2) \cdot (x_3, y_3)) = (x_1, y_1) \cdot (x_2 x_3 - y_2 y_3, x_2 y_3 + x_3 y_2) \]
        Wir setzen erneut die die Definition der Mulitplikation ein:
        \begin{gather*}
            (x_1, y_1) \cdot (x_2 x_3 - y_2 y_3, x_2 y_3 + x_3 y_2) = \\
            ( x_1 (x_2 x_3 - y_2 y_3) - y_1 (x_2 y_3 + x_3 y_2) , x_1 (x_2 y_3 + x_3 y_2) + (x_2 x_3 - y_2 y_3) y_1 )
        \end{gather*}    
        Wir klammern die Terme um, da die reelen Zahlen ein Körper sind und für sie Assoziativ- und Distributivgesetz gelten:
        \begin{gather*}
            ( x_1 (x_2 x_3 - y_2 y_3) - y_1 (x_2 y_3 + x_3 y_2) , x_1 (x_2 y_3 + x_3 y_2) + (x_2 x_3 - y_2 y_3) y_1 ) = \\
            ( x_1 x_2 x_3 - x_1 y_2 y_3 - y_1 x_2 y_3 - y_1 x_3 y_2, x_1 x_2 y_3 + x_1 x_3 y_2 + x_2 x_3 y_1 - y_2 y_3 y_1) \\
            ( x_1 x_2 x_3 - y_1 y_2 x_3 - x_1 y_2 y_3 - x_2 y_1 y_3, x_1 x_2 y_3 - y_1 y_2 y_3 + x_3 x_1 y_2 + x_3 x_2 y_1) \\
            ( (x_1 x_2 - y_1 y_2) x_3 - (x_1 y_2 + x_2 y_1) y_3, (x_1 x_2 - y_1 y_2) y_3 + x_3 (x_1 y_2 + x_2 y_1) )
        \end{gather*}
        Wir setzen die Definition der Mulitplikation rückwärts ein:
        \begin{gather*}
            ( (x_1 x_2 - y_1 y_2) x_3 - (x_1 y_2 + x_2 y_1) y_3, (x_1 x_2 - y_1 y_2) y_3 + x_3 (x_1 y_2 + x_2 y_1) ) = \\
            (x_1 x_2 - y_1 y_2, x_1 y_2 + x_2 y_1) \cdot (x_3, y_3)
        \end{gather*}
        Wir setzen die Definition der Mulitplikation erneut rückwärts ein:
        \[ (x_1 x_2 - y_1 y_2, x_1 y_2 + x_2 y_1) \cdot (x_3, y_3) = ((x_1, y_1) \cdot (x_2, y_2)) \cdot (x_3, y_3) \]
        Damit gilt für die Gruppe \( (\mathbb{R}^2 \backslash \{(0, 0)\}, \cdot) \) das Assoziativgesetz

        \item Zu zeigen: Es existiert ein neutrales Element
        \[ \exists \ n \in \mathbb{R}^2 \backslash \{(0, 0)\} : \forall \ a \in \mathbb{R}^2 \backslash \{(0, 0)\} : a + n = n + a = a \]
        Seien \( (1, 0), \ (x, y) \in \mathbb{R}^2 \backslash \{(0, 0)\} \) \\
        \[ (1, 0) \cdot (x, y) = (1 \cdot x - 0 \cdot y, 1 \cdot y + x \cdot 0) = (x , y) \]
        \[ (x, y) \cdot (1, 0) = (x \cdot 1 - y \cdot 0, x \cdot 0 + 1 \cdot y) = (x , y) \]
        Wir haben gezeigt, dass ein neutrales Element existiert. Dieses Element ist \( (1, 0) \)

        \item Zu zeigen: Es existiert ein inverses Element
        \[ \forall \ a \in \mathbb{R}^2 \backslash \{(0, 0)\}: \exists \ b \in \mathbb{R}^2 \backslash \{(0, 0)\}: a + b = b + a = n \]
        Seien \( (x, y), ( \frac{x}{ x^2 + y^2 }, - \frac{y}{ x^2 + y^2 } ) \in \mathbb{R}^2 \backslash \{(0, 0)\} \)
        \begin{gather*}
            (x, y) \cdot ( \frac{x}{ x^2 + y^2 }, - \frac{y}{ x^2 + y^2 } ) = \\
            ( x \cdot \frac{x}{ x^2 + y^2 } - y \cdot - \frac{y}{ x^2 + y^2 }, x \cdot - \frac{y}{ x^2 + y^2 } + \frac{x}{ x^2 + y^2 } \cdot y ) = \\
            ( \frac{ x^2 + y^2 }{ x^2 + y^2 }, \frac{ -xy + xy }{ x^2 + y^2} ) = (1, 0)
        \end{gather*}
        Weiterhin:
        \begin{gather*}
            ( \frac{x}{ x^2 + y^2 }, - \frac{y}{ x^2 + y^2 } ) \cdot (x, y) = \\
            ( \frac{x}{ x^2 + y^2 } \cdot x - - \frac{y}{ x^2 + y^2 } \cdot y, \frac{x}{ x^2 + y^2 } \cdot y + x \cdot - \frac{y}{ x^2 + y^2 } ) = \\
            ( \frac{ x^2 + y^2 }{ x^2 + y^2 }, \frac{ -xy + xy }{ x^2 + y^2} ) = (1, 0)
        \end{gather*}
        Wir haben gezeigt, dass zu jedem Element \((x , y)\) ein inverses Element \(( \frac{x}{ x^2 + y^2 }, - \frac{y}{ x^2 + y^2 } )\) existiert.

        \item Zu zeigen: Die Gruppe ist abelsch: \\
        \( \forall a, b \in \mathbb{R}^2 \backslash \{(0, 0)\}: a + b = b + a\) \\
        Seien \( (x_1, y_1), \ (x_2, y_2) \in \mathbb{R}^2 \backslash \{(0, 0)\} \) \\
        Wir setzen die Definition der Mulitplikation ein:
        \[ (x_1, y_1) \cdot (x_2, y_2)  = (x_1 x_2 - y_1 y_2, \ x_1 y_2 + x_2 y_1) \]
        Wir ändern die Reihenfolge der Variablen, da die reelen Zahlen ein Körper sind und für sie das Kommutativgesetz gilt:
        \[ (x_1 x_2 - y_1 y_2, \ x_1 y_2 + x_2 y_1) = (x_2 x_1 - y_2 y_1, \ x_2 y_1 + x_1 y_2) \]
        Wir setzen die Definition der Mulitplikation rückwärts ein:
        \[ (x_2 x_1 - y_2 y_1, \ x_2 y_1 + x_1 y_2) = (x_2, y_2) \cdot (x_1, y_1) \]
        Damit ist die Gruppe abelsch. Es gilt das Kommutativgesetz.

    \end{enumerate}

    \textbf{Zu zeigen: Es gilt das Distributivgesetz:}
    \[ \forall \ (x_1, y_1), (x_2, y_2), (x_3, y_3) \in \mathbb{R}^2: (x_1, y_1) \cdot ((x_2, y_2) + (x_3, y_3)) = (x_1, y_1) \cdot (x_2, y_2) + (x_1, y_1) \cdot (x_3, y_3) \]
    Wir setzen die Definition der Addition ein:
    \[ (x_1, y_1) \cdot ((x_2, y_2) + (x_3, y_3)) = (x_1, y_1) \cdot (x_2 + x_3, y_2 + y_3) \]
    Wir setzen zweimal die Definition der Mulitplikation ein:
    \[ (x_1, y_1) \cdot (x_2 + x_3, y_2 + y_3) = ( x_1 (x_2 + x_3) - y_1 (y_2 + y_3), x_1 (y_2 + y_3) + (x_2 + x_3) y_1 ) \]
    Wir klammern die Terme um, da die reelen Zahlen ein Körper sind und für sie Assoziativ- und Distributivgesetz gelten:
    \begin{gather*}
        ( x_1 (x_2 + x_3) - y_1 (y_2 + y_3), x_1 (y_2 + y_3) + (x_2 + x_3) y_1 ) = \\
        ( x_1 x_2 + x_1 x_3 - y_1 y_2 - y_1 y_3, x_1 y_2 + x_1 y_3 + x_2 y_1 + x_3 y_1 ) = \\
        ( (x_1 x_2 - y_1 y_2) + ( x_1 x_3 - y_1 y_3), (x_1 y_2 + x_2 y_1) + (x_1 y_3 + x_3 y_1) )
    \end{gather*}
    Wir setzen die Definition der Addition rückwärts ein:
    \begin{gather*}
        ( (x_1 x_2 - y_1 y_2) + ( x_1 x_3 - y_1 y_3), (x_1 y_2 + x_2 y_1) + (x_1 y_3 + x_3 y_1) ) = \\
         (x_1 x_2 - y_1 y_2, x_1 y_2 + x_2 y_1 ) + ( x_1 x_3 - y_1 y_3, x_1 y_3 + x_3 y_1 )
    \end{gather*}
    Wir setzen zweimal die Definition der Mulitplikation rückwärts ein:
    \begin{gather*}
         (x_1 x_2 - y_1 y_2, x_1 y_2 + x_2 y_1 ) + ( x_1 x_3 - y_1 y_3, x_1 y_3 + x_3 y_1 ) = \\
         (x_1, y_1) \cdot (x_2, y_2) + (x_1, y_1) \cdot (x_3, y_3)
    \end{gather*}
    Damit gilt das Distributivgesetz. \\

    Alle Bedingunge sind erfüllt. \( (\mathbb{R}^2 \backslash \{(0, 0)\}, \cdot) \) ist ein Körper. \\
    \( \mathbb{C} \) ist ein Körper.

    
    \section*{A6.3 Beweisen sie die folgenden Gleichungen für \(w, c \in \mathbb{C} \). }
    
    \begin{enumerate}[ label = (\alph*) ]
        \item \( \overline{ w+z} = \overline{w} + \overline{z} \) und \( \overline{wz} = \overline{wz} \) \\
        \begin{enumerate}

            \item \( \overline{ w+z} =^! \overline{w} + \overline{z} \) \\
            Seien die komplexen Zahlen \(w, z\) auch dargestellt als \(w = x_1 + iy_1 \) und \(z = x_2 + iy_2 \) \\
            Wir setzen die Definitionen der komplexen Zahlen \(w, z\) ein:
            \[ \overline{ w+z} = \overline{ x_1 + iy_1 + x_2 + iy_2} = \overline{ (x_1 + x_2) + i (y_1 + y_2) } \]
            Das komplex Konjugierte unterscheid sich von der komplexen Zahl in einem negierten Imaginärteil:
            \[ \overline{ (x_1 + x_2) + i (y_1 + y_2) } = (x_1 + x_2) - i (y_1 + y_2) = (x_1 - iy_1) + (x_2 -iy_2) \]
            Wir sehen, dass hier die Summe der komplex Konjugierten von \(w\) und \(z\) steht:
            \[ (x_1 - iy_1) + (x_2 -iy_2) = \overline{w} + \overline{z} \] \\

            Außerdem gilt es zu beweisen dass \( \overline{wz} =^! \overline{wz} \) gilt. \\
            \[ \overline{wz} = \overline{wz} \ | \ \backslash \ \overline{wz}\]
            \[ \frac{\overline{wz}}{\overline{wz}} = \frac{\overline{wz}}{\overline{wz}}\]
            \[ 1 = 1\]
            Dies ist eine wahre Aussage. Es gilt \( \overline{wz} = \overline{wz} \)
        
            \item \( Re(z) = \frac{1}{2} ( z + \overline{z} ) \) und \( Im(z) = \frac{1}{2i} ( z - \overline{z} ) \) \\
            Zuerst \( Re(z) =^! \frac{1}{2} ( z + \overline{z} ) \) \\
            Sei die komplexe Zahl \(z\) auch dargestellt als \(z = x + iy \) \\
            Wir setzen die Definition von \(z\) ein:
            \[ \frac{1}{2} ( z + \overline{z} ) = \frac{1}{2} ( (x + iy) + (x -iy))  = \frac{1}{2} (2x) = x = Re(z) \]

            Nun \( Im(z) =^! \frac{1}{2i} ( z - \overline{z} ) \) \\
            Sei die komplexe Zahl \(z\) auch dargestellt als \(z = x + iy \) \\
            Wir setzen die Definition von \(z\) ein:
            \[ \frac{1}{2i} ( z - \overline{z} ) = \frac{1}{2i} ( (x + iy) - (x -iy))  = \frac{1}{2i} (2iy) = y = Im(z) \]

            \item \( |z| \geq 0 \) und \( |z| = 0 \Leftrightarrow z = 0\) \\
            Zuerst \( |z| \geq^! 0 \) \\
            Sei die komplexe Zahl \(z\) auch dargestellt als \(z = x + iy \) \\
            \[ |z| = | x + iy | = \sqrt{ x^2 + y^2 } \]
            Da das Quadrat einer Zahl immer größer gleich 0 ist, die Summe zweier Quadrate ebenfalls positiv.
            Die Wurzel einer positiven Zahl ist positiv. \\
            Es gilt:
            \[ |z| = \sqrt{ x^2 + y^2 } \geq 0 \]

            Nun \( |z| = 0 \Leftrightarrow^! z = 0\) \\
            Sei die komplexe Zahl \(z\) auch dargestellt als \(z = x + iy \) \\
            \[ |z| = | x + iy | = \sqrt{x^2 + y^2 } = 0 \]
            \begin{gather*}
                \sqrt{x^2 + y^2 } = 0 \ | \ Quadrieren \\
                x^2 + y^2 = 0 \ | \ -y^2 \\
                x^2 = -y^2
            \end{gather*}
            Die Gleichung \( x^2 = -y^2 \) ist nur für \( x=y=0\) erfüllt, damit ist \(z = 0\).

            \item \( |\overline{z}| = |z| \)  und \( z \cdot \overline{z} = {|z|}^2 \) \\
            Zuerst \( |\overline{z}| = |z| \) \\
            Sei die komplexe Zahl \(z\) auch dargestellt als \(z = x + iy \) \\
            Das komplex Konjugierte von \(z\) ist damit \(\overline{z} = x - iy\)
            Wir setzen in die Gleichung ein, und führen den Betrag aus:
            \[ |\overline{z}| = |x - iy| = \sqrt{ x^2 + {(-y)}^2 } = \sqrt{ x^2 + y^2 } \]
            Dies ist aber gerade der Betrag der Zahl \(z\):
            \[ \sqrt{ x^2 + y^2 } = |x + iy| = |z| \]

            Nun \( z \cdot \overline{z} = {|z|}^2 \) \\
            Wir setzen die Definition von \(z\) und \(\overline{z}\) ein und vereinfachen:
            \[ z \cdot \overline{z} = (x + iy) (x -iy) = x^2 - iyx + iyx -i^2y^2 \]
            Wir schreiben hier um mit \( a = {\sqrt{a}}^2 \)
            \[ x^2 + y^2 = {\sqrt{ x^2 + y^2}}^2 = {|z|}^2 \]
            Weil \( \sqrt{x^2 + y^2}\) der Betrag von \(z\) ist.

            \item \( |w \cdot z| = |w| \cdot |z| \) \\
            Seien die komplexen Zahlen \(w, z\) auch dargestellt als \(w = x_1 + iy_1 \) und \(z = x_2 + iy_2 \) \\
            Wir setzen die Definitionen der komplexen Zahlen \(w, z\) ein:
            \[ |w \cdot z| = | (x_1 + iy_1) (x_2 + iy_2) | = | x_1 x_2 + x_1 i y_2 + iy_1 x_2 + iy_1 iy_2 |  = | (x_1 x_2 - y_1 y_2) + i( x_1 y_2 +  y_1 x_2) | \]
            Wir formen so um, das die komplexe Zahl in Realteil und Imaginärteil gegliedert ist:
            \[ | x_1 x_2 + x_1 i y_2 + iy_1 x_2 + iy_1 iy_2 | = | (x_1 x_2 - y_1 y_2) + i( x_1 y_2 +  y_1 x_2) | \]
            Wir setzen die Definition des komplexen Betrags ein:
            \[ | (x_1 x_2 - y_1 y_2) + i( x_1 y_2 +  y_1 x_2) | = \sqrt{ {(x_1 x_2 - y_1 y_2)}^2 + {( x_1 y_2 +  y_1 x_2)}^2 } \]
            Wir multiplizieren die Klammern aus:
            \begin{gather*}
                \sqrt{ {(x_1 x_2 - y_1 y_2)}^2 + {( x_1 y_2 +  y_1 x_2)}^2 } = \\
                \sqrt{ (x_1^2 x_2^2 -2 (x_1 x_2 - y_1 y_2) + y_1^2 y_2^2 ) + (x_1^2 y_2^2 + 2( x_1 y_2 +  y_1 x_2) + y_1^2 x_2^2) } 
            \end{gather*}
            Wir kürzen den Term \( (2 (x_1 x_2 - y_1 y_2)) \)
            \begin{gather*}
                \sqrt{ (x_1^2 x_2^2 -2 (x_1 x_2 - y_1 y_2) + y_1^2 y_2^2 ) + (x_1^2 y_2^2 + 2( x_1 y_2 +  y_1 x_2) + y_1^2 x_2^2) } = \\
                \sqrt{ x_1^2 x_2^2 + y_1^2 y_2^2 + x_1^2 y_2^2 + y_1^2 x_2^2 }
            \end{gather*}
            Wir faktorisieren den Ausdruck:
            \[ \sqrt{ x_1^2 x_2^2 + x_1^2 y_2^2 + y_1^2 x_2^2 + y_1^2 x_2^2 } = \sqrt{ (x_1^2 + y_1^2)  (x_2^2 + y_2^2) } \]
            Wir teilen auf zwei Wurzeln auf:
            \[ \sqrt{ (x_1^2 + y_1^2)  (x_2^2 + y_2^2) } = \sqrt{ x_1^2 + y_1^2} \sqrt{ x_2^2 + y_2^2 } \]
            Wir setzen die Definition des komplexen Betrags rückwärts ein:
            \[ \sqrt{ x_1^2 + y_1^2 } \sqrt{ x_2^2 + y_2^2 } = |x_1 + iy_1| | x_2 + iy_2| \]
            Wir setzen die Definitionen der komplexen Zahlen \(w, z\) rückwärts ein:
            \[ |x_1 + iy_1| | x_2 + iy_2| = |w| |z| \]

            \item \( |w + z| \leq |w| + |z| \) und \( |w-z| \geq |w||-|z|| \) \\
            Zuerst \( |w + z| \leq^! |w| + |z| \) \\
            Seien die komplexen Zahlen \(w, z\) auch dargestellt als \(w = x_1 + iy_1 \) und \(z = x_2 + iy_2 \) \\
            Wir setzen die Definitionen der komplexen Zahlen \(w, z\) ein:
            \begin{align*}
                |w + z| &\leq |w| + |z| \\
                \Leftrightarrow |(x_1 + iy_1) + (x_2 + iy_2) | &\leq |x_1 + iy_1| + |x_2 + iy_2| \\ 
                \Leftrightarrow |(x_1 + x_2) + i(y_1 + y_2) | &\leq |x_1 + iy_1| + |x_2 + iy_2| \\
            \end{align*}
            Wir setzen die Definition des komplexen Betrags ein und quadrieren:
            \begin{align*}
                |(x_1 + x_2) + i(y_1 + y_2) | &\leq |x_1 + iy_1| + |x_2 + iy_2| \\
                \Leftrightarrow \sqrt{ {(x_1 + x_2)}^2 + {(y_1 + y_2)}^2 } &\leq \sqrt{ x_1^2 + y_1^2} + \sqrt{ x_2^2 + y_2^2} \ | \ Quadrieren \\
                \Leftrightarrow {(x_1 + x_2)}^2 + {(y_1 + y_2)}^2 &\leq {\left( \sqrt{ x_1^2 + y_1^2} + \sqrt{ x_2^2 + y_2^2} \right)}^2 \ | \ Ausmultiplizieren \\
                \Leftrightarrow x_1^2 + 2 x_1 x_2 + x_2^2 + y_1^2 + 2 y_1 y_2 + y_2^2 &\leq \\ 
                x_1^2 + y_1^2 + 2 \sqrt{ x_1^2 + y_1^2} & \sqrt{ x_2^2 + y_2^2} + x_2^2 + y_2^2 \\
            \end{align*}
            Wir können die Terme \(x_1^2\), \(x_2^2\), \(y_1^2\) und \(y_2^2\) kürzen:
            \begin{align*}
                x_1^2 + 2 x_1 x_2 + x_2^2 + y_1^2 + 2 y_1 y_2 + y_2^2 &\leq \\ 
                x_1^2 + y_1^2 + 2 \sqrt{ x_1^2 + y_1^2} & \sqrt{ x_2^2 + y_2^2} + x_2^2 + y_2^2 \\
                \Leftrightarrow 2 x_1 x_2  + 2 y_1 y_2 \leq 2 \sqrt{ (x_1^2 + y_1^2) (x_2^2 + y_2^2) }
            \end{align*}
            Nun quadrieren wir erneut und multiplizieren aus:
            \begin{align*}
                2 x_1 x_2  + 2 y_1 y_2 &\leq 2 \sqrt{ (x_1^2 + y_1^2) (x_2^2 + y_2^2) } \ | \ Quadrieren \\
                \Leftrightarrow  {( 2 x_1 x_2  + 2 y_1 y_2 )}^2 &\leq 4 (x_1^2 + y_1^2) (x_2^2 + y_2^2) \ | \ Ausmultiplizieren \\
                \Leftrightarrow  4 x_1^2 x_2^2 + 8 x_1 x_2 y_1 y_2 + 4 y_1^2 y_2^2 & \leq 4 x_1^2 x_2^2 + 4 x_1^2 y_2^2 + 4 y_1^2 x_2^2 + 4 y_1^2 y_2^2
            \end{align*}
            Wir können die Terme \(4 x_1^2 x_2^2\) und \(4 y_1^2 y_2^2\) kürzen:
            \begin{align*}
               4 x_1^2 x_2^2 + 8 x_1 x_2 y_1 y_2 + 4 y_1^2 y_2^2 & \leq 4 x_1^2 x_2^2 + 4 x_1^2 y_2^2 + 4 y_1^2 x_2^2 + 4 y_1^2 y_2^2 \\
               \Leftrightarrow 8 x_1 x_2 y_1 y_2 & \leq 4 x_1^2 y_2^2 + 4 y_1^2 x_2^2 \ | \ - 8 x_1 x_2 y_1 y_2 \\
               \Leftrightarrow 0 & \leq 4 x_1^2 y_2^2 - 8 x_1 x_2 y_1 y_2 + 4 y_1^2 x_2^2 \ | \ \backslash 4 \\
               \Leftrightarrow 0 & \leq x_1^2 y_2^2 - 2 x_1 x_2 y_1 y_2 + y_1^2 x_2^2\ | \ Faktorisieren \\
               \Leftrightarrow 0 & \leq {(x_1 y_2 - y_1 x_2)}^2
            \end{align*}
            Das Quadrat einer Zahl ist immer größer gleich Null. Damit ist dies eine wahre Aussage.
            Es gilt also \( |w + z| \leq |w| + |z| \)

            Nun \( |w-z| \geq^! ||w|-|z|| \): \\
            Wir setzen die Definitionen der komplexen Zahlen \(w, z\) ein:
            \begin{align*}
                |w-z| \geq ||w|-|z|| \\
                \Leftrightarrow | x_1 + iy_1 - x_2 -iy_2 | & \geq ||x_1 + iy_1| - |x_2 + iy_2|| \\
                \Leftrightarrow | x_1 - x_2 + i(y_1 -y_2) | & \geq ||x_1 + iy_1| - |x_2 + iy_2|| \\
            \end{align*}
            Wir setzen die Definition des komplexen Betrags ein und quadrieren:
            \begin{align*}
                | x_1 - x_2 + i(y_1 -y_2) | & \geq ||x_1 + iy_1| - |x_2 + iy_2|| \\
                \Leftrightarrow \sqrt{ {(x_1 - x_2)}^2 + {( y_1 - y_2)}^2 } & \geq | \sqrt{ x_1^2 + y_1^2 } - \sqrt{ x_2^2 + y_2^2 } | \ | \ Quadrieren \\
                \Leftrightarrow {(x_1 - x_2)}^2 + {( y_1 - y_2)}^2 & \geq { \sqrt{ x_1^2 + y_1^2 } - \sqrt{ x_2^2 + y_2^2 } }^2 \ | \ Ausmultiplizieren \\
                \Leftrightarrow x_1^2 - 2 x_1 x_2 + x_2^2 + y_1^2 - 2 y_1 y_2 + y_2^2 & \geq \\
                (x_1^2 + y_1^2) - &2 \sqrt{ x_1^2 + y_1^2 } \sqrt{ x_2^2 + y_2^2 } + (x_2^2 + y_2^2)
            \end{align*}
            Wir können die Terme \(x_1^2\), \(x_2^2\), \(y_1^2\) und \(y_2^2\) kürzen:
            \begin{align*}
                x_1^2 - 2 x_1 x_2 + x_2^2 + y_1^2 - 2 y_1 y_2 + y_2^2 & \geq \\
                (x_1^2 + y_1^2) - &2 \sqrt{ x_1^2 + y_1^2 } \sqrt{ x_2^2 + y_2^2 } + (x_2^2 + y_2^2) \\
                \Leftrightarrow - 2 x_1 x_2 - 2 y_1 y_2 & \geq - 2 \sqrt{ x_1^2 + y_1^2 } \sqrt{ x_2^2 + y_2^2 }
            \end{align*}
            Wir dividieren durch \(-1\), dabei dreht sich die Umgleichung um:
            \begin{align*}
                2 x_1 x_2 + 2 y_1 y_2 & \leq 2 \sqrt{ x_1^2 + y_1^2 } \sqrt{ x_2^2 + y_2^2 }
            \end{align*}
            Nun quadrieren wir erneut und multiplizieren aus:
            \begin{align*}
                2 x_1 x_2 + 2 y_1 y_2 & \leq 2 \sqrt{ x_1^2 + y_1^2 } \sqrt{ x_2^2 + y_2^2 } \ | \ Quadrieren \\
                \Leftrightarrow {\left( 2 x_1 x_2 + 2 y_1 y_2 \right)}^2 & \leq 4 (x_1^2 + y_1^2) (x_2^2 + y_2^2) \ | \ Ausmultiplizieren \\
                \Leftrightarrow 4 x_1^2 x_2^2 + 8 x_1 x_2 y_1 y_2 + 4 y_1^2 y_2^2 & \leq 4 x_1^2 x_2^2 + 4 x_1^2 y_2^2 + 4 x_2^2 y_1^2 + 4 y_1^2 y_2^2
            \end{align*}
            Wir können die Terme \(4 x_1^2 x_2^2\) und \(4 y_1^2 y_2^2\) kürzen:
            \begin{align*}
                4 x_1^2 x_2^2 + 8 x_1 x_2 y_1 y_2 + 4 y_1^2 y_2^2 & \leq 4 x_1^2 x_2^2 + 4 x_1^2 y_2^2 + 4 x_2^2 y_1^2 + 4 y_1^2 y_2^2 \\
                \Leftrightarrow 8 x_1 x_2 y_1 y_2 & \leq 4 x_1^2 y_2^2 + 4 x_2^2 y_1^2 \ | \ - 8 x_1 x_2 y_1 y_2 \\
                \Leftrightarrow 0 & \leq 4 x_1^2 y_2^2 - 8 x_1 x_2 y_1 y_2 + 4 y_1^2 x_2^2 \ | \ \backslash 4 \\
                \Leftrightarrow 0 & \leq x_1^2 y_2^2 - 2 x_1 x_2 y_1 y_2 + y_1^2 x_2^2\ | \ Faktorisieren \\
                \Leftrightarrow 0 & \leq {(x_1 y_2 - y_1 x_2)}^2
            \end{align*}
            Das Quadrat einer Zahl ist immer größer gleich Null. Damit ist dies eine wahre Aussage.
            Es gilt also: \( |w-z| \geq^! ||w|-|z|| \) \\
        \end{enumerate}

        \section*{A6.4 Beweisen Sie die Konvergenz der folgenden Folgen und bestimmen Sie deren Grenzwert}
        \begin{enumerate}[ label = (\roman*) ]

            \item \(z_n := \frac{i^n}{3^n}\) \\
            Das Epsilonkriterium:
            \[\forall \: \epsilon > 0 \; \exists \; N_{\epsilon} \in \mathbb{N} : \forall n \geq N_{\epsilon}: \: |x_n - x| < \epsilon
                \text{ mit } \epsilon \in \mathbb{R} \text{ und } n \in \mathbb{N}\]
            Zu zeigen: Die Folge \((z_n)\) konvergiert gegen 0 \\
            Es soll also gelten \(|\frac{i^n}{3^n} -0|\)
            Die Folge \(i^n\) ist zyklisch und nimmt die folgenden Werte an: \\
            \(i\) für \(n \% 4 = 1\),  \(-1\) für \(n \% 4 = 2\), \(-i\) für \(n \% 4 = 3\), \(1\) für \(n \% 4 = 0\) \\
            Damit ist ein Folgenglied entweder komplett reel: \(\frac{-1}{3^n}\) und \(\frac{1}{3^n}\), sein Betrag damit \(\frac{1}{3^n}\) \\
            Oder es ist komplett imaginär: \(\frac{-i}{3^n}\) und \(\frac{i}{3^n}\), sein Betrag damit ebenfalls \(\frac{1}{3^n}\) \\
            Der Betrag eines jeden Folgengliedes ist gleich, er ist \(\frac{1}{3^n}\). \\
            Wir setzen die Definition der komlexen Wurzel ein:
            \[ |\frac{i^n}{3^n} -0| = |\frac{i^n}{3^n}| = \frac{1}{3^n} < \frac{1}{n}\]
            Die Aussage \(n < 3^n\) ist triviall mit vollständiger Induktion zu zeigen. \\
            Es gilt nun also: \\
            \[\forall \: \epsilon > 0 \; \exists \; N_{\epsilon} \in \mathbb{N} : \forall n \geq N_{\epsilon}: \: \frac{1}{n} < \epsilon
                \text{ mit } \epsilon \in \mathbb{R} \text{ und } n \in \mathbb{N}\]
            Dies ist eine wahre Aussage. Diese ist equivalent zum Supremumsaxiom (Folgerungen aus dem Supremumsaxiom (c)).
            Also war die Annahme richtig. 0 ist der Grenzwert und die Folge konvergiert.

            \item \( \sum_{n=0}^{\infty} \frac{i^n}{3^n} \) \\
            \[ |\frac{i^n}{3^n} -0| = |\frac{i^n}{3^n}| = \sqrt{ {\left( \frac{1}{3^n} \right)}^2 } = \frac{1}{3^n} = {\frac{1}{3}}^n\]
            Die Reihe lässt sich also auch schreiben als \(\sum_{n=0}^{\infty} {\frac{1}{3}}^n \), dass ist die geometrische Reihe.
            Da \(\frac{1}{3}\) kleiner als 1 ist, konvergiert sie. Die Reihe konvergiert gegen \(\frac{1}{1-q}\) also \(\frac{3}{2}\)
            
            \item \(z_n := \frac{ 3n^2 - 4n + in^2 }{ 4in^2 -in + 1 } \) \\
            Wir erweitern den Bruch mit dem komplex Konjugierten des Nenners:
            \[ \frac{ 3n^2 - 4n + in^2 }{ 4in^2 -in + 1 } = \frac{ 3n^2 - 4n + in^2 }{ 4in^2 -in + 1 } \frac{ -4in^2 +in + 1 }{ -4in^2 +in + 1 } \]
            Nun ausmultiplizieren und vereinfachen:
            \begin{gather*}
                \frac{ (3n^2 - 4n + in^2) (-4in^2 +in + 1) }{ (4in^2 -in + 1) (-4in^2 +in + 1) } = \\
                \frac{ -4 i^2 n^4 + i^2 n^3 - 12 i n^4 + 19i n^3 - 3 i n^2 + 3 n^2 - 4 n }{ - 16 i^{2} n^{4} + 8 i^{2} n^{3} - i^{2} n^{2} + 1 } = \\
                \frac{ 4 n^4 - n^3 - 12 i n^4 + 19i n^3 - 3 i n^2 + 3 n^2 - 4 n }{ 16 n^4 - 8 n^3 + n^2 + 1 } = \\
                \frac{ 4 n^4 - n^3 + 3 n^2 - 4n }{16 n^4 - 8 n^3 + n^2 + 1} + i\frac{ -12 n^4 +19 n^3 -3n^2}{16 n^4 - 8 n^3 + n^2 + 1} = \\
                \frac{n^4}{n^4} \frac{ 4 - n^{-1} + 3 n^{-2} - 4n^{-3} }{16 - 8 n^{-1} + n^{-2} + 1n^{-4} } 
                    + \frac{n^4}{n^4} i\frac{ -12 + 19 n^{-1} -3n^{-2} }{16 - 8 n^{-1} + n^{-2} + 1 n^{-4}} = \\
                \frac{ 4 - n^{-1} + 3 n^{-2} - 4n^{-3} }{16 - 8 n^{-1} + n^{-2} + 1n^{-4} } 
                    + i\frac{ -12 + 19 n^{-1} -3n^{-2} }{16 - 8 n^{-1} + n^{-2} + 1 n^{-4}}
            \end{gather*}
            Nach Satz 3.10 konvergiert eine komplexe Folge gegen die Summe des Grenzwertes ihrer Folge für den reelen Anteil
            und des Grenzwertes der Folge für ihren imaginären Anteil. Wir untersuchen diese beiden Teilfolgen nun auf Konvergenz.
            Wir wissen das \(\frac{1}{n}\) gegen 0 konvergiert. Mit Satz 2.27(e) folgt daraus dass auch \(\frac{1}{n} \frac{1}{n} = \frac{1}{n^2}\) gegen 0 konvergiert.
            Ebenso alle weiteren Folgen \(n^{-x}\). Mit Satz 2.27(c) und Satz 2.27(e) folgt das eine Summe von \(\lambda \: n^{-x}\) ebenfalls gegen Null konvergiert.
            Bildet man nun den Grenzwert so gehen alle Terme außer (4, 16) für die erste Folge und (-12, 16) für die zweite gegen Null. Der Grenzwert ist also:
            \begin{gather*}
                \lim_{n \to \infty} \frac{ 4 - n^{-1} + 3 n^{-2} - 4n^{-3} }{16 - 8 n^{-1} + n^{-2} + 1n^{-4} }
                + i\frac{ -12 + 19 n^{-1} -3n^{-2} }{16 - 8 n^{-1} + n^{-2} + 1 n^{-4}} = \\ 
                \lim_{n \to \infty} \frac{ 4 - n^{-1} + 3 n^{-2} - 4n^{-3} }{16 - 8 n^{-1} + n^{-2} + 1n^{-4} }
                + \lim_{n \to \infty} i\frac{ -12 + 19 n^{-1} -3n^{-2} }{16 - 8 n^{-1} + n^{-2} + 1 n^{-4}} \\ 
                = \frac{4}{16} + i\frac{-12}{16} = \frac{1}{4} - i\frac{3}{4}
            \end{gather*}
            Die Folge konvergiert gegen \(\frac{1}{4} - i\frac{3}{4}\)

            \item \(z_n := \frac{3n - 1}{9n + 100} + \frac{n^2}{n^2 + n + 1}i \) \\
            Umformen:
            \begin{gather*}
                \frac{3n - 1}{9n + 100} + \frac{n^2}{n^2 + n + 1}i = \\
                \frac{n}{n} \frac{3 - 1n^{-1}}{9 + 100n^{n-1}} + \frac{n^2}{n^2} \frac{1}{1 + n^{-1} + n^{-2}}i \\
                \frac{3 - 1n^{-1}}{9 + 100n^{n-1}} + \frac{1}{1 + n^{-1} + n^{-2}}i
            \end{gather*}
            Nach Satz 3.10 konvergiert eine komplexe Folge gegen die Summe des Grenzwertes ihrer Folge für den reelen Anteil
            und des Grenzwertes der Folge für ihren imaginären Anteil. Wir untersuchen diese beiden Teilfolgen nun auf Konvergenz.
            Wir wissen das \(\frac{1}{n}\) gegen 0 konvergiert. Mit Satz 2.27(e) folgt daraus dass auch \(\frac{1}{n} \frac{1}{n} = \frac{1}{n^2}\) gegen 0 konvergiert.
            Ebenso alle weiteren Folgen \(n^{-x}\). Mit Satz 2.27(c) und Satz 2.27(e) folgt das eine Summe von \(\lambda \: n^{-x}\) ebenfalls gegen Null konvergiert.
            Bildet man nun den Grenzwert so gehen alle Terme außer (3, 9) für die erste Folge und (1, 1) für die zweite Folge gegen Null. Der Grenzwert ist also:
            \begin{gather*}
                \lim_{n \to \infty} \frac{3 - 1n^{-1}}{9 + 100n^{n-1}} + \frac{1}{1 + n^{-1} + n^{-2}}i = \\
                \lim_{n \to \infty} \frac{3 - 1n^{-1}}{9 + 100n^{n-1}} + \lim_{n \to \infty} \frac{1}{1 + n^{-1} + n^{-2}}i = \\
                \frac{3}{9} + \frac{1}{1}i = \frac{1}{3} + i
            \end{gather*}
            Die Folge konvergiert gegen \(\frac{1}{3} + i\)
        \end{enumerate}
    \end{enumerate}

\end{document}