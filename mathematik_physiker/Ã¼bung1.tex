\documentclass{article}

\usepackage[a4paper, total={16cm, 25cm}]{geometry}
\usepackage{babel}[german]
\usepackage{booktabs}

\date{21.10.2021}
\title{Aufgabenblatt 1, Mathematik für Physiker 1}
% \coursetitle{Aufgabenblatt 1, Mathematik für Physiker 1}
% \courselabel{MfP 1}
% \exercisesheet{Aufgabenblatt 1}
\author{Finn Jannik Wagner}
% \student{Finn Jannik Wagner}
% \semester{Winter 2021}
% \university{Justus Liebig Universität}

\begin{document}
    \maketitle
    \section{A1.1}
    Es ist zu zeigen das
    \((M \Delta N) \Delta (N\Delta P) = M \Delta P\) gilt.

    \(A := (M \Delta N) \Delta (N\Delta P)\), 
    \(B := M \Delta P\)

    \subsection{Hierzu eine Fallunterscheidung:}
    
    \begin{enumerate}
        \item Fall \( x \notin M, N, P \)
        Ist x in keiner der drei Menge, so ist es weder in A noch B 

        \item Fall \(x \in M \land x \notin N, P\) \\
        \(\Rightarrow x \in M \Delta N \), weil es nicht in beiden Mengen ist. \\
        \(\Rightarrow x \in N \Delta P \), weil es weder in N noch in P ist. \\
        \(\Rightarrow x \in ( M \Delta N )\Delta(N \Delta P ) \), weil es nicht in beiden Mengen ist. \\
        \(\Rightarrow x \in M \Delta P \), weil es nicht in beiden Mengen ist.

        Damit gilt \(x \in A, B \)

        \item Fall \(x \in P \land x \notin M, N\) \\
        Dieser Fall ist equivalent zu Fall 2
        
        \item Fall \(x \in N \land x \notin M, N \) \\
        \(\Rightarrow x \in M \Delta N \), weil es nicht in beiden Mengen ist. \\
        \(\Rightarrow x \in N \Delta P \), weil es nicht in beiden Mengen ist. \\
        \(\Rightarrow x \notin ( M \Delta N )\Delta(N \Delta P ) \), weil es in beiden Mengen ist. \\
        \(\Rightarrow x \in M \Delta P \), weil es weder in M noch P ist.

        Damit gilt \(x \notin A, B \) 

        \item Fall \(x \in M, N \land x \notin P \) \\
        \(\Rightarrow x \notin M \Delta N \), weil es in beiden Mengen ist. \\
        \(\Rightarrow x \in N \Delta P \), weil es nicht in beiden Mengen ist. \\
        \(\Rightarrow x \in ( M \Delta N )\Delta(N \Delta P ) \), weil es nur in \(N \Delta P\) ist. \\
        \(\Rightarrow x \in M \Delta P \), weil es nur in M ist.

        Damit gilt \(x \in A, B \) 

        \item Fall \(x \in M, P \land x \notin N \) \\
        \(\Rightarrow x \in M \Delta N \), weil es nur in M ist. \\
        \(\Rightarrow x \in N \Delta P \), weil es nur in P ist. \\
        \(\Rightarrow x \notin ( M \Delta N )\Delta(N \Delta P ) \), weil es in \(M \Delta N\) und \(N \Delta P\) ist. \\
        \(\Rightarrow x \notin M \Delta P \), weil es in M und P ist.

        Damit gilt \(x \notin A, B \) 
        
        \item Fall \(x \in N, P \land x \notin M \) \\
        \(\Rightarrow x \in M \Delta N \), weil es nur in N ist. \\
        \(\Rightarrow x \notin N \Delta P \), weil es in N und P ist. \\
        \(\Rightarrow x \in ( M \Delta N )\Delta(N \Delta P ) \), weil es nur in \(N \Delta P\) ist. \\
        \(\Rightarrow x \in M \Delta P \), weil es nur in P ist.

        Damit gilt \(x \in A, B \)

        \item Fall \(x \in M, P, N \) \\
        \(\Rightarrow x \notin M \Delta N \), weil es in M und N ist. \\
        \(\Rightarrow x \notin N \Delta P \), weil es in N und P ist. \\
        \(\Rightarrow x \notin ( M \Delta N )\Delta(N \Delta P ) \), weil es weder in \(M \Delta N\) noch in \(N \Delta P\) ist. \\
        \(\Rightarrow x \notin M \Delta P \), weil es in M und P ist.

        Damit gilt \(x \notin A, B \) 
    \end{enumerate}
    \( \Rightarrow \) Da in allen acht möglichen Fällen wie ein Element x
    in den Mengen verteilt ist die Operationen A und B zum gleichen Ergebnis kommen sind sie gleich.


    \section{1.1}
    \subsection{\(M \cap ( N \Delta P) = (M \cap N) \Delta (M \cap N) \)}
    Zu zeigen: \( A \Delta B = (A \backslash B ) \cup ( B \backslash A ) = (A \cup B) \backslash ( A \cap B) \) \\
    Füge beiden Seiten \( A \cap B \) hinzu. \\
    \(\Rightarrow (A \backslash B ) \cup ( B \backslash A ) \cup ( A \cap B ) = ( A \cup B ) \backslash ( A \cap B ) \cup ( A \cap B ) \) \\
    \(\Rightarrow (A \backslash B ) \cup ( B \backslash A ) \cup ( A \cap B ) = (A \cup B) \) \\
    Mit Assoziativgesetz \(A \backslash (B \backslash C) = (A \backslash B) \cup (A \cap C) \) \\
    \(\Rightarrow (A \backslash B ) \cup ( B \backslash (A \backslash A) ) = (A \cup B) \) \\
    \(\Rightarrow (A \backslash B ) \cup B = (A \cup B) \) \\
    \(\Rightarrow ( A \cup B ) = (A \cup B) \) \\

    \section{1.2 Aufgabe}
    Zu zeigen: \(M \cap (N \Delta P) = (M \cap N) \Delta (M \cap P) \) \\
    \(\Rightarrow M \cap (N \Delta P) = ((M \cap N) \backslash (N \cap P)) \cup ((M \cap P) \backslash (M \cap P))\)


    \section{2.1}
    \subsection{(i)}
    \begin{tabular}{c c c c c c c}
        A & B & \( A \lor B \) & \( \lnot ( A \lor B) \) & \( \lnot A \) & \( \lnot B \) & \( \lnot A \land \lnot B \) \\
        \midrule
        w & w & w & f & f & f & f\\
        w & f & w & f & f & w & f\\
        f & w & w & f & w & f & f\\
        f & f & f & w & w & w & w\\
    \end{tabular} \\
    Die Spalten vier und sieben sind für alle Wertekombinationen von A und B gleich.
    \( \Rightarrow \) \( \lnot ( A \lor B) \) und \( \lnot A \land \lnot B \) sind equivalent.

    \subsection{(ii)}
    \begin{tabular}{c c c c c c c}
        A & B & \( A \land B \) & \( \lnot ( A \land B ) \) & \( \lnot A \) & \( \lnot B \) & \( \lnot A \lor \lnot B \) \\
        \midrule
        w & w & w & f & f & f & f\\
        w & f & f & w & f & w & w\\
        f & w & f & w & w & f & w\\
        f & f & f & w & w & w & w\\
    \end{tabular} \\
    Die Spalten vier und sieben sind für alle Wertekombinationen von A und B gleich.
    \( \Rightarrow \) \( \lnot ( A \land B) \) und \( \lnot A \lor \lnot B \) sind equivalent.



\end{document}