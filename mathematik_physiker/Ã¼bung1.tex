\documentclass{article}

%\usepackage[a4paper, total={16cm, 25cm}]{geometry}
\usepackage{babel}[german]
\usepackage{booktabs}
\usepackage{mathtools}
\usepackage{enumitem}

\date{21.10.2021}
\title{Aufgabenblatt 1, Mathematik für Physiker 1}
\author{Finn Jannik Wagner}

\begin{document}
    \maketitle
    
    \section*{Aufgabe 1}

        \subsection*{Aufgabe 1.1 i)}

        Es ist zu zeigen das \((M \Delta N) \Delta (N\Delta P) = M \Delta P\) gilt.
        \(A := (M \Delta N) \Delta (N\Delta P)\), \(B := M \Delta P\)

        Hierzu eine Fallunterscheidung:

        \begin{enumerate}
            \item Fall \( x \notin M, N, P \)
            Ist x in keiner der drei Menge, so ist es weder in A noch B 

            \item Fall \(x \in M \land x \notin N, P\) \\
            \(\Rightarrow x \in M \Delta N \), weil es nicht in beiden Mengen ist. \\
            \(\Rightarrow x \in N \Delta P \), weil es weder in N noch in P ist. \\
            \(\Rightarrow x \in ( M \Delta N )\Delta(N \Delta P ) \), weil es nicht in beiden Mengen ist. \\
            \(\Rightarrow x \in M \Delta P \), weil es nicht in beiden Mengen ist.

            Damit gilt \(x \in A, B \)

            \item Fall \(x \in P \land x \notin M, N\) \\
            Dieser Fall ist equivalent zu Fall 2
            
            \item Fall \(x \in N \land x \notin M, N \) \\
            \(\Rightarrow x \in M \Delta N \), weil es nicht in beiden Mengen ist. \\
            \(\Rightarrow x \in N \Delta P \), weil es nicht in beiden Mengen ist. \\
            \(\Rightarrow x \notin ( M \Delta N )\Delta(N \Delta P ) \), weil es in beiden Mengen ist. \\
            \(\Rightarrow x \in M \Delta P \), weil es weder in M noch P ist.

            Damit gilt \(x \notin A, B \) 

            \item Fall \(x \in M, N \land x \notin P \) \\
            \(\Rightarrow x \notin M \Delta N \), weil es in beiden Mengen ist. \\
            \(\Rightarrow x \in N \Delta P \), weil es nicht in beiden Mengen ist. \\
            \(\Rightarrow x \in ( M \Delta N )\Delta(N \Delta P ) \), weil es nur in \(N \Delta P\) ist. \\
            \(\Rightarrow x \in M \Delta P \), weil es nur in M ist.

            Damit gilt \(x \in A, B \) 

            \item Fall \(x \in M, P \land x \notin N \) \\
            \(\Rightarrow x \in M \Delta N \), weil es nur in M ist. \\
            \(\Rightarrow x \in N \Delta P \), weil es nur in P ist. \\
            \(\Rightarrow x \notin ( M \Delta N )\Delta(N \Delta P ) \), weil es in \(M \Delta N\) und \(N \Delta P\) ist. \\
            \(\Rightarrow x \notin M \Delta P \), weil es in M und P ist.

            Damit gilt \(x \notin A, B \) 
            
            \item Fall \(x \in N, P \land x \notin M \) \\
            \(\Rightarrow x \in M \Delta N \), weil es nur in N ist. \\
            \(\Rightarrow x \notin N \Delta P \), weil es in N und P ist. \\
            \(\Rightarrow x \in ( M \Delta N )\Delta(N \Delta P ) \), weil es nur in \(N \Delta P\) ist. \\
            \(\Rightarrow x \in M \Delta P \), weil es nur in P ist.

            Damit gilt \(x \in A, B \)

            \item Fall \(x \in M, P, N \) \\
            \(\Rightarrow x \notin M \Delta N \), weil es in M und N ist. \\
            \(\Rightarrow x \notin N \Delta P \), weil es in N und P ist. \\
            \(\Rightarrow x \notin ( M \Delta N )\Delta(N \Delta P ) \), weil es weder in \(M \Delta N\) noch in \(N \Delta P\) ist. \\
            \(\Rightarrow x \notin M \Delta P \), weil es in M und P ist.

            Damit gilt \(x \notin A, B \) 
        \end{enumerate}

        \( \Rightarrow \) Da in allen acht möglichen Fällen, wie ein Element x
        in den Mengen verteilt sein kann, die Operationen A und B zum gleichen Ergebnis kommen, sind sie gleich.


        \subsection*{Aufgabe 1.1 ii)}
            \newtheorem{thm}{Theorem}
            \newtheorem{lemma}[thm]{Lemma}
            \subsubsection*{Hilfslemma 1: Weitere Definition der symmetrischen Differenz \( \Delta \)}
                Zu zeigen: \(A \Delta B = (A \backslash B) \cup (B \backslash A) = (A \cup B) \backslash (A \cap B) \) \\
                Füge beiden Seiten \( A \cap B \) hinzu. \\
                \(\Rightarrow (A \backslash B ) \cup ( B \backslash A ) \cup ( A \cap B ) = ( A \cup B ) \backslash ( A \cap B ) \cup ( A \cap B ) \) \\
                \(\Rightarrow (A \backslash B ) \cup ( B \backslash A ) \cup ( A \cap B ) = (A \cup B) \) \\
                Mit Assoziativgesetz \(A \backslash (B \backslash C) = (A \backslash B) \cup (A \cap C) \) \\
                \(\Rightarrow (A \backslash B ) \cup ( B \backslash (A \backslash A) ) = (A \cup B) \) \\
                \(\Rightarrow (A \backslash B ) \cup B = (A \cup B) \) \\
                \(\Rightarrow ( A \cup B ) = (A \cup B) \) \\

            \subsubsection*{Eigentliche Aufgabe Finn}
                Zu zeigen: \(M \cap (N \Delta P) = (M \cap N) \Delta (M \cap P) \) \\
                \(M \cap (N \Delta P)\) \\
                \indent \(\Downarrow \) Hilfslemma 1 \\
                \[= M \cap ((N \cup P) \backslash (N \cap P))\] \\
                \indent \(\Downarrow \) Anwenden des Distributivgesetzes \(A \cap (B \backslash C) = (A \cap B ) \backslash (A \cap C)\) \\
                \(= (M \cap (N \cup P)) \backslash (M \cap N \cap \ P)\) \\
                \indent \(\Downarrow \) Anwenden des Distributivgesetzes \(A \cap (B \cup C) = (A \cap B ) \cup (A \cap C)\) \\
                \(=(M \cap N) \cup (M \cap P) \backslash (M \cap N) \cap (M \cap P)\) \\
                \indent \(\Downarrow \) Hilfslemma 1 rückwärts\\
                \(=(M \cap N) \Delta (M \cap P)\) \\
                

    
    \subsection*{Aufgabe 1.2 i)}
        \begin{tabular}{c c c c c c c}
            A & B & \( A \lor B \) & \( \lnot ( A \lor B) \) & \( \lnot A \) & \( \lnot B \) & \( \lnot A \land \lnot B \) \\
            \midrule
            w & w & w & f & f & f & f\\
            w & f & w & f & f & w & f\\
            f & w & w & f & w & f & f\\
            f & f & f & w & w & w & w\\
        \end{tabular} \\
        Die Spalten vier und sieben sind für alle Wertekombinationen von A und B gleich. \\
        \( \Rightarrow \) \( \lnot ( A \lor B) \) und \( \lnot A \land \lnot B \) sind equivalent.

    \subsection*{Aufgabe 1.2 ii)}

        \begin{tabular}{c c c c c c c}
            A & B & \( A \land B \) & \( \lnot ( A \land B ) \) & \( \lnot A \) & \( \lnot B \) & \( \lnot A \lor \lnot B \) \\
            \midrule
            w & w & w & f & f & f & f\\
            w & f & f & w & f & w & w\\
            f & w & f & w & w & f & w\\
            f & f & f & w & w & w & w\\
        \end{tabular} \\
        Die Spalten vier und sieben sind für alle Wertekombinationen von A und B gleich. \\
        \( \Rightarrow \) \( \lnot ( A \land B) \) und \( \lnot A \lor \lnot B \) sind equivalent.

    \subsection*{Aufgabe 1.4}
        Gegeben \(M \not= \emptyset \), \(S:=\{x \in M | \text{x hat Sorgen}\} \), \(L:=\{x \in M | \text{x trinkt Likör}\} \) und \(\text{Likörproduzent} \in M\)
        \subsubsection*{(a) \(\text{Wer Sorgen hat, trinkt Likör} = S \subset L \)}
            \begin{enumerate}[label=(\roman*)]
                \item \(\text{Wer Likör trinkt, hat Sorgen} = L \subset S\) Falsch
                \item \(\text{Wer keiene Likör trinkt, hat keine Sorgen} = M \backslash L \subset M \backslash S\) Wahr
                \item \(\text{Niemand hat Sorgen und trinkt keinen Likör} = S \cap (M \backslash L) = \emptyset\) Wahr
                \item \(\text{Jemand hat Sorgen und trinkt keinen Likör} = S \cap (M \backslash L) \not= \emptyset\) Falsch
                \item \(\text{Jemand trinkt Likör} = L \not= \emptyset\) Unbestimmt
            \end{enumerate}

        \subsubsection*{(b) \(\text{Wer Likör trinkt, hat keine Sorgen} = L \subset (M \backslash S)\)}
            \begin{enumerate}[label=(\roman*)]
                \item \(\text{Jeder hat Sorgen} = S = M\) Falsch
                \item \(\text{Jemand hat Sorgen} = S \neq \emptyset\) Falsch
                \item \(\text{Niemand hat Sorgen} = S = \emptyset\) Wahr
            \end{enumerate}

        \subsubsection*{(c) \(\text{Trinkt niemand Likör, so haben Likörproduzenten Sorgen} = L = \emptyset \Rightarrow \text{Likörproduzent} \in S\)}
            \begin{enumerate}[label=(\roman*)]
                \item \(\text{Jeder trinkt Likör} = L = M\) Unbestimmt
                \item \(\text{Jemand trinkt Likör} = L \neq \emptyset\) Wahr
                \item \(\text{Niemand trinkt Likör} = L = \emptyset\) Falsch
            \end{enumerate}

\end{document}