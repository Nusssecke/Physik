\documentclass{article}

%\usepackage[a4paper, total={170mm,257mm}]{geometry}
\usepackage[a4paper, top=2cm, bottom=3cm]{geometry}
\usepackage{babel}[german]
\usepackage{booktabs}
\usepackage{mathtools}
\usepackage{amssymb}
\usepackage{enumitem}

\date{28.10.2021}
\title{Aufgabenblatt 3, Mathematik für Physiker 1}
\author{Florian Adamczyk, Finn Wagner}

\begin{document}
    \maketitle
    
    \section*{A 3.1}
       Seien \(a, b, c, d \in \mathbb{R}\)
       \begin{enumerate}[label = (\alph*)]
            \item 
                \begin{align*}
                    ab & \leq \frac{1}{2} (a^2 x^2 + \frac{b^2}{x^2}) \: \text{für alle} \: x \in \mathbb{R} \ {0} \\
                    \Leftrightarrow ab & \leq \frac{a^2 x^2}{2} + \frac{b^2}{2x^2} \: | * 2 \\
                    \Leftrightarrow 2ab & \leq a^2 x^2 + \frac{b^2}{x^2} \: | -2ab \\
                    \Leftrightarrow 0 & \leq a^2 x^2 + \frac{b^2}{x^2} - 2ab \: | \: \text{mit } x^2 \text{ erweitert} \\ 
                    \Leftrightarrow 0 & \leq \frac{a^2 x^4}{x^2} + \frac{-2abx^2}{x^2} + \frac{b^2}{x^2} \\
                    \Leftrightarrow 0 & \leq \frac{1}{x^2} (a^2 x^4  - 2abx^2 + b^2) \: | * x^2 \: \text{Binomische Formel} | \\
                    \Leftrightarrow 0 & \leq {(ax^2 - b)}^2 \\
                    & \text{Ein Quadrat ist immer größer gleich Null}
                \end{align*}
            \item 
                \begin{gather*}
                    ab + bc + ac \leq a^2 + b^2 + c^2 \: | \: *!2\\
                    \Leftrightarrow 2ab + 2bc + 2ac \leq 2a^2 + 2b^2 + 2c^2 \\
                    \text{Umformen mit: } -2ab -2bc -2ac \text{ und Umsortieren}\\
                    \Leftrightarrow 0 \leq (a^2 -2ab + b^2) + (a^2 -2ac + c^2) + (b^2 -2bc + c^2) \\
                    \text{2.te Binomische Formel anwenden:} \\
                    \Leftrightarrow 0 \leq {(a - b)}^2 + {(a - c)}^2 + {(b - c)}^2 \\
                    \text{Ein Quadrat ist immer größer gleich Null.} \\
                    \text{Eine Summe aus Quadrate ist ebenfalls immer größer gleich Null}
                \end{gather*}
            \item 
                \begin{gather*}
                    \frac{a}{b} < \frac{c}{d} \Leftrightarrow^! \frac{a}{b} < \frac{a+c}{b+d} < \frac{c}{d} \\
                    \text{1. Richtung} \\
                    \Leftrightarrow \frac{a}{b} < \frac{a+c}{b+d} \Leftrightarrow a * (b + d) < b * (a + c) \\
                    \Leftrightarrow ab + ad < ba + bc \: | \: - ab \\
                    \Leftrightarrow ad < bc \: | \: /b \: /d\\
                    \Leftrightarrow \frac{a}{b} < \frac{c}{d} \\
                    \\
                    \text{2. Richtung} \\
                    \frac{a+c}{b+d} <^! \frac{ad}{cb} \Leftrightarrow (a+c)*d < (b+d)*c \\
                    \Leftrightarrow ad + cd < bc + dc \: | \: -cd \\
                    \Leftrightarrow ad < bc \: | \: /b \: /d \\
                    \Leftrightarrow \frac{a}{b} < \frac{c}{d}
                \end{gather*}
            \item
                Summe macht Flo
        \end{enumerate}
    
    \section*{A 3.2}
        \begin{enumerate}[label = (\alph*)]
            \item 
                \begin{gather*}
                    \text{Für } n \in \mathbb{N} \text{ und } x > -1 \text{ gilt die Bernoulische Ungleichung:} \\
                    {(1+x)}^n \geq 1 + nx \\
                    \textbf{Induktionsanfang mit } n=0: \\
                    {1+x}^0 \geq 1 \geq 1 + 0x \\
                    \textbf{Induktionsschritt} \\
                    {(1+x)}^n \geq 1 + nx \Rightarrow {(1+x)}^{n+1} \geq 1 + (n+1)x \\
                    {(1+x)}^{n+1} \geq 1 + (n+1)x \\
                    \Rightarrow {(1+x)}^{n}*(1+x) \geq 1 + nx + x \\
                    \text{Anwenden der Inuktionsvorraussetzung} \\
                    \Rightarrow {(1+x)}^{n}*(1+x) \geq (1 + nx)*(1 + x) \\
                    \text{Ausmultiplizieren} \\
                    \Rightarrow (1 + nx)*(1 + x) = 1 + nx + nx^2 \\
                    \Rightarrow 1 + nx + x + nx^2 \geq 1 + nx + x  \: | \: -x - nx - 1\\
                    \Rightarrow nx^2 \geq 0 \\
                    \text{Ein Quadrat ist immer größer gleich Null} \\
                    \textbf{Da Induktionsanfang und Schritt gelten,} \\
                    \textbf{ist die Aussage für alle natürlichen Zahlen gültig}
                \end{gather*}

            \item
                \begin{align*}
                    q \in \mathbb{R}, q \neq 1; \: & \sum_{k=0}^{n} q^k = \frac{1-q^{n+1}}{1-q} \\
                    \textbf{Induktionsafang mit } & n=0: \\
                    \sum_{k=0}^{0} q^k = q^0 = 1 & = \frac{1-q^{0+1}}{1-q} = \frac{1-q}{1-q} = 1 \\
                    \textbf{Induktionsschritt} & \\
                    \sum_{k=0}^{n} q^k & = \frac{1-q^{n+1}}{1-q} \Rightarrow \sum_{k=0}^{n+1} q^k = \frac{1-q^{n+2}}{1-q} \\
                    \sum_{k=0}^{n+1} q^k &= \frac{1-q^{n+2}}{1-q} \\
                    \Leftrightarrow q^{n+1} + \sum_{k=0}^{n} q^k &= \frac{1-q^{n+1} * q}{1-q} \: | - q^{n+1} \\
                    \Leftrightarrow \sum_{k=0}^{n} q^k &= \frac{1-q^{n+1} * q}{1-q} - q^{n+1} \: | \: \text{Letzten Summand erweitern}\\
                    \Leftrightarrow \sum_{k=0}^{n} q^k &= \frac{1-q^{n+1} * q}{1-q} - \frac{q^{n+1}*(1-q)}{1-q} \\
                    \Leftrightarrow \sum_{k=0}^{n} q^k &= \frac{1-q^{n+1} * q}{1-q} - \frac{q^{n+1}- q^{n+1} * q}{1-q} \\
                    \Leftrightarrow \sum_{k=0}^{n} q^k &= \frac{1-q^{n+1} * q - q^{n+1} + q^{n+1} * q}{1-q} \\
                    \Leftrightarrow \sum_{k=0}^{n} q^k &= \frac{1 - q^{n+1}}{1-q} \\
                    &\textbf{Da Induktionsanfang und Schritt gelten,} \\
                    &\textbf{ist die Aussage für alle natürlichen Zahlen gültig}
                \end{align*}
                
            \item
                \begin{gather*}
                    n \in \mathbb{N}\{0\} \: n! \leq 4 * { \left( \frac{n}{2} \right) }^{n+1} \\
                    \textbf{Induktionsafang mit } n=1: \\
                    1! = 1 = 4 * {\left( \frac{1}{2} \right) }^{1+1} = 4 * {\left( \frac{1}{2} \right)}^{2} = 1 \\
                    \textbf{Induktionsschritt} \\
                    n! \leq 4 * {\left( \frac{n}{2} \right) }^{n+1} \Rightarrow (n+1)! \leq 4 * {\left( \frac{n+1}{2} \right) }^{n+2} \\
                    (n+1)! = n! * (n+1) \\
                    \text{Verwenden der Induktionsannahme} \\
                    \Rightarrow n! * (n+1) \leq 4 * {\left( \frac{n}{2} \right) }^{n+1} * (n + 1) \\
                    \\
                    \text{Als Hilfsterm: } {(1 + \frac{1}{n})}^{n+1} \geq 2 \: \: \forall \: n \in \mathbb{N} \\
                    \text{Da gilt: }
                    {(1 + \frac{1}{n})}^{n+1} \Rightarrow^\text{Bernoulli Ungleichung} 1 + \frac{n+1}{n}\\
                    \frac{n+1}{n} \text{ ist immer größer gleich 1 für } n \in \mathbb{N} \text{. Dieser Wert +1 ist also größer als 2} \\
                    \\
                    \Rightarrow \frac{{(1 + \frac{1}{n})}^{n+1}}{2} \geq 1 \\
                    \text{Abschätzen nach oben mit Faktor größer gleich 1} \\
                    4 * {\left( \frac{n}{2} \right) }^{n+1} * (n + 1) \leq  4 * {\left( \frac{n}{2} \right) }^{n+1} * (n + 1) * \frac{{(1 + \frac{1}{n})}^{n+1}}{2} \\
                    \text{Vereinfachen} \\
                    = 4 * {\left( \frac{n}{2} \right) }^{n+1} * {(1 + \frac{1}{n})}^{n+1} * \frac{(n + 1)}{2} \\
                    \text{Vereinfachen} \\
                    = 4 * {\left( \frac{n}{2} * {(1 + \frac{1}{n})} \right)}^{n+1} * \frac{(n + 1)}{2} \\
                    \text{Ausmultiplizieren} \\
                    = 4 * {\left( \frac{n}{2} + \frac{1}{2} \right)}^{n+1} * \frac{(n + 1)}{2} \\
                    \text{Vereinfachen} \\
                    = 4 * {\left( \frac{n+1}{2} \right)}^{n+1} * \frac{(n + 1)}{2} \\
                    \text{Faktor einklammern} \\
                    = 4 * {\left( \frac{n+1}{2} \right) }^{n+2} \\
                    \text{Damit gilt der Induktionsschritt} \\
                    \textbf{Da Induktionsanfang und Schritt gelten,} \\
                    \textbf{ist die Aussage für alle natürlichen Zahlen gültig}
                \end{gather*}
        \end{enumerate}

    \section*{A 3.3}
        \begin{enumerate}[label = (\alph*)]
            \item 
                \begin{gather*}
                    M_1 = \{1- \frac{1}{n}: n \in \mathbb{N} \} \\
                    \text{Zu zeigen: 1 ist eine obere Schranke von } M_1 \\
                    1 - \frac{1}{n} \leq 1 \: \forall n \in \mathbb{N} \text{ Es folgt} 0 \leq \frac{1}{n} \\
                    \Rightarrow \text{1 ist eine obere Schranke von } M_1
                \end{gather*} 
                \begin{gather*}
                    \text{Zu zeigen: 1 ist die kleinste obere Schranke von } M_1 \\
                    \text{Angenommen 1 wäre nicht die kleinste obere Schranke,} \\
                    \text{so gäbe es eine kleinere im Abstand} \epsilon \\
                    \text{Sei } \epsilon > 0 \in \mathbb{R} \\
                    1 - \frac{1}{n} \leq 1 - \epsilon \Rightarrow \epsilon \leq \frac{1}{n} \: \forall n \in \mathbb{N} \\
                    \text{Nach Lemma 2.18 gilt aber } \epsilon \leq 0 \text{ Das ist aber ein Wiederspruch, da } \epsilon > 0 \\
                    \Rightarrow \text{Somit war die Annahme falsch, es gibt keine kleinere obere Schranke.} \\
                    \text{1 ist das Supremum von } M_1
                \end{gather*}
                \begin{gather*}
                    \text{Zu zeigen: 1 ist Maximum von } M_1 \\
                    1-\frac{1}{n} = 1 \Rightarrow \frac{1}{n} = 0 \Rightarrow 1 = 0 \\
                    \text{Die Annahme war falsch } M_1 \text{ hat kein Maximum}
                \end{gather*}
                \begin{gather*}
                    \text{Zu zeigen: 0 ist eine untere Schranke von } M_1 \\
                    0 \leq 1 - \frac{1}{n} \Rightarrow \frac{1}{n} \leq 1 \Rightarrow 1 \leq n \\
                    \text{Alle natürliche Zahlen sind größer gleich 1} \\
                    \text{0 ist eine untere Schranke von } M_1
                \end{gather*}
                \begin{gather*}
                    \text{Zu zeigen: 0 ist die größte untere Schranke von } M_1 \\
                    \text{Angenommen 0 wäre nicht die größte untere Schranke,} \\
                    \text{so gäbe es eine größere im Abstand} \epsilon \\
                    \text{Sei } \epsilon > 0 \in \mathbb{R} \\
                    0 + \epsilon \leq 1 - \frac{1}{n} \Rightarrow \epsilon + \frac{1}{n} \leq 1 \\
                    \text{Sei } n=1 \\
                    \Rightarrow \epsilon + 1 = 1 \Rightarrow \epsilon = 0 \\
                    \text{Da } \epsilon \text{ größer als Null sein muss ist dies ein Wiederspruch} \\
                    \Rightarrow \text{Somit war die Annahme falsch, es gibt keine größere untere Schranke.} \\
                    \text{0 ist das Infimum von } M_1
                \end{gather*}
                \begin{gather*}
                    \text{Zu zeigen: 0 ist Minimum von } M_1 \\
                    0 = 1 - \frac{1}{n} \Rightarrow \frac{1}{n} = 1 \Rightarrow n = 1 \\
                    \text{0 ist Minimum von } M_1
                \end{gather*}
                Die Menge \(M_1\) ist nach oben beschränkt und hat in 1 ein Supremum, jedoch kein Maximum.
                Die Menge ist außerdem nach unten beschränkt und hat in 0 ein Infimum und Minimum
            \item 
                \[
                     M_2 = \{t \in \mathbb{R}, \text{t ist obere Schranke von } M_1\}
                \]
                In (a) haben wir das Supremum von \(M_1\) als 1 festgestellt. Jede weitere obere Schranke
                muss also größer sein als 1. Die Menge lässt sich umbschreiben als: \\
                \[
                    M_2 = \{x \geq 1, \: x \in \mathbb{R}\}    
                \]
                Und das wiederum als: \\
                \[[1,\inf)\] \\
                Die Menge \(M_2\) ist also nach Definition nach unten beschränkt mit dem Supremum von \(M_1\),
                1 ist damit Infimum und da es in der Menge liegt auch Minimum von \(M_2\).
                \(M_2\) ist aber nicht nach oben beschränkt, da eine obere Schranke beliebig viel größer sein kann
                als das Supremum. Oder trivial abzulesen aus der Intervallschreibweise.

                \item 
                \begin{gather*}
                    M_3 = \{{\left(1- \frac{1}{n^2}\right)}^n: n \in \mathbb{N} \} \\
                    \text{Zu zeigen: 1 ist eine obere Schranke von } M_3 \\
                    {\left(1- \frac{1}{n^2}\right)}^n \leq 1 \: \forall n \in \mathbb{N} \\
                    \Rightarrow 1- \frac{1}{n^2} \leq 1 \\
                    \Rightarrow 0 \leq \frac{1}{n^2} \Rightarrow 0 < 1 \\
                    \Rightarrow \text{1 ist eine obere Schranke von } M_3
                \end{gather*} 
                \begin{gather*}
                    \text{Zu zeigen: 1 ist die kleinste obere Schranke von } M_3 \\
                    \text{Angenommen 1 wäre nicht die kleinste obere Schranke,} \\
                    \text{so gäbe es eine kleinere im Abstand} \epsilon \\
                    \text{Sei } \epsilon > 0 \in \mathbb{R} \\
                    {\left(1- \frac{1}{n^2}\right)}^n \leq 1 - \epsilon \\
                    \text{Anwenden der Bernoulli-Ungleichung} \\
                    1 - \epsilon \geq {\left(1- \frac{1}{n^2}\right)}^n \geq 1 - n * \frac{1}{n^2} = 1 - \frac{1}{n} \\
                    \Rightarrow 1- \epsilon \geq 1 - \frac{1}{n} \\
                    \Rightarrow \frac{1}{n} \geq \epsilon \:\: \forall \: n \in \mathbb{N}\\
                    \text{Nach Lemma 2.18 gilt aber } \epsilon \leq 0 \text{ Das ist aber ein Wiederspruch, da } \epsilon > 0 \\
                    \Rightarrow \text{Somit war die Annahme falsch, es gibt keine kleinere obere Schranke.} \\
                    \text{1 ist das Supremum von } M_3
                \end{gather*}
                \begin{gather*}
                    \text{Zu zeigen: 1 ist Maximum von } M_3 \\
                    {\left(1- \frac{1}{n^2}\right)}^n = 1 \Rightarrow 1- \frac{1}{n^2} = 1 \\
                    \Rightarrow 0 = \frac{1}{n^2} \Rightarrow n^2 = 0 \Rightarrow n=0 \\
                    \text{n kann nicht null sein da in der Definition von } M_3 \text{ durch n geteilt wird.} \\
                    \text{Die Annahme war falsch } M_3 \text{ hat kein Maximum}
                \end{gather*}
                \begin{gather*}
                    \text{Zu zeigen: 0 ist eine untere Schranke von } M_3 \\
                    0 \leq {\left(1- \frac{1}{n^2}\right)}^n \Rightarrow 0 \leq 1- \frac{1}{n^2} \\
                    \frac{1}{n^2} \leq 1 \Rightarrow 1 \leq n^2 \Rightarrow 1 \leq n \\
                    \text{Alle natürliche Zahlen sind größer gleich 1} \\
                    \text{0 ist eine untere Schranke von } M_1
                \end{gather*}
                \begin{gather*}
                    \text{Zu zeigen: 0 ist die größte untere Schranke von } M_3 \\
                    \text{Angenommen 0 wäre nicht die größte untere Schranke,} \\
                    \text{so gäbe es eine größere im Abstand} \epsilon \\
                    \text{Sei } \epsilon > 0 \in \mathbb{R} \\
                    0 + \epsilon \leq {\left(1- \frac{1}{n^2}\right)}^n
                    \text{Sei } n=1 \\
                    \Rightarrow 0 + \epsilon \leq {\left(1- \frac{1}{1^2}\right)}^1 \\
                    \Rightarrow \epsilon \leq 0^1 \Rightarrow \epsilon \leq 0 \\
                    \text{Da } \epsilon \text{ größer als Null sein muss ist dies ein Wiederspruch} \\
                    \Rightarrow \text{Somit war die Annahme falsch, es gibt keine größere untere Schranke.} \\
                    \text{0 ist das Infimum von } M_3
                \end{gather*}
                \begin{gather*}
                    \text{Zu zeigen: 0 ist Minimum von } M_3 \\
                    0 =  {\left(1- \frac{1}{n^2}\right)}^n \Rightarrow 0 = 1- \frac{1}{n^2} \\
                    \Rightarrow \frac{1}{n^2} = 1 \Rightarrow 1 = n^2 \Rightarrow n = 1
                    \text{0 ist Minimum von } M_3
                \end{gather*}
            Die Menge \(M_3\) ist nach oben beschränkt und hat in 1 ein Supremum, jedoch kein Maximum.
            Die Menge ist außerdem nach unten beschränkt und hat in 0 ein Infimum und Minimum
        \end{enumerate}
\end{document}