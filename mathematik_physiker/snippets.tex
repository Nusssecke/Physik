\subsection{Alternative Beweisführung}
\((M \Delta N) \Delta (N\Delta P) = M \Delta P\) \\
Entferne auf beiden Seiten \( M \Delta P\) \\
\( ( M \Delta N ) \Delta ( N \Delta P ) \backslash ( M \Delta P ) = ( M \Delta P ) \backslash ( M \Delta P ) \) \\
\( ( M \Delta N ) \Delta ( N \Delta P ) \backslash ( M \Delta P ) = \emptyset \) \\
\( ( M \Delta N ) \Delta ( N \Delta P ) \backslash ( M \Delta P ) = \emptyset \) \\
Symmetrische Differenz auflösen
\( ((M \backslash N ) \cup (N \backslash M) \backslash ( N \backslash P ) \cup ( P \backslash N ) 
\cup
( N \backslash P ) \cup ( P \backslash N) \backslash ( M \backslash N ) \cup ( N \backslash M )) 
\backslash ( M \backslash P \cup P \backslash M)
= \emptyset \)
Falsch?

\subsubsection*{Hilfslemma 2}
Zu zeigen: \((M \cap N) \backslash P = (M \cap (N \backslash P))\) \\
\(\Rightarrow (M \cap N) \backslash P = (M \backslash P) \cap (N \backslash P)\) \\
Unter Verwendung des Distributivgesetzes \((A \cap B) \backslash C = (A \backslash C) \cap (B \backslash C) \) \\
\((M \backslash P) \cap (N \backslash P)\) lässt sich logisch schreiben als
\(x \in M \land x \notin P \land x \in N \land x \notin P\) was sich vereinfachen lässt zu \(x \in M \land x \in N \land x \notin P\)
was genau \(M \cap (N \backslash P)\) ist.

\subsubsection*{Hilfslemma 3}
Zu zeigen: \((M \cap N) \backslash P = (M \cap N ) \backslash (M \cap P)\) \\
\((M \cap N ) \backslash (M \cap P)\) lässt sich logisch schreiben als \((x \in M \land x \in N) \land \lnot (x \in M \land x \in P)\)
Mit den De Morgan'schen Regeln aus 1.2 lässt sich dies umformen zu: \((x \in M \land x \in N) \land (x \notin M \lor x \notin P)\)
Zu zeigen ist nun das dies equivalent zu \(x \in M \land x \in N \land x \notin P\) ist. \\
Wenn gilt das \(x \in (M \cap N) \backslash P \Rightarrow x \in M \land x \in N \land x \notin P \)  
\(\Rightarrow x \in M \land x \in N \land x \in M \land x \notin P\)
Was sich umformen lässt zu \(\Rightarrow x \in (M \cap N) \lnot\land x \in M \land x \notin P\) Böse

\subsubsection*{Eigentliche Aufgabe Ralph}
Zu zeigen: \(M \cap (N \Delta P) = (M \cap N) \Delta (M \cap P) \) \\
\((M \cap (N \Delta P)\) \\
Einsetzen der Definition der symmetrischen Differenz \\
\(M \cap (N \backslash P \cup P \backslash N)\) \\
Anwenden des Distributivgesetzes \(A \cap (B \cup C) = (A \cap B ) \cup (A \cap C)\) \\
\((M \cap (N \backslash P)) \cup (M \cap (P \backslash N)) \) \\
Zweimal Hilfslemma 2 angewandt \\
\((M \cap N) \backslash P \cup (M \cap P) \backslash N \) \\
Zweimal Hilfslemma 3 angewandt \\
\((M \cap N) \backslash (M \cap P) \cup (M \cap P) \backslash (M \cap N) \) \\
Daraus erhält man durch zweifaches einsetzen der Definition der symmetrischen Differenz \\
\((M \cap N) \Delta (M \cap P)\)