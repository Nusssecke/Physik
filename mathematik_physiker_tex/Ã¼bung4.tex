\documentclass[11pt]{article}

%\usepackage[a4paper, total={170mm,257mm}]{geometry}
\usepackage[a4paper, top=2cm, bottom=3cm]{geometry}
\usepackage{babel}[german]
\usepackage{booktabs}
\usepackage{mathtools}
\usepackage{amssymb}
\usepackage{enumitem}

\date{14.11.2021}
\title{Aufgabenblatt 4, Mathematik für Physiker 1}
\author{Florian Adamczyk, Finn Wagner}

\begin{document}
    \maketitle

    \section*{A 4.1}
    Zeigen sie mit Hilfe von Definition 2.24, dass die folgenden Folgen konvergieren.
    Definition 2.24 ist das Epsilonkriterium:
    \[\forall \: \epsilon > 0 \; \exists \; N_{\epsilon} \in \mathbb{N} : \forall n \geq N_{\epsilon} \: |x_n - x| < \epsilon
        \text{ mit } \epsilon \in \mathbb{R} \text{ und } n \in \mathbb{N}\]

    \begin{enumerate}[ label= (\roman*) ]
        \item \(a_n := \frac{ {(-1)}^n }{ n } \) \\
        Behauptung: Die Folge konvergiert gegen den Grenzwert 0, \(\lim_{n \to \infty } \frac{ {(-1)}^n }{ n } = 0 \). \\
        Es soll also gelten: \( |a_n - 0| < \epsilon \) mit \( \mathbb{R} \ni \epsilon > 0\) beliebig. \\
        Da \( { (-1) }^n \) als Folge immer zwischen -1 und 1 wechselt ist ihr Betrag 1. Damit lässt sich die die linke Seite wie folgt umformen.
        \[|a_n - 0| = |a_n| = | \frac{ {(-1)}^n }{ n } | = \frac{ |{(-1)}^n| }{ |n| } = \frac{ 1 }{ n } \]
        Es gilt nun also:
        \[ \frac{ 1 }{ n } < \epsilon \: \forall \: \epsilon > 0 \]
        Das ist eine wahre Aussage. Diese ist equivalent zum Supremumsaxiom (Folgerungen aus dem Supremumsaxiom (c))
        Also war die Behauptung wahr. 0 ist der Grenzwert und die Folge konvergiert.

        \item \(b_n := \frac{ 3n + 1 }{ 2n - 1 } \) \\
        Behauptung: Die Folge konvergiert gegen den Grenzwert \( \frac{3}{2} \), \(\lim_{n \to \infty } \frac{ 3n + 1 }{ 2n - 1 } = \frac{3}{2} \). \\
        Es soll also gelten: \( |b_n - \frac{3}{2}| < \frac{5}{2} \epsilon \)  mit \( \mathbb{R} \ni \epsilon > 0\) beliebig. \\
        Wir formen die linke Seite wie folgt um:
        \[ |b_n - \frac{3}{2}| = | \frac{ 3n + 1 }{ 2n - 1 } - \frac{3}{2} | \]
        Erweitere \( \frac{3}{2} \) mit \( n - \frac{1}{2} \) und vereinfache:
        \[ 
            = \left| \frac{ 3n + 1 }{ 2n - 1 } - \frac{ 3 (n - \frac{1}{2}) }{ 2 (n - \frac{1}{2}) } \right|
            = \left| \frac{ 3n + 1 }{ 2n - 1 } - \frac{ 3n - \frac{3}{2}}{ 2n - 1 } \right|
            = \left| \frac{ 3n + 1 - ( 3n - \frac{3}{2} ) }{ 2n - 1 } \right|
            = \left| \frac{ \frac{5}{2} }{ 2n - 1 } \right|
        \]
        Die Betragsstriche sind nicht notwendig, da \(2n - 1 > 0 \: \forall \: n \in \mathbb{N} \).
        \[
            = \left| \frac{ \frac{5}{2} }{ 2n - 1 } \right|
            = \frac{ \frac{5}{2} }{ 2n - 1 }
        \]
        Wir schätzen nun nach oben ab. \( 2n - 1\) ist nach Lemma 1 immer größer gleich als \(n\).
        Damit ist \( \frac{1}{ 2n -1 } \) immer kleiner gleich als \( \frac{1}{n} \)
        \[
            = \frac{ \frac{5}{2} }{ 2n - 1 } \leq \frac{ \frac{5}{2} }{ n }
        \]
        Setzt man nun wieder in die Behauptung ein, so ergibt sich:
        \[
            \frac{ \frac{5}{2} }{ n } \leq \frac{5}{2} \epsilon
            \Leftrightarrow \frac{ 1 }{ n } \leq \epsilon
        \]
        Das ist eine wahre Aussage. Diese ist equivalent zum Supremumsaxiom (Folgerungen aus dem Supremumsaxiom (c))
        Also war die Behauptung wahr. \(\frac{3}{2}\) ist der Grenzwert und die Folge konvergiert.

        \item \(c_n := \frac{1}{\sqrt{n}} \)
        Behauptung: Die Folge konvergiert gegen den Grenzwert 0, \(\lim_{n \to \infty } \frac{1}{\sqrt{n}} = 0 \). \\
        Es soll also gelten: \( |c_n - 0| < \epsilon \) mit \( \mathbb{R} \ni \epsilon > 0\) beliebig. \\
        Wir wissen das \(\frac{1}{n}\) gegen 0 konvergiert. Also
        \[\forall \: \epsilon > 0 \; \exists \; N_{\epsilon} \in \mathbb{N} : \forall n \geq N_{\epsilon} \: |\frac{1}{n} - 0| < \epsilon^2
        \text{ mit } \epsilon \in \mathbb{R} \text{ und } n \in \mathbb{N}\]
        Nehmen wir nun diese Ungleichung und ziehen auf beiden Seiten die Wurzel, erhalten wir auf Grund der Monotonie der Wurzelfunktion:
        \[
            |\frac{1}{n} - 0| < \epsilon^2
            = |\frac{1}{n}| < \epsilon^2
            = \frac{1}{n} < \epsilon^2
        \]
        \[
            \Rightarrow \sqrt{ \frac{1}{n} }  < \sqrt{ \epsilon^2 }
            = \frac{ \sqrt{1} }{ \sqrt{n} }  < \epsilon
            = \frac{ 1 }{ \sqrt{n} }  < \epsilon
            = \frac{ 1 }{ \sqrt{n} } - 0 < \epsilon
            = | \frac{ 1 }{ \sqrt{n} } - 0 | < \epsilon
        \]
        Was zu beweisen war. Damit ist der Grenzwert 0 und die Folge konvergiert.


    \end{enumerate}
    
    
    \section*{A 4.2}
    Sei \(M \in \mathbb{R}\) nichtleer und nach oben beschränkt, und sei \(s \in \mathbb{R}\).
    \begin{enumerate}[ label= (\alph*) ]
        \item Zeigen Sie: Genau dann ist \(s = \sup M\), wenn \(s\) eine obere Schranke von \(M\) ist
        und wenn eine Folge \({(x_n)}_{n \in \mathbb{N}}\) in \(M\) existiert mit \( \lim_{n \to \infty} x_n = s\).

        Beweisen der Äquivalenz durch Folgerung der Aussagen in beide Richtungen.
        \begin{center}
            \textbf{Es gilt \(s\) ist Supremum von \(M\).}
        \end{center}
        Es folgt das \(s\) insbesondere auch eine obere Schranke von \(M\) ist.
        \begin{enumerate}[ label = \arabic*. Fall ]
            \item \(s\) ist ein Element von \(M\), also \(s\) ist das Maximum von \(M\). \\
            So existiert immer die Folge \( {(x_n)}_{n \in \mathbb{N}} := s \), deren Elemente in \(M\) liegen und die gegen \(s\) konvergiert.

            \item \(s\) ist kein Element von \(M\). \\
            Sei \( {(y_n)}_{n \in \mathbb{N}} := s - \frac{1}{n} \). Die Folge \( (y_n) \) ist immer kleiner als \(s\) und konvergiert gegen \(s\).
            Sei nun \( {(x_n)}_{n \in \mathbb{N}} \) die gesuchte Folge. Für alle Elemente von \(x_n\) gelte \(x_n \in M \) und \(x_n > y_n\) für ein bestimmtes \(n\). \\
            \textbf{Zu zeigen: \(x_n\) ist immer definiert.} \\
            Angenommen es gäbe kein Element in \(M\) das größer als \(y_n\) für ein bestimmtes \(n\) ist, so wäre \(y_n\) eine obere Schranke der Menge \(M\).
            Da aber \(y_n\) gegen \(s\) konvergiert und damit immer kleiner als s ist, kann \(y_n\) keine obere Schranke sein, weil es kleiner ist als das Supremum \(s\).
            Es folgt, dass \(x_n\) immer definiert ist. \\
            
            Nun schätze man \({(x_n)}_{n \in \mathbb{N}}\) nach unten und nach oben ab. Es gilt \(x_n \geq y_n\) da \(x_n\) per Konstruktion immer größer gewählt wurde.
            Außerdem ist \(x_n\) immer kleiner als \(s\), da \(s\) kein Element der Menge \(M\) ist.
            Es folgt:
            \[ y_n \leq x_n \leq s \]
            Wendet man nun Satz 2.27(f) (Sandwichtheorem) an, so ergibt sich:
            \begin{align*}
                \lim_{n \to \infty} s - \frac{1}{n} \leq \lim_{n \to \infty} &x_n \leq \lim_{n \to \infty} s \\
                \Rightarrow s \leq &x \leq s
            \end{align*}
            Die Folge \((x_n)\) konvergiert gegen \(s\).
        \end{enumerate}
        Es folgt, dass \(s\) eine obere Schranke ist, und es eine Folge in der Menge gibt, die gegen \(s\) konvergiert.

        Nun in die andere Richtung.
        \begin{center}
            \textbf{Es gilt \(s\) ist eine obere Schranke von \(M\) und es existiert eine Folge \( {(x_n)}_{n \in \mathbb{N}} \) mit \( \lim_{n \to \infty} x_n = s \) }
        \end{center}
        Angenommen \(s\) ist kein Supremum. So muss es eine obere Schranke geben die kleiner als \(s\) ist. Dieses Supremum nennen wir \(k\).
        \[ k < s \Rightarrow k + \epsilon_0 = s \text{ mit } \epsilon_0 > 0 \]
        \[ \Leftrightarrow \epsilon_0 = s - k\]
        Da es eine Folge gibt deren Folgenglieder in \(M\) liegen und diese Folge gegen \(s\) konvergiert gilt:
        \[ \exists \: N_{\epsilon_0} \in \mathbb{N} \: \forall \: n \geq N_{\epsilon_0}  |x_n - s| < \epsilon_0 \]
        Setzen wir nun unser festes \( \epsilon_0 \) ein, so folgt:
        \[ |x_n - s| < \epsilon_0 \Leftrightarrow |x_n - s| < s - k \]
        
        \begin{enumerate}[ label = \arabic*. Fall ]
            \item \(x_n\) ist größer als \(s\) \\
            Das ist ein Wiederspruch, da \(k\) als Supremum angenommen wurde (\(s > k\)), ein Element der Menge kann nicht größer als das Supremum sein.

            \item \(x_n\) ist kleiner gleich \(s\) \\
            Man ersetze  die Betragsstriche mit einem Minus.
            \begin{align*}
                |x_n - s| < s - k &\Leftrightarrow -(x_n - s) < s - k \\
                  \Leftrightarrow -x_n + s < s - k &\Leftrightarrow - x_n < - k \\
                  &\Leftrightarrow x_n > k
            \end{align*}
            Das ist ein Wiederspruch, da \(k\) als Supremum angenommen wurde, ein Element der Menge kann nicht größer als das Supremum sein.
        \end{enumerate}
        Es folgt, dass es keine kleinere Schranke als geben kann. \(s\) ist also das Supremum.

        \item Wenn die Menge unendlich viele Elemente hat, können Sie dann die Folge aus (a) so wählen, dass \( x_n < x_{n+1} \) für alle \( n \in \mathbb{N} \) gilt? \\
        Angenommen, eine solche Folge \( { (x_n) }_{ n \in \mathbb{N} } \) würde existieren. Wähle die Menge \(M = [0,1] \cup {2} \) \\
        Angenommen die 2 ist ein Folgenglied von \((x_n)\). Nach Folgendefintion muss aber das nächste Element der Folge
        echt größer sein als 2. Diese Element läge nicht mehr in der Menge. Die 2 ist also kein Folgenglied. \\
        Alle Folgenglieder sind also kleiner gleich 1. \\
        Die Folge \((x_n)\) konvergiert gegen das Supremum von M, hier 2. In mathematischer Notation:
        \[\forall \: \epsilon > 0 \: \exists \: N_{\epsilon} \in \mathbb{N}: \: \forall n \geq N_{\epsilon}: |x_n - 2| < \epsilon \]
        Wähle \( \epsilon = 0,5 \):
        \[ |x_n - 2| < \epsilon \Rightarrow |x_n - 2| < 0,5\]
        Da alle \(x_n\) kleiner gleich 1 sind, lässt sich der Betrag durch ein Minus umschreiben:
        \[ -(x_n - 2) < 0,5 \Leftrightarrow -x_n -2 < 0,5 \Leftrightarrow -x_n < 2,5 \Leftrightarrow x_n > 2,5 \]
        Da alle \(x_n\) kleiner gleich 1 sind kann \(x_n\) nicht größer sein als 2,5.
        Nein, man kann die Folge nicht streng monoton wachsend wählen.
    \end{enumerate}
    
    \section*{A 4.3}
    \begin{enumerate}[ label= (\roman*) ]
        \item Konvergieren die folgenden Folgen. Wenn ja berechnen Sie auch den Grenzwert.
        \[a_n = \frac{ 5 n^5 + 4 n^3 - n + 5 }{ 10 n ^5 + n^2 - n + 100 } \]
        Ausklammern von \(n^5\)
        \[ 
            \frac{ 5 n^5 + 4 n^3 - n + 5 }{ 10 n ^5 + n^2 - n + 100}
            = \frac{n^5}{n^5} \left( \frac{5 + 4 \frac{1}{n^2} - \frac{1}{n^4} + 5 \frac{1}{n^5} }{ 10 + \frac{1}{n^3} - \frac{1}{n^4} + 100 \frac{1}{n^5} } \right)\
            = \frac{5 + 4 \frac{1}{n^2} - \frac{1}{n^4} + 5 \frac{1}{n^5} }{ 10 + \frac{1}{n^3} - \frac{1}{n^4} + 100 \frac{1}{n^5} }
        \]
        Wir wissen das \(\frac{1}{n}\) gegen 0 konvergiert. Mit Satz 2.27(e) folgt daraus dass auch \(\frac{1}{n} \frac{1}{n} = \frac{1}{n^2}\) gegen 0 konvergiert.
        Ebenso alle weiteren Folgen \(n^{-x}\). Mit Satz 2.27(c) und Satz 2.27(e) folgt das eine Summe von \(\lambda \: n^{-x}\) ebenfalls gegen Null konvergiert.
        Bildet man nun den Grenzwert von \(a_n\) so gehen alle Terme außer der 5 und der 10 gegen Null. Der Grenzwert ist also:
        \[
            \lim_{n \to \infty} \frac{5 + 4 \frac{1}{n^2} - \frac{1}{n^4} + 5 \frac{1}{n^5} }{ 10 + \frac{1}{n^3} - \frac{1}{n^4} + 100 \frac{1}{n^5} }
            = \frac{5}{10} = \frac{1}{2}
        \]
        Die Folge konvergiert gegen ein Halb.

        \[b_n := \frac{n^2}{2^n} \]
        Die Folge \(b_n\) lässt sich nach unten abschätzen gegen 0, da sowohl Nenner als auch Zähler immer größer als 0 sind.
        Außerdem lässt sie sich nach oben abschätzen. Nach Lemma 2 ist \(2^n\) immer größer als \(n^3\) für \(n\) genügend groß.
        \( \frac{1}{2^n} \) ist also immer kleiner als \( \frac{1}{n^3} \)
        Schätzt man also nach oben ab ergibt sich.
        \[ \frac{n^2}{2^n} < \frac{n^2}{n^3} = \frac{1}{n} \]
        Die Folge \(\frac{1}{n}\) geht gegen 0.
        Nach dem Sandwichtheorem Satz 2.27(f) gilt also:
        \[0 \leq b_n \leq \frac{1}{n} \]
        \(b_n\) konvergiert also gegen 0.

        \item Es sei \( {(a_n)}_{n \in \mathbb{N} } \) die Fibonacci-Folge, also \( a_1 = a_2 := 1 \) und für \( n \geq 3 \) ist \(a_n := a_{n-1} + a_{n-2} \).
        Wir definieren nun
        \[ u_n := \frac{ a_{n+1} }{ a_n } \]
        Die Folge \( {(u_n)}_{ n \in \mathbb{N} } \) konvergiert gegen ein \( u \in \mathbb{R} \). Berechnen Sie u. \\
        Wir fangen an und formen die Definition der Folge um. Wir setzen die Definition für \(a_{n+1}\) ein und vereinfachen.
        \[ \frac{ a_{n+1} }{ a_n } = \frac{ a_n + a_{n-1} }{ a_n } = 1 + \frac{ a_{n-1} }{ a_n } \]
        Wir wollen nun die Definition von \(u_n\) wieder einsetzen um \(u_n\) rekursiv zu definieren.
        \[1 + \frac{ a_{n-1} }{ a_n } = 1 + \frac{1}{ \frac{ a_n }{ a_{n-1} } } = 1 + \frac{1}{ u_{n-1} } \]
        Da \(u_n\) gegen \(u\) konvergiert, konvergiert auch \(u_{n-1}\) gegen \(u\). Wir bilden auf beiden Seiten den Grenzwert
        \begin{align*}
            \lim_{n \to \infty} u_n &= \lim_{n \to \infty} 1 + \frac{1}{u_{n-1}} \\
            \Leftrightarrow u &= 1 + \frac{1}{u} \: | \: -u \\
            \Leftrightarrow 0 &= -u + 1 + \frac{1}{u} \: | \: \cdot u \\
            \Leftrightarrow 0 &= -u^2 + u + 1 \: | \: \cdot (-1) \\
            \Leftrightarrow 0 &= u^2 - u - 1
        \end{align*}
        Diese Gleichung lässt sich mit der pq-Formel lösen.
        \begin{align*}
             u_1, u_2 &= - \left( -\frac{1}{2} \right) \pm \sqrt{ {\left( -\frac{1}{2} \right)}^2 - (-1) } \\
             \Leftrightarrow u_1, u_2 &= \frac{1}{2} \pm \sqrt{ \frac{5}{4} } \\
             \Leftrightarrow u_1, u_2 &= \frac{ 1 \pm \sqrt{5} }{2} 
        \end{align*}
        Da \( \frac{ 1 - \sqrt{5} }{2} \) negativ ist, die Folge aber aus Konstruktion nich negativ werden kann,
        ist der Grenzwert \( u = \frac{ 1 + \sqrt{5} }{2} \)

    \end{enumerate}
    
    \section*{A 4.4}
        Gibt es Folgen \( {(x_n)}_{n \in \mathbb{N}} \), sodass:
        \begin{enumerate}[ label = (\roman*)]
            \item Für alle \( x \in \mathbb{R} \) gibt es eine Teilfolge \( (x_{n_k}) \), sodass \(x_{n_k} \to x\) für \( k \to \infty \)
            Wähle \(x_n\) als Folge von rationalen Zahlen nach dem Cantorsches Diagonalargument.
            Bilde die Teilfolge von von \((x_n)\) die gegen \(x\) konvergiert wie folgt:
            Jedes Folgenglied von \( (x_{n_k}) \) die Bedingung \( x - \frac{1}{k} < x_{n_k} < x + \frac{1}{k} \) erfüllen.
            Zusätzlich wähle für das k.te Folgenglied ein Folgenglied aus \((x_n)\) das hinter dem (k-1).ten Folgenglied in \((x_n)\) kommt.
            Für das erste Folgenglied wähle beliebig.

            \textbf{Zu zeigen \(x_{n_k}\) ist immer definiert} \\
            Da es nach Lemma 4 unendlich viele rationale Zahlen gibt, mit denen man mit beliebig Genauigkeit \( x_{n_k} \) approximieren kann,
            liegt auch immer einer dieser Zahlen im gegebenen Intervall um \( x_{n_k} \). Diese Zahl oder eine ihrer unendlich vielen
            besseren Approximation liegt damit auch in \( (x_n) \) nach der (k-1).ten Stelle. Das es vor \(x_{n_{(k-1)}}\) nur endlich viele Zahlen in \( (x_n) \) gibt.

            Jedes Folgenglied von \((x_{n_k})\) liegt nach Konstruktion zwischen den beiden Werten der Folgen
            \( {(x - \frac{1}{k}) }_{k \in \mathbb{N}}\) und \( { (x + \frac{1}{k}) }_{k \in \mathbb{N}} \).
            Beide Folgen konvergieren, da sie nur eine Summe aus Konstante und der Folge \(\frac{1}{n}\) sind gegen \(x\).

            Wendet man nun Satz 2.27(f) (Sandwichtheorem) an, so ergibt sich:
            \begin{align*}
                \lim_{k \to \infty} x - \frac{1}{k} \leq \lim_{k \to \infty} &x_{n_k} \leq \lim_{k \to \infty} x - \frac{1}{k} \\
                \Rightarrow x \leq &\lim_{k \to \infty} x_{n_k} \leq x
            \end{align*}
            Die Folge \((x_{n_k})\) konvergiert gegen \(x\).
            Somit gibt es eine Folge deren Teilfolgen gegen jeden Wert \( x \in \mathbb{R} \) konvergieren.
            
            \item Für alle \( x \in (0, 1) \) gibt es eine Teilfolge \( (x_{n_k}) \), sodass \(x_{n_k} \to x\) für \( k \to \infty \), aber für \(y \notin (0, 1)\) existiert keine solche Teilfolge. \\
            Angenommen es gäbe eine solche Folge \(x_n\).
            Sei \( {(y_n)}_{n \in \mathbb{N}} := \frac{1}{n} \). Die Folge \( (y_n) \) ist immer größer als 0 und konvergiert gegen 0.
            Sei nun \( {(z_n)}_{n \in \mathbb{N}} \) eine Teilfolge von \((x_n)\).
            Für das erste Element von \(z_n\) wähle man ein beliebiges Element aus dem Intervall \( (\frac{1}{8}, \frac{3}{8}) \), dass ein Folgenglied von \((x_n)\) ist. 
            Alle weiteren Elemente wähle man so, dass \(0 < z_n < y_n\) für ein bestimmtes \(n\) gilt. \\
            \textbf{Zu zeigen, \(z_n\) ist immer definiert:} \\
            \((x_n)\) hat Teilfolgen, so dass diese für alle \(x \in (0, 1)\) konvergieren. Es existiert also auch eine Teilfolge die gegen \(\frac{1}{4}\) konvergiert.
            Nach der Definition der Konvergenz eine Folge (Teilfolge) gilt:
            \[ | x_{n_k} - \frac{1}{4} | < \epsilon \ \forall \epsilon > 0 \]
            Dies gilt also insbesondere auch für \(\epsilon = \frac{1}{8}\). Damit gilt:
            \[ | x_{n_k} - \frac{1}{4} | < \frac{1}{8} \]
            Der Abstand von \(x_{n_k}\) zu \(\frac{1}{4}\) ist also kleiner als \(\frac{1}{8}\). Damit liegt \(x_{n_k}\) im Intervall \( (\frac{1}{8}, \frac{3}{8}) \),
            also liegt ein Folgenglied von \((x_n)\) im Intervall was man für \(z_1\) wählen kann.

            Jedes \(z_n\) soll kleiner als \(y_n\) sein.
            Es existiert immer ein \( 0 < b:= \frac{y_{n}}{2} < y_{n}\).
            Weiterhin existiert auch immer ein Folgenglied in \((x_n)\) das echt kleiner ist als \(y_{n}\),
            weil es (analog wie für das erste Folgenglied) immer eine Teilfolge \((a_{m})\) gibt, die gegen \(b\) konvergiert.
            Wähle \( \epsilon = y_{n} - \frac{y_{n}}{2} \)
            \[ |a_{n} - \frac{y_{n}}{2} | < \frac{y_{n}}{2} \]
            Damit gibt es ein Element von \((x_n)\) das in diesem Intervall \((0, \frac{y_{n}}{2})\) liegt.
            Ab einem genügend großen \(N' \in \mathbb{N}\) sind alle \(a_m\) im oben gegebenen Intervall.
            Von diesen unendlich vielen Elementen können nur endlich viele vor \(z_n\) liegen, mann wähle eines der anderen Elemente.
            
            Es gilt somit:
            \[ 0 < z_n < y_n \]
            Nach dem Sandwichtheorem Satz 2.27(f) gilt:
            \[ \lim_{n \to \infty} 0 \leq \lim_{n \to \infty} z_n \leq \lim_{n \to \infty} y_n \]
            \[ \Rightarrow 0 \leq z \leq 0\]
            Wobei hier \(z\) der Grenzwert von \(z_n\) ist. Die hier konstruierte Folge \((z_n)\) konvergiert gegen 0. Dies ist aber ein Wiederspruch zu der Annahme,
            dass alle Teilfolgen von \((x_n)\) nur im Intervall \(0, 1\) konvergieren.
            Somit gibt es keine Folge \((x_n)\) die diese Eigenschaft hat.

        \end{enumerate}

    \newcounter{lemma_counter}
    \newtheorem{lemma1}[lemma_counter]{Lemma}
    \begin{lemma1}
        \[ 2n -1 \geq n \: \forall \: n \in \mathbb{N} \]
        Die Aussage beweisen wir mit vollständiger Induktion. \\
        \textbf{Induktionsanfang \(n=1\)}
        \[2n - 1 = 2 \cdot 1 - 1 = 1 = n\]
        \textbf{Induktionsannahme:}
        \[2n - 1 \geq n \Rightarrow 2(n+1) -1 \geq (n+1)\]
        \textbf{Induktionsschritt:} \\
        Wir zeigen, dass die Aussage für \(n+1\) gilt, indem wir sie zu einer wahren Aussage führen.
        \begin{align*}
            2 (n+1) - 1 & \geq n + 1 \\
            \Leftrightarrow 2n + 2 - 1 & \geq n + 1 \\
            \Leftrightarrow 2n + 1 & \geq n + 1 \\
            \Leftrightarrow 2n & \geq n \\
            \Leftrightarrow 2 & \geq 1 
        \end{align*}
   
        Durch Umformungen erhalten wir eine wahre Aussage. Damit ist bewiesen das \(2n - 1\) immer größer gleich als \(n\) ist.
    \end{lemma1}

    \newtheorem{lemma2}[lemma_counter]{Lemma}
    \begin{lemma2}
        \[ n^3 \leq 2^n \text{ für } n \geq 10 \]
        Die Aussage beweisen wir mit vollständiger Induktion. \\
        \textbf{Induktionsanfang \(n=10\)}
        \[10^3 = 1000 \leq 1024 = 2^{10} \]
        \textbf{Induktionsannahme:}
        \[ n^3 \leq 2^n \Rightarrow {(n+1)}^3 \leq 2^{ n+1 } \text{ für } n \geq 10 \]
        \textbf{Induktionsschritt:} \\
        Klammern auflösen
        \begin{align*}
            {(n+1)}^3 & \leq 2^{ n+1 } \\
            \Leftrightarrow {(n+1)}^3 & \leq 2 \cdot 2^n \\
            \Leftrightarrow n^3 + 3n^2 + 3n + 1 & \leq 2 \cdot 2^n
        \end{align*} 
        Nach Lemma 3 lässt sich \(3n^2 + 3n + 1\) nach oben gegen \(n^3\) abschätzen.
        \[ n^3 + 3n^2 + 3n + 1 \leq n^3 + n^3 \]
        Nach Induktionsannahme kann man \(n^3\) nach oben gegen \(2^n\) abschätzen.
        \[ n^3 + n^3 = 2 n^3 \leq 2 \cdot 2^n \leq 2^{n+1} \]
        Es gilt also insgesamt:
        \[n^3 + 3n^2 + 3n + 1 \leq n^3 + n^3 = 2 n^3 \leq 2 \cdot 2^n \leq 2^{n+1} \]
        Was zu beweisen war. Damit gilt die Anfangsausagen nach dem Prinzip der vollständigen Induktion.
    \end{lemma2}

    \newtheorem{lemma3}[lemma_counter]{Lemma}
    \begin{lemma3}
        \[n^3 \geq 3n^2 + 3n + 1 \text{ für } n \geq 10 \]
        Die Aussage beweisen wir mit vollständiger Induktion. \\
        \textbf{Induktionsanfang \(n=10\)}
        \[10^3 = 1000 \geq 331 = 300 + 30 + 1 = 3 \cdot 10^2 + 3 \cdot 10 + 1\]
        \textbf{Induktionsannahme:}
        \[ n^3 \geq 3n^2 + 3n + 1 \Rightarrow {(n+1)}^3 \geq 3{(n+1)}^2 + 3(n+1) + 1 \text{ für } n \geq 10 \]
        \textbf{Induktionsschritt:} \\
        Klammern auflösen:
        \begin{align*}
            {(n+1)}^3 & \geq 3{(n+1)}^2 + 3(n+1) + 1 \\
            \Leftrightarrow n^3 + 3n^2 + 3n + 1 & \geq (n^2 + 2n + 1) + (3n + 1) + 1 \\
            \Leftrightarrow n^3 + 3n^2 + 3n + 1 & \geq n^2 + 5n + 3 \: | -3n^2, -3n, -1 \\
            \Leftrightarrow n^3 & \geq -3n^2
        \end{align*}
        Weil \(n \in \mathbb{N}\) ist \(3n^2\) immer größer 0. Und \(-3n^2\) immer kleiner 0.
        \(n^3 \geq -3n^2\) ist also eine wahre Aussage. Somit gilt die Induktion.
    \end{lemma3}

    \newtheorem{lemma4}[lemma_counter]{Lemma}
    \begin{lemma4}
        Zu zeigen: jede reelle Zahl lässt sich beliebig nah durch beliebig viele verschiedene rationale Zahlen immer besser approximieren. \\
        Es gilt: 
        \[\forall \: \epsilon > 0 \: \forall \: x \in \mathbb{R} \: \exists \: q_{e,x} \in \mathbb{Q}: |q_{e,x} - x| < \epsilon \]
        \begin{center}
            Eine reele Zahl lässt sich beliebig nah durch eine rationale Zahl approximieren.
        \end{center}
        Aus dieser Aussage folgt die Existenz einer reelen Zahl \(q_1\) \\
        Nun sei der Abstand von \(q_1\) zu \(x\): \( |q_1 - x| =: \epsilon_1 \) \\
        Setzt man dieses Epsilon nun in die Aussage ein, so erhält man: \( |q_2 - x| < \epsilon_1 \) \\
        Dieses \(q_2\) ist echt kleiner als \(q_1\), da gilt:
        \( |q_2 - x| < \epsilon_1 \Leftrightarrow |q_2 - x| < |q_1 - x| \) \\
        Durch beliebig oft wiederholtes Anwenden dieser zwei Schritte, erhält man eine Folge von \(q \in \mathbb{Q} \) die sich \(x\) immer weieter annähert. 
    \end{lemma4}

\end{document}