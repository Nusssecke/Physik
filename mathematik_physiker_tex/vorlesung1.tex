\documentclass{article}

\usepackage{babel}[german]
\usepackage{amsfonts}
\usepackage{hyperref}

\title{Mengen, Schreibweisen, Aussagen}
\author{Finn Wagner}
\date{Heute}

\begin{document}

\maketitle

\section{Grundlagen}
Definition einer Menge: "Unter einer Menge verstehen wir hede Zusammenfassung
von bestimmten wohl unterschiedenene Objekten 'm' unserer Anschauung oder
unseres Denkens (welche Elemente von 'M' gennant werden) zu einem Ganzen."
Alle Objekte unterscheidbar; Nicht zweimal das Gleiche!

\section{Beispiel 1.1}
$\mathbb{N}=\{0, 1, 2, \ldots\}$
Mengenklammern. Die Menge der natürlichen Zahlen
$\mathbb{Z}=\{\ldots, -2, -1, 0, 1, 2, \ldots\}$
Die Menge der ganzen Zahlen
$M = \{Apfel, Birne, Ananas\}$
Eine Menge mit beliebigen Elementen.

\section{Konstruktion der natürlichen Zahlen}
\textbf{Existenz einer unendlichen Menge}
$\exists x [\emptyset \in x \land \forall y (y \in x \rightarrow y \cup \{y\} \in x)]$\\
Sagt man das y $y \cup \{y\}$ die nachfolgende Menge (oder Nachfolger) von y
ist, so bedeutet dies (es folgt) die Existenz einer Menge x mit
N(x) $\emptyset \in x$ und für alle $y \in x$ ist der Nachfolger $y \cup \{y\}\in x$

\textbf{Hauptsatz}
Es gibt eine kleinste Menge, mit $\mathbb{N}$ bezeichnet, die
die Eigenschaft N besitzt, d.h. es gilt N($\mathbb{N}$) und für alle
Mengen x mit N(x) gilt $x \ni \mathbb{N}$
Mann definiere: $0=\emptyset, 1=0\cup\{\emptyset\}=$
\href{https://www.mathematik.uni-marburg.de/~portenier/Analyse/Skript/nat-zahlen.pdf}{Weiteres}

\section{Gängige Schreibweisen}
\begin{enumerate}
    \item Mengen werden mit Großbuchstaben bezeichnet. Elemente von Mengen werden
    mit Kleinbuchstaben bezeichnet.
    Gehört ein Objekt m zu einer Menge M , so schreibt man:
    $m \in M$, wenn nicht $m \notin M$
    Beispiel: $-1 \in \mathbb{Z}; -1 \notin \mathbb{N}$
    \item Ist E(m) eine Aussagenform für eine Element so bezeichnet


\end{enumerate}

\end{document}