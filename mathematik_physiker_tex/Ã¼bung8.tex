\documentclass{article}

\usepackage[a4paper, top=2cm, bottom=3cm]{geometry}
\usepackage{babel}[german]
\usepackage{booktabs}
\usepackage{mathtools}
\usepackage{amssymb}
\usepackage{enumitem}
\usepackage{amsmath}
\newcommand{\proofeq}{\overset{!}{=}}
\newcommand{\proofeqv}{\overset{!}{\Leftrightarrow}}
\newcommand{\equivalent}{\overset{\scriptscriptstyle\wedge}{=}}


\date{12.12.2021}
\title{Aufgabenblatt 8, Mathematik für Physiker 1}
\author{Florian Adamczyk, Finn Wagner}

\begin{document}
    \maketitle

    \section*{A 8.1}
    Es seien \(x,y \in (0,\infty)\). Zeigen Sie, dass
    \[ \ln(xy) \proofeq \ln(x) + \ln(y) \]
    Es gilt die in Kapitel 4.1.3 hergeleitete Funktionalgleichung der \(e\) Funktion:
    \[ e^{a+b} = e^a \cdot e^b \text{ mit } a,b \in \mathbb{R} \]
    Wir setzen \(a := \ln(x) \) und \(b := \ln(y) \) mit \(a,b\) im Definitonsbereich der natürlichen Logarithmusfunktion also \(x,y \in (0, \infty ) \) \\
    Damit folgt für \(e^a = e^{\ln(x)} = x\) und für \(e^b = e^{\ln(y)} = y\), mit dem natürlichen Logarithmus als Umkehrfunktion der Exponentialfunktion (Definiton 4.30)
    Es gilt also:
    \[e^{a+b} = e^a \cdot e^b = x \cdot y \]
    Wir wenden nun den natürlichen Logarithmus auf beide Seiten der Gleichung an, da er stetig und monoton wachsend ist folgt:
    \begin{gather*}
        \ln(e^{a+b}) = \ln(xy) \\
        \Leftrightarrow \ln(e^{\ln(x) + \ln(y)}) = \ln(x) + \ln(y) = \ln(xy)
    \end{gather*}
    
    \section*{A 8.2}
    Berechnen Sie die folgenden Grenzwerte für \(\alpha \in \mathbb{R} \)
    \begin{enumerate}[ label = (\roman*) ]
        \item \( \lim_{x \to 0} x^{\alpha} \ln(x) \) \\
        
        \begin{enumerate}[ label = \arabic*. Fall]
            \item Ist \(\alpha > 0\): \\
            Wir definieren \(y := \frac{1}{x} \Leftrightarrow x = \frac{1}{y}\):
            Um die Variable des Grenzwerts ersetzen zu können, müssen den Grenzwert von \(y\) berechnen, wenn \(x \to 0\). \\
            \( \lim_{x \to 0} y(x) = \lim_{x \to 0} \frac{1}{x} = \infty \). (Rechtsseitiger Grenzwert, da Logarithmus nur auf positiven Zahlen definiert) \\
            Wir setzen \(x = \frac{1}{y}\) und ersetzen \(\lim_{x \to 0}\) mit \(\lim_{y \to \infty}\)
            \[ \lim_{x \to 0} x^{\alpha} \ln(x) = \lim_{y \to \infty} { \left( \frac{1}{y} \right) }^{\alpha} \ln(\frac{1}{y}) \]
            Wir formen nun die Exponenten um und wenden das dritte Logarithmusgesetze an, das aus der Defintion der verallgemeinerten Exponentialfunktion folgt:
            \begin{gather*}
                \lim_{y \to \infty} { \left( \frac{1}{y} \right) }^{\alpha} \ln(\frac{1}{y}) = \lim_{y \to \infty} \frac{1}{y^{\alpha}} \ln(y^{-1}) \\
                = \lim_{y \to \infty} -\frac{1}{y^{\alpha}} \ln(y) = \lim_{y \to \infty} -\frac{\ln(y)}{y^{\alpha}} 
            \end{gather*}
            Sei nun \(z := \alpha \ln(y) \Leftrightarrow y = e^{\frac{z}{\alpha}} \). Wir berechnen den Grenzwert von \(z\), wenn \( x \to \infty \) \\
            \( \lim_{y \to \infty} \alpha \ln(y)\) konvergiert nach Proposition 4.31 (c) gegen \( \infty \)
            Wir formen nun zuerst um mit Definiton 4.32, der verallgemeinerten Exponentialfunktion und setzen dann \(z\) und den neuen Limes ein:
            \begin{gather*}
                \lim_{y \to \infty} -\frac{\ln(y)}{y^{\alpha}} = \lim_{y \to \infty} -\frac{\ln(y)}{e^{\alpha \ln(y)}} 
                = \lim_{z \to \infty} -\frac{\ln(y)}{e^{\alpha \ln(y)}} \\
                = \lim_{z \to \infty} -\frac{\ln(e^{\frac{z}{\alpha}})}{e^{z}}
                = \lim_{z \to \infty} -\frac{\frac{z}{\alpha}}{e^z} = -\frac{1}{\alpha} \lim_{z \to \infty} \frac{z}{e^z} \\
                = -\frac{1}{\alpha} \lim_{z \to \infty} z e^{-z}
            \end{gather*}
            Aus A8.2 (ii) mit \(\alpha = 1\) folgt:
            \[ -\frac{1}{\alpha} \lim_{z \to \infty} \frac{z}{e^z} = -\frac{1}{\alpha} \cdot 0 = 0 \]

            \item Ist \(\alpha \leq 0\)
            Wir definieren \(\beta = |\alpha| \), \(\Rightarrow \beta > 0\)
            \[\lim_{x \to 0} x^{\alpha} \ln(x) = \lim_{x \to 0}  x^{-\beta} \ln(x) = \lim_{x \to 0} \frac{\ln(x)}{x^{\beta}} \]
            Da man bei stetigen Funktionen hier \(\ln(x)\)(aus Definiton 4.30) und \(\frac{1}{x^{\beta}}\)(als Verkettung stetiger Funktionen Satz 4.13)
            die Funktion und den Grenzwert tauschen kann (Bem. 4.11) gilt:
            \[ \lim_{x \to 0} \frac{\ln(x)}{x^{\beta}} = \lim_{x \to 0} \ln(x) \cdot \lim_{x \to 0} \frac{1}{x^{\beta}}\]
            In Prop 4.31(c) ist gegeben \(\lim_{x \to 0} \ln(x) = -\infty \).
            \(\frac{1}{x^{\beta}}\) konvergiert, da wir hier den rechtsseitigen Grenzwert betrachten (Logarithmus nur auf positiven Zahlen definiert)
            und da \(\beta > 0\) mit Proposition 4.33(b) gegen \( \infty \).
            Es folgt:
            \[\lim_{x \to 0} \ln(x) \cdot \lim_{x \to 0} \frac{1}{x^{\beta}} = - \infty \]

        \end{enumerate}

        \item \( \lim_{x \to \infty} x^{\alpha} e^{-x}\) \\
        Wir leiten zuerst eine Abschätzung der e-Funktion durch ihre Reihendarstellung her. Wir lassen alle bis auf einen der unendlich vielen Summanden weg:
        \begin{gather*}
            e^x = \sum_0^{\infty} \frac{x^k}{k!} > \frac{x^{k+1}}{(k+1)!} \text{ mit } k \in \mathbb{N} \\
            \Rightarrow \frac{x^{k+1}}{(k+1)!} > \frac{x^{\alpha +1}}{(\alpha + 1)!} \text{ mit } k \geq \alpha \text{ und } \alpha \in \mathbb{R}
        \end{gather*}
        Hierraus folgt:
        \begin{gather*}
            e^x > \frac{x^{\alpha +1}}{(\alpha + 1)!} \ | \ \cdot (\alpha + 1)! \ | \ :  e^x
            \Leftrightarrow (\alpha + 1)! > \frac{x^{\alpha +1}}{e^x} \ | \ :x \\
            \Leftrightarrow \frac{(\alpha+1)!}{x} > x^{\alpha} e^{-x}
        \end{gather*}
        Sei O.B.d.A \(x > 0\), es spielt hier für den Grenzwert gegen Unendlich keine Rolle.
        Damit ist \(x^{\alpha} > 0\), da \(x > 0\) und der Exponent die Zahl nicht negativ machen kann. Und \(e^{-x} > 0\) nach Beispiel 2.45.
        Es gilt damit
        \[ \frac{(\alpha+1)!}{x} > x^{\alpha} e^{-x} > 0 \]
        Wir wenden das Sandwichkriterium an:
        \begin{gather*}
            \lim_{x \to \infty} \frac{(\alpha+1)!}{x} > \lim_{x \to \infty} x^{\alpha} e^{-x} > \lim_{x \to \infty} 0 \\
            (\alpha+1)! \lim_{x \to \infty} \frac{1}{x} \geq \lim_{x \to \infty} x^{\alpha} e^{-x} \geq \lim_{x \to \infty} 0 \\
            0 \geq \lim_{x \to \infty} x^{\alpha} e^{-x} \geq 0
        \end{gather*}
        Damit ist \( \lim_{x \to \infty} x^{\alpha} e^{-x} = 0\)

    \end{enumerate}

    \section*{A 8.3}
    Wir formulieren die Summendarstellung des Polynoms um mit \(k' = 2k + 1\)
    \[P(x) = \sum_{k=0}^{n} a_{(2k + 1)} x^{2k + 1} \]

    Sei \(P: \mathbb{R} \to \mathbb{R}\) ein Polynom ungeraden Grades also
    \[P(x) = \sum_{k=0}^{n} a_k x^{k}, \quad a_k \in \mathbb{R}, \ a_n \neq 0, \ n \in \mathbb{N} \text{ ungerade}\]
    Zeigen Sie das P eine Nullstelle hat.
    
    O.B.d.A sei \(a_n > 0\)
    Wir berechnen nun den Grenzwert des Polynoms für \(x \to \infty \)
    \begin{gather*}
        \lim_{x \to \infty} \sum_{k=0}^{n} a_k x^{k}
        = \lim_{x \to \infty} \left( a_n \cdot x^n + a_{n-2} \cdot x^{n-2} + \dotsm + a_0 \cdot x^0 \right) \\
        = \lim_{x \to \infty} x^n \cdot \left( a_n + a_{n-2} \cdot \frac{x^{n-2}}{x^n} + \dotsm + a_0 \cdot \frac{1}{x^n} \right)
    \end{gather*}
    Da alle Monome stetig sind und mit Satz 4.13 auch alle ihre Linearkombinationen stetig sind, sowie das Produkt
    zweier stetiger Funktionen, ist auch der umgeformte Ausdruck stetig. Mit Bemerkung 4.11 ziehen wir den Grenzwert in den Ausdruck.
    \begin{gather*}
        \lim_{x \to \infty} x^n \cdot \left( a_n + a_{n-2} \cdot \frac{x^{n-2}}{x^n} + \dotsm + a_0 \cdot \frac{1}{x^n} \right) \\
        = \lim_{x \to \infty} x^n \cdot \lim_{x \to \infty} \left( a_n + a_{n-2} \cdot \frac{x^{n-2}}{x^n} + \dotsm + a_0 \cdot \frac{1}{x^n} \right)
    \end{gather*}
    Der hintere Grenzwert ist \(a_n\),
    da alle restlichen Terme Vielfache der Reihe \(\frac{1}{n^{\alpha}}, \ \alpha \in \mathbb{R}\) sind und diese gegen 0 konvergieren.
    Übrig bleibt:
    \[ \lim_{x \to \infty} \sum_{k=0}^{n} a_k x^{k} = a_n \lim_{x \to \infty} x^n = \infty \]
    Da \(a_n > 0\) konvergiert dies zu \(\infty \). \\
    Analog konvergiert
    \[ \lim_{x \to -\infty} \sum_{k=0}^{n} a_k x^{k} = - \infty \]
    Da die Funktion gegen \(\infty \) und \(-\infty \) konvergiert,
    existiert ein Intervall \([a,b]\) mit \(P(a) \cdot P(b) < 0\). Und da \(P(x)\) als Polynom auf ganz \(\mathbb{R}\) stetig ist,
    können wir den Zwischenwertsatz anwenden. \\
    Das Polynom ungeraden Grades nimmt also alle Werte in \(\mathbb{R}\) an, wird also auch insbesondere 0.


    \section*{A 8.5}
    {\Large Lies dir die Aufgabe mal durch und schau mal ob du verstehst was ich da so geschrieben habe. ;)}
    \begin{enumerate}[ label = (\alph*)]
        \item \( \lim_{x \to x_0} f(x) = z_0 \proofeqv f(x_0-) = z_0 = f(x_0+) \) \\
        Die Funktion f sei definiert als \(f: I \to A\) \\
        ''\(\Rightarrow \)'' Hinrichtung \\
        Der bedseitige Grenzwert einer Funktion ist definiert als:
        \[ \lim_{x \to x_0} f(x) = z_0 \equivalent
        \text{ Für jede Folge } {(x_n)}_{n \in \mathbb{N}} \in I \backslash \{ x_0 \} \text{ mit } x_n \to x_0: \lim_{n \to \infty} f(x_n) = z_0 \]
        Der linksseitige Grenzwert (Definition 4.19) einer Funktion ist definiert als: 
        \[f(x_0-) = z_0 \equivalent \text{Für jede Folge } { (x_n)}_{n \in \mathbb{N}} \in I \backslash \{ x_0 \} \text{ mit }
             x_n \to x_0 \text{ und } x_n < x_0 \ \forall \ n \in \mathbb{N}: \lim_{n \to \infty} f(x_n) = z_0 \]
        Hierraus ist ersichtlich, das die Menge an Folgen aus der Definition des linksseitigen Grenzwerts,
        mit dem zusätzlichen Kriterium \((x_n < x_0)\),
        eine Teilmenge der Folgen aus der Definition des beidseitigen Grenzwerts ist. \\
        Der linksseitige Grenzwert existiert also da die Konvergenz seiner Folgen gegeben ist.
        Ebenso existiert der rechtsseitige Grenzwert da auch alle Folgen
        mit der Einschränkung \((x_n > x_0)\) in der Menge der Folgen des beidseitigen Grenzwerts vorhanden sind.
        Folglich gilt: \( f(x_0-) = z_0 = f(x_0+) \) \\
        \linebreak
        ''\( \Leftarrow \)'' Rückrichtung \\
        Gegeben sind nun die Konvergenz des rechts- und linksseitigen Grenzwerts. Wir setzen ihre Definitionen ein (\(x_0 = y_0\)):
        \[f(x_0-) = z_0 \equivalent \text{Für jede Folge } { (x_n)}_{n \in \mathbb{N}} \in I \backslash \{ x_0 \} \text{ mit }
             x_n \to x_0 \text{ und } x_n < x_0 \ \forall \ n \in \mathbb{N}: \lim_{n \to \infty} f(x_n) = z_0 \]
        \[f(y_0+) = z_0 \equivalent \text{Für jede Folge } { (y_n)}_{n \in \mathbb{N}} \in I \backslash \{ y_0 \} \text{ mit }
             y_n \to y_0 \text{ und } y_n < y_0 \ \forall \ n \in \mathbb{N}: \lim_{n \to \infty} f(y_n) = z_0 \]
        
        Wir setzen nun die Definition der Konvergenz für \(\lim_{n \to \infty} f(x_n) = z_0\) ein:
        \begin{equation}
            \forall \ \varepsilon > 0 \ \exists \ N_x \in \mathbb{N}: |f(x_n) - z_0| < \varepsilon \ \forall \ n \geq N_x
        \end{equation}
        genauso für \(\lim_{n \to \infty} f(y_n) = z_0\)
        \begin{equation}
            \forall \ \varepsilon > 0 \ \exists \ N_y \in \mathbb{N}: |f(y_n) - z_0| < \varepsilon \ \forall \ n \geq N_y
        \end{equation}
        Sei \({(z_n)}_{n \in \mathbb{N}} \) eine Folge mit \(z_n \in I \backslash \{ z_0 \} \).
        Ihre Folgenglieder seien nur Elemente der Folgen \((x_n)\) und \((y_n)\), mit der Bedingung
        \[z_n = x_m \lor z_n = y_m \text{ mit } n \leq m, \ m \in \mathbb{N} \]
        Betrachten wir nun die Konvergenz von \(\lim_{n \to \infty} f(z_n)\)
        und setzen die Defintion der Konvergenz ein, erhalten wir:
        \[\forall \ \varepsilon > 0 \ \exists \ M \in \mathbb{N}: |f(z_n) - z_0| < \varepsilon \ \forall \ n \geq M \]
        Wir setzen nun \(M = \max \{N_x, N_y\} \). \\
        Ist das Folgenglied \(z_n\) gerade \(x_n\), so ist \( |f(x_n) - z_0| < \varepsilon \) wahr,
        da \(n > M \geq N_x\) also Gleichung (1) wahr ist. \\
        Ist das Folgenglied \(z_n\) gerade \(y_n\), so ist \( |f(y_n) - z_0| < \varepsilon \) wahr,
        da \(n > M \geq N_y\) also Gleichung (2) wahr ist.

        \(\lim_{n \to \infty} f(z_n) \) konvergiert also gegen \(z_0\).
        Da \((z_n)\), \((x_n)\) und \((y_n)\) beliebig waren konvergiert jede beliebige Folge \((z_n)\) gegen \(z_0\). \\
        Es gilt also nach Definition des beidseitigen Grenzwertes 
        \[\lim_{n \to \infty} f(z_n) = z_0 = \lim_{x \to x_0} f(x) = z_0\]

        \item \(f\) ist stetig in \(x_0 \in I\) \( \proofeqv \) \(f(x_0-) = f(x_0) = f(x_0+)\)
        Mit Satz 4.10 lässt sich die Stetigkeit von \(f\) im Punkt \(x_0\) umformulieren:
        \[f \text{ ist stetig in } x_0 \equivalent \lim_{x \to x_0} f(x) = f(x_0) \]
        Die Beweisführung ist equivalent zum Aufgabenteil \((a)\) mit der Ersetzung \(z_0 = f(x_0)\).
        Es handelt sich hier um einen Spezialfall bei dem der Grenzwert auch von der Funktion angenommen wird.

    \end{enumerate}

\end{document}