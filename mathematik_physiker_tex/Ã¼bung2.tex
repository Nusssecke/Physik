\documentclass{article}

%\usepackage[a4paper, total={170mm,257mm}]{geometry}
\usepackage[a4paper, top=2cm, bottom=3cm]{geometry}
\usepackage{babel}[german]
\usepackage{booktabs}
\usepackage{mathtools}
\usepackage{amssymb}
\usepackage{enumitem}

\date{28.10.2021}
\title{Aufgabenblatt 2, Mathematik für Physiker 1}
\author{Florian Adamczyk, Finn Wagner}

\begin{document}
    \maketitle
    
    \section*{A 2.1}
        \subsection*{(i)}
        \large Zeigen Sie, dass die Abbildung aus Bsp. 1.15 (b)
        \(f : \mathbb{N} \times \mathbb{N} \to \mathbb{N}, (n, m) \mapsto 2^{n-1} (2m -1)\)
        eine Bijektion ist.

        Hierzu sei die Umkehrfunktion (Inverse) \(g\) definiert als: \\
        Verfahren für \(x \in \mathbb{N}\) beliebig: 
        Man suche für \(x\) die größte Zweierpotenz die es restlos teilt. Nun nehme man den Logarithmus zur Basis 2 dieser Potenz.
        Hierzu addiere man 1. Dieser Wert ist \(n\).
        Für \(m\) teile man \(x\) durch \(n\), addiere 1 hinzu und teile durch 2.

        Es folgt das f, auf Grund der Existenz der Umkehrfunktion, bijektiv ist.
        
        \subsection*{(ii)}
            Zeigen Sie für \(M_1, M_2\) abzählbar, dass auch \(M_1 \times M_2\) abzählbar ist. \\
            In (i) ist \(f\) eine bijektive Abbildung \(\mathbb{N} \times \mathbb{N} \to \mathbb{N}\). \\
            Für \(M_1, M_2\) existieren, da beide Menge abzählbar sind, die injektiven Abbildungen \(\phi_1(n): M_1 \to \mathbb{N}\) und \(\phi_2(n): M_2 \to \mathbb{N}\) \\
            Sei \(g: (M_1, M_2) \to (\mathbb{N}, \mathbb{N}) \: (a, b) \mapsto (\phi_1(a), \phi_2(b))\).
            \(g\) ist, da \(\phi_1, \phi_2\) beide injektiv waren, auch injektiv.
            Nun verknüpft man f mit g. Die resultierende Funktion \(f \circ g\) ist, da f bijektiv ist, injektiv.
            Somit existiert mit \(f \circ g\) eine Funktion die injektiv von \(M_1 \times M_2 \to \mathbb{N}\) abbildet.
            Daraus folgt das \(M_1 \times M_2\) eine abzählbare Menge ist.

        \section*{A 2.2}
            \begin{enumerate}[label = (\roman*)]
                \item Es seien \(f: A \to B\) und \(g: B \to C\) injektiv. Zeigen Sie, dass dann auch \(g \circ f\) injektiv ist. \\
                f injektiv: \(\forall a_1, a_2 \in A \: f(a_1)=f(a_2) \Leftrightarrow a_1 = a_2\) \\
                g injektiv: \(\forall b_1, b_2 \in B \: g(b_1)=g(b_2) \Leftrightarrow b_1 = b_2\) \\
                Zu zeigen ist: \(\forall a_1, a_2 \in A \: g(f(a_1)) = g(f(a_2)) \Leftrightarrow a_1 = a_2 \) \\
                \(\forall a_1, a_2 \in A \: g(f(a_1)) = g(f(a_2))\), da g injektiv ist, folgt \(f(a_1) = f(a_2)\). Da aber auch f injektiv ist, folgt
                \(a_1 = a_2\) \\
                Die linke Richtung gilt auch, da \(f\) und \(g\) Funktionen sind.
                
                Damit ist \(g \circ f\) injektiv

                \item Zeigen Sie dass die Umkehrung falsch ist, indem Sie eine injektive Funktion \(g \circ f\) angeben, bei der \(f\) oder \(g\) nicht injektiv ist.\\
                \(f: \mathbb{R}_+ \to \mathbb{R}: a \mapsto a\) \\
                \(g: \mathbb{R} \to \mathbb{R}: a \mapsto a^2\) \\
                \(g \circ f: \mathbb{R}_+ \to \mathbb{R}: a^2\) \\
                f ist injektiv, g ist nicht injektiv, aber \(g \circ f\) ist wieder injektiv.
            \end{enumerate}

        \section*{A 2.3}
        \begin{enumerate}
            \item Annahme: \((0, 1) \subset \mathbb{R}\) ist abzählbar. Es folgt die Existenz einer surjektiven Abbildung
            \(\phi: \mathbb{N} \to (0, 1)\) \\
            Sei \(\phi(n)\) mit \(n \in \mathbb{N}\) \\
            \(\phi(n) = 0,a_1^n a_2^n a_3^n a_4^n \dots \)
            mit \(a_i^n \in \{0,1,2,3,4,5,6,7,8,9\} \)

            \item Sei nun \(z = 0,d_1 d_2 d_3 d_4 \dots \)
            mit
            \begin{equation*}
                d_i =
                \begin{cases}
                    1 & \text{für } a_n^n \neq 1 \\
                    2 & \text{alle anderen Fälle}
                \end{cases}
            \end{equation*}
            Diese Zahl weicht an der n-ten Nachkommastelle von allen \(\phi(n)\) ab.
            Sie ist somit anders als alle \(\phi(n)\), weil sie sich immer and der n-ten Nachkommastelle unterscheidet.

            \item Es gilt \(z \in (0,1 )\) aber \(z \notin \text{Bild}(\phi)\) was ein Wiederspruch zur Surjektivität von \(\phi \) ist.
            Also war die Annahme falsch. \(q.e.d\)
        \end{enumerate}

        \section*{A 2.4}
            \begin{enumerate}[label= (\roman*)]
                \item Es sei K ein Körper. Zeigen sie, dass \(a \cdot 0 = 0\) für alle \(a \in K\) \\
                Zu zeigen ist das das neutrale Element der Addition bei der Multiplikation mit einem anderen Element des Körpers sich selbst ergibt.
                \begin{equation*}
                    0 =^{\text{Erweitert mit} \: 0 \cdot a} 0 \cdot a - 0 \cdot a = (0+0) \cdot a - 0 \cdot a =^{\text{mit Distributivgesetz}} 0 \cdot a + 0 \cdot a - 0 \cdot a
                    = 0 \cdot a + 0 = 0 \cdot a
                \end{equation*}

                \item Zeigen sie das \(\mathbb{F}_n\) kein Körper ist falls \(n \in \mathbb{N}\) keine Primzahl ist. \\
                mit \(\mathbb{F}_n: (\mathbb{Z}_n, +, \cdot) \) \\
                und \(\mathbb{Z}_m: (\{0, \dots, m-1\}, +, \cdot \) \\
                wobei für \(\mathbb{Z}_m\) \(+, \cdot \) definiert sind als: \\
                \begin{align*}
                    + := \mathbb{Z}_m \times \mathbb{Z}_m \to \mathbb{Z}_m \: (z_1 + z_2) \mod z_m \\
                    \cdot := \mathbb{Z}_m \times \mathbb{Z}_m \to \mathbb{Z}_m \: (z_1 \cdot z_2) \mod z_m
                \end{align*}
                

                Ist n keine Primzahl, so lässt sie sich in ihre Primfaktoren zerlegen. Ein Primfaktor von n ist immer kleiener als n.
                Nun teilt man die Menge der Primfaktoren in zwei \(P_1, P_2\) mit \(P_1, P_2 \neq \emptyset \). Seien \(q_1, q_2\) das Produkt aller Zahlen
                der Mengen \(P_1\) und \(P_2\). Setzt man nun \(q_1 \text{und} q_2\) in \(\cdot \) ein. So ergibt \(q_1 \cdot q_2\) wieder n. \(n \mod n = 0\).
                Da in diesem Ausdruch aber weder \(q_1\) noch \(q_2\) die Null waren ist die Multiplikation nicht nullteilerfrei.
                Somit ist \(\mathbb{F}_n\) kein Körper.
                
                \item Für \(m \in \mathbb{N}\) definieren wir
                    \begin{equation*}
                        m\mathbb{Z} := \{m \cdot z | z \in \mathbb{Z}\}
                    \end{equation*}
                    Zeigen Sie dass \((m\mathbb{Z}, +)\) mit der von \(\mathbb{Z}\) induzierten Addition eine abelsche Gruppe ist. \\
                    Wir überprüfen die Gruppen Axiome: \\
                    \begin{enumerate}[label = (\alph*)]
                        \item Assoziativgesetz
                        \(\rightarrow \) Assoziativ mit \(+\) aus \(\mathbb{Z}\)
                        \item Existenz eines neutralen Elements:
                        Die 0 ist immer in \(m\mathbb{Z}\), weil \(0 \in \mathbb{Z}\) und \(m*0 = 0\)
                        \item Existenz eines inversen Elements:
                        \(\forall \: x \in m\mathbb{Z} \: \exists \: z \in \mathbb{Z}: x = m*z \Rightarrow (-x)= m*(-z)\)
                        Mit \((-x) \in m\mathbb{Z}, \: \text{da} (-z) \in \mathbb{Z}\)
                        \item Abgeschlossenheit von + auf \(\mathbb{Z}\):
                        \(+: m\mathbb{Z} \times m\mathbb{Z} \to^{!} m\mathbb{Z}\) \\
                        \(\forall a,b \in m\mathbb{Z} \: \exists \: x,y \mathbb{Z}\)
                        \(a := mx, b:= my)\)
                        \(\Rightarrow a+b = mx + my = m(x + y)\)
                        Weil \((x+y) \in \mathbb{Z}\) ist + auf \(m\mathbb{Z}\) abgeschlossen.
                        \item Abelsche Gruppe:
                        \(
                            a + b = mx + my = m (x + y) = m(y + x) = my + mx = b + a
                        \)
                    \end{enumerate}
                    Zeigen Sie weiter, dass für alle \(z \in \mathbb{Z}\) und \(a \in I\)(\(I\) ist die Menge der Vielfachen von m) gilt, dass \(az \in m\mathbb{Z}\): \\
                    Für beliebige \(a \in I\) und \(z \in \mathbb{Z}\) \\
                    \(a := m * b \text{ mit } b \in \mathbb{N} \Rightarrow a * z = m * b * z \Rightarrow az \in m\mathbb{Z} \) weil \(b*z \in \mathbb{Z}\)

                \end{enumerate}
\end{document}