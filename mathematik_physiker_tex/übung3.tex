\documentclass{article}

%\usepackage[a4paper, total={170mm,257mm}]{geometry}
\usepackage[a4paper, top=2cm, bottom=3cm]{geometry}
\usepackage{babel}[german]
\usepackage{booktabs}
\usepackage{mathtools}
\usepackage{amssymb}
\usepackage{enumitem}

\date{5.10.2021}
\title{Aufgabenblatt 3, Mathematik für Physiker 1}
\author{Florian Adamczyk, Finn Wagner}

\begin{document}
    \maketitle
    
    \section*{A 3.1}
       Seien \(a, b, c, d \in \mathbb{R}\)
       \begin{enumerate}[label = (\alph*)]
            \item 
                \begin{align*}
                    ab & \leq \frac{1}{2} (a^2 x^2 + \frac{b^2}{x^2}) \: \text{für alle} \: x \in \mathbb{R} \backslash \{0\} \\
                    \Leftrightarrow ab & \leq \frac{a^2 x^2}{2} + \frac{b^2}{2x^2} \: | \cdot 2 \\
                    \Leftrightarrow 2ab & \leq a^2 x^2 + \frac{b^2}{x^2} \: | -2ab \\
                    \Leftrightarrow 0 & \leq a^2 x^2 + \frac{b^2}{x^2} - 2ab \: | \: \text{mit } x^2 \text{ erweitert} \\ 
                    \Leftrightarrow 0 & \leq \frac{a^2 x^4}{x^2} + \frac{-2abx^2}{x^2} + \frac{b^2}{x^2} \\
                    \Leftrightarrow 0 & \leq \frac{1}{x^2} (a^2 x^4  - 2abx^2 + b^2) \: | \cdot x^2 \: \text{Binomische Formel} \\
                    \Leftrightarrow 0 & \leq {(ax^2 - b)}^2 \\
                    & \text{Ein Quadrat ist immer größer gleich Null}
                \end{align*}
            \item 
                \begin{gather*}
                    ab + bc + ac \leq a^2 + b^2 + c^2 \ | \: \cdot 2\\
                    \Leftrightarrow 2ab + 2bc + 2ac \leq 2a^2 + 2b^2 + 2c^2 \\
                    \text{Umformen mit: } -2ab -2bc -2ac \text{ und Umsortieren}\\
                    \Leftrightarrow 0 \leq (a^2 -2ab + b^2) + (a^2 -2ac + c^2) + (b^2 -2bc + c^2) \\
                    \text{2.te Binomische Formel anwenden:} \\
                    \Leftrightarrow 0 \leq {(a - b)}^2 + {(a - c)}^2 + {(b - c)}^2 \\
                    \text{Ein Quadrat ist immer größer gleich Null.} \\
                    \text{Eine Summe aus Quadraten ist ebenfalls immer größer gleich Null}
                \end{gather*}
            \item 
                \begin{gather*}
                    \frac{a}{b} < \frac{c}{d} \Leftrightarrow^! \frac{a}{b} < \frac{a+c}{b+d} < \frac{c}{d} \\
                    \text{Vordere Ungleichung} \\
                    \frac{a}{b} <^! \frac{a+c}{b+d} \Leftrightarrow a \cdot (b + d) < b \cdot (a + c) \\
                    \Leftrightarrow ab + ad < ba + bc \: | \: - ab \\
                    \Leftrightarrow ad < bc \: | \: /b \: /d\\
                    \Leftrightarrow \frac{a}{b} < \frac{c}{d} \\
                    \\
                    \text{Hintere Ungleichung} \\
                    \frac{a+c}{b+d} <^! \frac{c}{d} \Leftrightarrow (a+c)\cdot d < (b+d)\cdot c \\
                    \Leftrightarrow ad + cd < bc + dc \: | \: -cd \\
                    \Leftrightarrow ad < bc \: | \: /b \: /d \\
                    \Leftrightarrow \frac{a}{b} < \frac{c}{d}
                \end{gather*}
            \item 
                Zu zeigen: \\
                $n \in \mathbb{N}  \setminus \{0\} :\, a^n - b^n = (a-b) \cdot \sum_{k=0}^{n-1} a^k b^{n-1-k} $ \\
                \\
                Summen ausmultiplizieren:
                $$ a^n-b^n = a \cdot \sum_{k=0}^{n-1} a^k \, b^{n-1-k} - b \cdot \sum_{k=0}^{n-1} a^k \, b^{n-1-k} $$
                Vorfaktoren in Summen ziehen: 
                $$ = \sum_{k=0}^{n-1} a^{k+1} \, b^{n-1-k} - \sum_{k=0}^{n-1} a^k \, b^{n-k} $$
                Indexshift bei der hinteren Summe: $z=k-1 \Leftrightarrow k=z+1$
                $$ = \sum_{k=0}^{n-1} a^{k+1} \, b^{n-1-k} - \sum_{z+1=0}^{n-2} a^{z+1} \, b^{n-(z+1)} $$
                $$ = \sum_{k=0}^{n-1} a^{k+1} \, b^{n-1-k} - \sum_{z=-1}^{n-2} a^{z+1} \, b^{n-z-1} $$
                Variable ersetzen: $z=k$ 
                $$ = \sum_{k=0}^{n-1} a^{k+1} \, b^{n-1-k} - \sum_{k=-1}^{n-2} a^{k+1} \, b^{n-k-1} $$
                ersten und letzten Summanden aus den Summen ziehen:
                $$ = a^{n-1+1} \cdot b^{n-1-(n-1)} + \sum_{k=0}^{n-2} a^{k+1} \, b^{n-1-k} - \left( a^{-1+1} \cdot b^{n-1+1} +  \sum_{k=0}^{n-2} a^{k+1} \, b^{n-k-1}  \right) $$
                umsortieren und Minusklammer auflösen:
                $$ = a^n \cdot b^0 - a^0 \cdot b^n \, + \sum_{k=0}^{n-2} a^{k+1} \, b^{n-1-k}  \, -  \sum_{k=0}^{n-2} a^{k+1} \, b^{n-k-1}   $$
                Die Summen kürzen sich weg:
                $$ = b^0 \cdot a^n - a^0 \cdot b^n = 1 \cdot a^n - 1 \cdot b^n = a^n -b^n $$
                $\hfill \square$
                
        \end{enumerate}
    
    \section*{A 3.2}
        \begin{enumerate}[label = (\alph*)]
            \item 
                \begin{gather*}
                    \text{Für } n \in \mathbb{N} \text{ und } x > -1 \text{ gilt die Bernoullische Ungleichung:} \\
                    {(1+x)}^n \geq 1 + nx \\
                    \textbf{Induktionsanfang mit } n=0: \\
                    {1+x}^0 \geq 1 \geq 1 + 0x \\
                    \textbf{Induktionsschritt} \\
                    {(1+x)}^n \geq 1 + nx \Rightarrow^! {(1+x)}^{n+1} \geq 1 + (n+1)x \\
                    {(1+x)}^{n+1} \geq 1 + (n+1)x \\
                    \Leftrightarrow {(1+x)}^{n}\cdot(1+x) \geq 1 + nx + x \\
                    \text{Anwenden der Inuktionsvorraussetzung} \\
                    \Leftrightarrow {(1+x)}^{n}\cdot(1+x) \geq (1 + nx)\cdot(1 + x) \\
                    \text{Ausmultiplizieren} \\
                    = 1 + nx + x + nx^2 \geq 1 + nx + x  \: | \: -x - nx - 1\\
                    \Leftrightarrow nx^2 \geq 0 \\
                    \text{Ein Quadrat ist immer größer gleich Null} \\
                    \textbf{Da Induktionsanfang und Schritt gelten,} \\
                    \textbf{ist die Aussage für alle natürlichen Zahlen gültig}
                \end{gather*}

            \item
                \begin{align*}
                    q \in \mathbb{R}, q \neq 1; \: & \sum_{k=0}^{n} q^k = \frac{1-q^{n+1}}{1-q} \\
                    \textbf{Induktionsanfang mit } & n=0: \\
                    \sum_{k=0}^{0} q^k = q^0 = 1 & = \frac{1-q^{0+1}}{1-q} = \frac{1-q}{1-q} = 1 \\
                    \textbf{Induktionsschritt} & \\
                    \sum_{k=0}^{n} q^k & = \frac{1-q^{n+1}}{1-q} \Rightarrow^! \sum_{k=0}^{n+1} q^k = \frac{1-q^{n+2}}{1-q} \\
                    \sum_{k=0}^{n+1} q^k &= \frac{1-q^{n+2}}{1-q} \\
                    \Leftrightarrow q^{n+1} + \sum_{k=0}^{n} q^k &= \frac{1-q^{n+1} \cdot q}{1-q} \: | - q^{n+1} \\
                    \Leftrightarrow \sum_{k=0}^{n} q^k &= \frac{1-q^{n+1} \cdot q}{1-q} - q^{n+1} \: | \: \text{Letzten Summanden erweitern}\\
                    \Leftrightarrow \sum_{k=0}^{n} q^k &= \frac{1-q^{n+1} \cdot q}{1-q} - \frac{q^{n+1}\cdot(1-q)}{1-q} \\
                    \Leftrightarrow \sum_{k=0}^{n} q^k &= \frac{1-q^{n+1} \cdot q}{1-q} - \frac{q^{n+1}- q^{n+1} \cdot q}{1-q} \\
                    \Leftrightarrow \sum_{k=0}^{n} q^k &= \frac{1-q^{n+1} \cdot q - q^{n+1} + q^{n+1} \cdot q}{1-q} \\
                    \Leftrightarrow \sum_{k=0}^{n} q^k &= \frac{1 - q^{n+1}}{1-q} \\
                    &\textbf{Da Induktionsanfang und Schritt gelten,} \\
                    &\textbf{ist die Aussage für alle natürlichen Zahlen gültig}
                \end{align*}
                
            \item
                \begin{gather*}
                    n \in \mathbb{N} \backslash \{0\} \: n! \leq 4 \cdot { \left( \frac{n}{2} \right) }^{n+1} \\
                    \textbf{Induktionsanfang mit } n=1: \\
                    1! = 1 = 4 \cdot {\left( \frac{1}{2} \right) }^{1+1} = 4 \cdot {\left( \frac{1}{2} \right)}^{2} = 1 \\
                    \textbf{Induktionsschritt} \\
                    n! \leq 4 \cdot {\left( \frac{n}{2} \right) }^{n+1} \Rightarrow^! (n+1)! \leq 4 \cdot {\left( \frac{n+1}{2} \right) }^{n+2} \\
                    (n+1)! = n! \cdot (n+1) \\
                    \text{Verwenden der Induktionsannahme} \\
                    \Rightarrow n! \cdot (n+1) \leq 4 \cdot {\left( \frac{n}{2} \right) }^{n+1} \cdot (n + 1) \\
                    \\
                    \text{Als Hilfsterm: } {\left( 1 + \frac{1}{n} \right)}^{n+1} \geq 2 \: \: \forall \: n \in \mathbb{N} \\
                    \text{Da gilt: }
                    {\left(1 + \frac{1}{n} \right) }^{n+1} \geq^\text{Bernoulli Ungleichung} 1 + \frac{n+1}{n}\\
                    \frac{n+1}{n} \text{ ist immer größer gleich 1 für } n \in \mathbb{N} \text{. Dieser Wert +1 ist also größer gleich 2} \\
                    \\
                    \Rightarrow \frac{{(1 + \frac{1}{n})}^{n+1}}{2} \geq 1 \\
                    \text{Abschätzen nach oben mit Faktor größer gleich 1} \\
                    4 \cdot {\left( \frac{n}{2} \right) }^{n+1} \cdot (n + 1) \leq  4 \cdot {\left( \frac{n}{2} \right) }^{n+1} \cdot (n + 1) \cdot \frac{{(1 + \frac{1}{n})}^{n+1}}{2} \\
                    \text{Vereinfachen} \\
                    = 4 \cdot {\left( \frac{n}{2} \right) }^{n+1} \cdot {(1 + \frac{1}{n})}^{n+1} \cdot \frac{(n + 1)}{2} \\
                    \text{Vereinfachen} \\
                    = 4 \cdot {\left( \frac{n}{2} \cdot {(1 + \frac{1}{n})} \right)}^{n+1} \cdot \frac{(n + 1)}{2} \\
                    \text{Ausmultiplizieren} \\
                    = 4 \cdot {\left( \frac{n}{2} + \frac{1}{2} \right)}^{n+1} \cdot \frac{(n + 1)}{2} \\
                    \text{Vereinfachen} \\
                    = 4 \cdot {\left( \frac{n+1}{2} \right)}^{n+1} \cdot \frac{(n + 1)}{2} \\
                    \text{Faktor einklammern} \\
                    = 4 \cdot {\left( \frac{n+1}{2} \right) }^{n+2} \\
                    \text{Damit gilt der Induktionsschritt} \\
                    \textbf{Da Induktionsanfang und Schritt gelten,} \\
                    \textbf{ist die Aussage für alle natürlichen Zahlen gültig}
                \end{gather*}
        \end{enumerate}

    \section*{A 3.3}
        \begin{enumerate}[label = (\alph*)]
            \item 
                \begin{gather*}
                    M_1 = \{1- \frac{1}{n}: n \in \mathbb{N} \} \\
                    \text{Zu zeigen: 1 ist eine obere Schranke von } M_1 \\
                    1 - \frac{1}{n} \leq 1 \: \forall n \in \mathbb{N} \text{ Es folgt } 0 \leq \frac{1}{n} \\
                    \Rightarrow \text{1 ist eine obere Schranke von } M_1
                \end{gather*} 
                \begin{gather*}
                    \text{Zu zeigen: 1 ist die kleinste obere Schranke von } M_1 \\
                    \text{Angenommen 1 wäre nicht die kleinste obere Schranke,} \\
                    \text{so gäbe es eine kleinere im Abstand } \epsilon \\
                    \text{Sei } \epsilon > 0 \in \mathbb{R} \\
                    1 - \frac{1}{n} \leq 1 - \epsilon \Rightarrow \epsilon \leq \frac{1}{n} \: \forall n \in \mathbb{N} \\
                    \text{Nach Lemma 2.18 gilt aber } \epsilon \leq 0 \text{ Das ist aber ein Widerspruch, da } \epsilon > 0 \\
                    \Rightarrow \text{Somit war die Annahme falsch, es gibt keine kleinere obere Schranke.} \\
                    \text{1 ist das Supremum von } M_1
                \end{gather*}
                \begin{gather*}
                    \text{Zu zeigen: 1 ist Maximum von } M_1 \\
                    1-\frac{1}{n} = 1 \Rightarrow \frac{1}{n} = 0 \Rightarrow 1 = 0 \\
                    \text{Die Annahme war falsch } M_1 \text{ hat kein Maximum}
                \end{gather*}
                \begin{gather*}
                    \text{Zu zeigen: 0 ist eine untere Schranke von } M_1 \\
                    0 \leq 1 - \frac{1}{n} \Rightarrow \frac{1}{n} \leq 1 \Rightarrow 1 \leq n \\
                    \text{Alle natürliche Zahlen sind größer gleich 1} \\
                    \text{0 ist eine untere Schranke von } M_1
                \end{gather*}
                \begin{gather*}
                    \text{Zu zeigen: 0 ist die größte untere Schranke von } M_1 \\
                    \text{Angenommen 0 wäre nicht die größte untere Schranke,} \\
                    \text{so gäbe es eine größere im Abstand} \epsilon \\
                    \text{Sei } \epsilon > 0 \in \mathbb{R} \\
                    0 + \epsilon \leq 1 - \frac{1}{n} \Rightarrow \epsilon + \frac{1}{n} \leq 1 \\
                    \text{Sei } n=1 \\
                    \Rightarrow \epsilon + 1 = 1 \Rightarrow \epsilon = 0 \\
                    \text{Da } \epsilon \text{ größer als Null sein muss ist dies ein Widerspruch} \\
                    \Rightarrow \text{Somit war die Annahme falsch, es gibt keine größere untere Schranke.} \\
                    \text{0 ist das Infimum von } M_1
                \end{gather*}
                \begin{gather*}
                    \text{Zu zeigen: 0 ist Minimum von } M_1 \\
                    0 = 1 - \frac{1}{n} \Rightarrow \frac{1}{n} = 1 \Rightarrow n = 1 \\
                    \text{0 ist Minimum von } M_1
                \end{gather*}
                Die Menge \(M_1\) ist nach oben beschränkt und hat in 1 ein Supremum, jedoch kein Maximum.
                Die Menge ist außerdem nach unten beschränkt und hat in 0 ein Infimum und Minimum
            \item 
                \[
                     M_2 = \{t \in \mathbb{R}, \text{t ist obere Schranke von } M_1\}
                \]
                In (a) haben wir das Supremum von \(M_1\) als 1 festgestellt. Jede weitere obere Schranke
                muss also größer sein als 1. Die Menge lässt sich umschreiben als: \\
                \[
                    M_2 = \{x \geq 1, \: x \in \mathbb{R}\}    
                \]
                Und das wiederum als: \\
                \[[1,\inf)\] \\
                Die Menge \(M_2\) ist also nach Definition nach unten beschränkt mit dem Supremum von \(M_1\),
                1 ist damit Infimum und da es in der Menge liegt auch Minimum von \(M_2\).
                \(M_2\) ist aber nicht nach oben beschränkt, da eine obere Schranke beliebig viel größer sein kann
                als das Supremum. Oder trivial abzulesen aus der Intervallschreibweise.

                \item 
                \begin{gather*}
                    M_3 = \{{\left(1- \frac{1}{n^2}\right)}^n: n \in \mathbb{N} \} \\
                    \text{Zu zeigen: 1 ist eine obere Schranke von } M_3 \\
                    {\left(1- \frac{1}{n^2}\right)}^n \leq 1 \: \forall n \in \mathbb{N} \\
                    \Rightarrow 1- \frac{1}{n^2} \leq 1 \\
                    \Rightarrow 0 \leq \frac{1}{n^2} \Rightarrow 0 < 1 \\
                    \Rightarrow \text{1 ist eine obere Schranke von } M_3
                \end{gather*} 
                \begin{gather*}
                    \text{Zu zeigen: 1 ist die kleinste obere Schranke von } M_3 \\
                    \text{Angenommen 1 wäre nicht die kleinste obere Schranke,} \\
                    \text{so gäbe es eine kleinere im Abstand} \epsilon \\
                    \text{Sei } \epsilon > 0 \in \mathbb{R} \\
                    {\left(1- \frac{1}{n^2}\right)}^n \leq 1 - \epsilon \\
                    \text{Anwenden der Bernoulli-Ungleichung} \\
                    1 - \epsilon \geq {\left(1- \frac{1}{n^2}\right)}^n \geq 1 - n \cdot \frac{1}{n^2} = 1 - \frac{1}{n} \\
                    \Rightarrow 1- \epsilon \geq 1 - \frac{1}{n} \\
                    \Rightarrow \frac{1}{n} \geq \epsilon \: \: \forall \: n \in \mathbb{N}\\
                    \text{Nach Lemma 2.18 gilt aber } \epsilon \leq 0 \text{ Das ist aber ein Widerspruch, da } \epsilon > 0 \\
                    \Rightarrow \text{Somit war die Annahme falsch, es gibt keine kleinere obere Schranke.} \\
                    \text{1 ist das Supremum von } M_3
                \end{gather*}
                \begin{gather*}
                    \text{Zu zeigen: 1 ist Maximum von } M_3 \\
                    {\left(1- \frac{1}{n^2}\right)}^n = 1 \Rightarrow 1- \frac{1}{n^2} = 1 \\
                    \Rightarrow 0 = \frac{1}{n^2} \Rightarrow n^2 = 0 \Rightarrow n=0 \\
                    \text{n kann nicht null sein da in der Definition von } M_3 \text{ durch n geteilt wird.} \\
                    \text{Die Annahme war falsch } M_3 \text{ hat kein Maximum}
                \end{gather*}
                \begin{gather*}
                    \text{Zu zeigen: 0 ist eine untere Schranke von } M_3 \\
                    0 \leq {\left(1- \frac{1}{n^2}\right)}^n \Rightarrow 0 \leq 1- \frac{1}{n^2} \\
                    \frac{1}{n^2} \leq 1 \Rightarrow 1 \leq n^2 \Rightarrow 1 \leq n \\
                    \text{Alle natürliche Zahlen sind größer gleich 1} \\
                    \text{0 ist eine untere Schranke von } M_1
                \end{gather*}
                \begin{gather*}
                    \text{Zu zeigen: 0 ist die größte untere Schranke von } M_3 \\
                    \text{Angenommen 0 wäre nicht die größte untere Schranke,} \\
                    \text{so gäbe es eine größere im Abstand} \epsilon \\
                    \text{Sei } \epsilon > 0 \in \mathbb{R} \\
                    0 + \epsilon \leq {\left(1- \frac{1}{n^2}\right)}^n
                    \text{Sei } n=1 \\
                    \Rightarrow 0 + \epsilon \leq {\left(1- \frac{1}{1^2}\right)}^1 \\
                    \Rightarrow \epsilon \leq 0^1 \Rightarrow \epsilon \leq 0 \\
                    \text{Da } \epsilon \text{ größer als Null sein muss ist dies ein Widerspruch} \\
                    \Rightarrow \text{Somit war die Annahme falsch, es gibt keine größere untere Schranke.} \\
                    \text{0 ist das Infimum von } M_3
                \end{gather*}
                \begin{gather*}
                    \text{Zu zeigen: 0 ist Minimum von } M_3 \\
                    0 =  {\left(1- \frac{1}{n^2}\right)}^n \Rightarrow 0 = 1- \frac{1}{n^2} \\
                    \Rightarrow \frac{1}{n^2} = 1 \Rightarrow 1 = n^2 \Rightarrow n = 1
                    \text{0 ist Minimum von } M_3
                \end{gather*}
            Die Menge \(M_3\) ist nach oben beschränkt und hat in 1 ein Supremum, jedoch kein Maximum.
            Die Menge ist außerdem nach unten beschränkt und hat in 0 ein Infimum und Minimum
        \end{enumerate}

        \section*{A 3.4}
                Die Türme von Hanoi. Die drei Türme sind \(T_1, T_2, T_3\). Zu Beginn sind auf Turm \(T_1\) \(n \in \mathbb{N} \) Scheiben und auf \(T_2\) und \(T_3\) 0 Scheiben.
                Die Scheiben sind von der kleinsten bis zur größten (n.ten) Scheiben nummeriert. \\
                Für 1 Scheibe ist die Lösung trivial. Es braucht nur genau einen Schritt (s). \\
                Für 2 Scheiben bewegt man die erste Scheibe auf Turm \(T_2\), die zweite auf \(T_3\) und dann die erste hinterher. Also ist \(s=3\)\\

                Nter Schritt \\
                Zu zeigen ist: Kann man einen Stapel von \(n\) Scheiben bewegen, so folgt daraus, dass man einen Stapel von \(n+1\) Scheiben bewegen kann.
                Dazu bewege man die obersten \(n\) Scheiben des ersten Turmes auf einen zweiten Turm, dann die \(n+1\)te Scheibe auf einen dritten Turm.
                Dann bewegt man die \(n\) Scheiben vom zweiten Turm auf den dritten. \\
                Somit hat man einen Stapel mit \(n+1\) Scheiben bewegt. \\

                Die oben gegebene Lösung folgt dem Prinzip der vollständigen Induktion. Der Induktionsanfang mit \(n=1\) ist angegeben.
                In der allgemeinen Lösung ist gezeigt, wie, wenn man einen Stapel mit \(n\) Scheiben hat, man einen Turm mit \(n+1\) Scheiben bewegt.
                Somit lässt sich ein belieg großer Stapel von einem auf einen anderen Turm bewegen. \\

                Beim eigentlichen Ausführen sind jedoch noch einige Dinge zu berücksichtigen. \\
                Beim bewegen von \(m\) großen Teilstapeln darf die \(m+1\) Scheibe nicht direkt mit der 1sten Scheibe bedeckt werden.

                Die minimale Zugzahl lässt sich genauso rekursiv ermitteln. Um die unterste Scheibe eines \(n\) Scheiben hohen Stapels zu bewegen,
                muss man alle darüber liegenden Scheiben auf einen anderen Turm legen. Angenommen man tut dies mit der minimalen Zuganzahl für
                \(n-1\) Scheiben \(s_{n-1}\). Erst dann lässt sich die unterste Scheibe auf einen anderen Turm bewegen, wofür man genau einen Zug braucht.
                Um die restlichen Scheiben nun wieder darauf zu legen, braucht man wieder \(s_{n-1}\) Züge.
                Für einen Stapel von \(n\) Scheiben gilt also: \(s_n = s_{n-1} + 1 + s_{n-1}\). Führt man dies fort, so kommt man zu \(n=1\) für das man
                nur genau einen Schritt braucht. Somit gilt mit vollständiger Induktion diese minimale Anzahl an Zügen.
                Diese rekursive Formel lässt sich von \(s_n = s_{n-1} + 1 + s_{n-1}\) umschreiben zu \(s_n = 2^{n} - 1\)

\end{document}