\documentclass{article}

\usepackage[a4paper, top=2cm, bottom=3cm]{geometry}
\usepackage{babel}[german]
\usepackage{booktabs}
\usepackage{mathtools}
\usepackage{amssymb}
\usepackage{enumitem}
\usepackage{amsmath}

\date{26.11.2021}
\title{Aufgabenblatt 7, Mathematik für Physiker 1}
\author{Florian Adamczyk, Finn Wagner}

\begin{document}
    \maketitle

    \section*{A 7.1}
    Bestimmen Sie alle \(x \in \mathbb{R}\), für die die Reihen:
    \[
        \sum_{n=0}^{\infty} \frac{1}{2^{3n-4}} {(x+1)}^{n+2} \text{ und } \sum_{n=0}^{\infty} (3n^3 + 4n^2 -4)x^n
    \]
    konvergiert und berechnen Sie für die erste Reihe auch eine geschlossene Formel in Abhängigkeit von \(x\). \\
    Der Konvergenzradius \(R\) einer Potenzreihe ist \(R = \frac{1}{ \limsup \sqrt[n]{|a_n|} } \). \\
    \begin{enumerate}[ label= (\alph*) ]
        \item {\large \( \sum_{n=0}^{\infty} \frac{1}{2^{3n-4}} {(x+1)}^{n+2} \) } \\
        Die Summe lässt sich umformen:
        \begin{gather*}
            \sum_{n=0}^{\infty} \frac{1}{2^{3n-4}} {(x+1)}^{n+2} = \sum_{n=0}^{\infty} \frac{1}{2^{3n} \cdot 2^{-4}} {(x+1)}^{n} \cdot {(x+1)}^{2} \\
            = \sum_{n=0}^{\infty} 16 \frac{1}{2^{3n}} {(x+1)}^{n} \cdot {(x+1)}^{2} = 16 {(x+1)}^{2} \sum_{n=0}^{\infty} \frac{ {(x+1)}^{n} }{2^{3n}} \\
            = 16 {(x+1)}^{2} \sum_{n=0}^{\infty} {\left( \frac{ {x+1} }{2^3} \right)}^n = 16 {(x+1)}^{2} \sum_{n=0}^{\infty} {\left( \frac{ {x+1} }{8} \right)}^n
        \end{gather*}
        Wir ersetzen nun \( y:= x + 1 \) ein, damit ergibt sich:
        \[ 16 {(x+1)}^{2} \sum_{n=0}^{\infty} {\left( \frac{ {x+1} }{8} \right)}^n = 16y^2 \sum_{k=0}^{\infty} { \left( \frac{y}{8} \right) }^n \]
        Damit ist \( a_n = { \left( \frac{1}{8} \right) }^n \). Die Folgenglieder der Reihe \((a_n)\) sind immer positiv, als positive Potenz von \(\frac{1}{8}\). \\
        Wir berechnen nun den Konvergenzradius der Reihe:
        \begin{gather*}
            R = \frac{1}{ \limsup \sqrt[n]{|a_n|} } = \frac{1}{ \limsup \sqrt[n]{ \left| { \left( \frac{1}{8} \right) }^n \right| } } \\
            \frac{1}{ \limsup \sqrt[n]{ { \left( \frac{1}{8} \right) }^n } } = \frac{1}{ \limsup \left( \frac{1}{8} \right) } \\
            = \frac{1}{ \left( \frac{1}{8} \right) } = 8
        \end{gather*}

        Der Konvergenzradius der Reihe \(\sum_{k=0}^{\infty} { \left( \frac{y}{8} \right) }^n\) ist also 8. Damit gilt: \(y \in (-8, 8) \).
        Setzten wir \(x\) wieder zurück ein erhalten wir: \(x \in (-9, 7) \). \\

        Da wir die Reihe bereits umgeformt haben, erkennen wir mit \(q:= \left( \frac { x+1 }{8} \right) \) an der Form:
        \[ 16 {(x+1)}^{2} \sum_{n=0}^{\infty} {\left( \frac {x+1}{8} \right)}^n = 16 {(x+1)}^{2} \sum_{n=0}^{\infty} {q}^n \]
        die geometrische Reihe mit der Variablen \(q\). Die Bedingung, dass die geometrische Reihe konvergiert ist,
        das \( |q| < 1 \). Hier \( |\frac{(x+1)}{8}| < 1 \).
        Setzen wir \(-9\) ein erhalten wir \(\frac{-9 + 1}{8} = -1\), für \(7\), \(\frac{7 + 1}{8} = 1\). 
        Da die geometrische Reihe nicht für \(-1\) und \(1\) konvergiert, konvergiert die Reihe \(\sum_{n=0}^{\infty} \frac{1}{2^{3n-4}} {(x+1)}^{n+2}\)
        für alle \(x \in (-9, 7)\).

        Von der umgeformten Reihe können wir auch den Grenzwert berechnen, sie konvergiert zu:
        \[
            16 {(x+1)}^{2} \sum_{n=0}^{\infty} {q}^n = 16 {(x+1)}^{2} \frac{1}{1 - q}
        \]
        Also:
        \[
            16 {(x+1)}^{2} \frac{1}{1 - \left( \frac{ {x+1} }{8} \right)} = 16 {(x+1)}^{2} \frac{1}{ \left( \frac{1-x}{8} \right) } \\
            = \frac{128 {(x+1)}^{2}}{7-x}
        \]
        Die Reihe lässt sich also als geschlossene Formel wie folgt angeben:
        \[
             f: (-9,7) \to \mathbb{R}; \ x \to \frac{128 {(x+1)}^{2}}{7-x}
        \]

        \item {\large \( \sum_{n=0}^{\infty} (3n^3 + 4n^2 -4)x^n \) } \\
        \(a_n\) ist hier \( (3n^3 + 4n^2 -4) \). \\
        Die Reihe \((a_n)\) ist immer positiv. Der kubische Term von \((a_n)\) ist nie negativ, da \(n \in \mathbb{N}\).
        Weiterhin ist der quadratische Term immer positiv und größer als 4.
        Da der quadratische Term wächst, ist \(4n^2 -4 \geq 0\). \\
        Wir berechnen den Konvergenzradius der Reihe:
        \begin{gather*}
            R = \frac{1}{ \limsup \sqrt[n]{|a_n|} } = \frac{1}{ \limsup \sqrt[n]{|3n^3 + 4n^2 -4|} } \\
            = \frac{1}{ \limsup \sqrt[n]{3n^3 + 4n^2 -4} } = \frac{1}{ \limsup \sqrt[n]{(n^3)} \sqrt[n]{3 + \frac{4}{n^2} - \frac{4}{n^3}} } \\
            = \frac{1}{ \limsup {\left( \sqrt[n]{n} \right)}^3 \sqrt[n]{3 + \frac{4}{n^2} - \frac{4}{n^3}} } \\
        \end{gather*}
        Nach Satz 2.27 können wir die beiden Grenzwerte seperat betrachten und dann zusammen multiplizieren: \\
        Zuerst \({\sqrt[n]{n}}^3\). Dies können wir umschreiben als
        \[ {\sqrt[n]{n}}^3 = \sqrt[n]{n} \cdot \sqrt[n]{n} \cdot \sqrt[n]{n} \]
        Von Aufgabenblatt 5 wissen wir das \(\sqrt[n]{n}\) gegen \(1\) konvergiert. Somit konvergiert auch \({\sqrt[n]{n}}^3\) gegen \(1\) \\

        Betrachten wir nun \(\sqrt[n]{3 + \frac{4}{n^2} - \frac{4}{n^3}}\) \\
        Den Term \(3 + \frac{4}{n^2} - \frac{4}{n^3}\), können wir nach oben für \(n > 4\) gegen \(n\)
        abschätzen. Weiterhin können wir \(\frac{4}{n}\) nach unten gegen \(\frac{4}{n^3}\) abschätzen, da wir den Nenner vergrößern.
        Es ergibt sich:
        \[
            3 = 3 + \frac{4}{n^3} - \frac{4}{n^3} \leq 3 + \frac{4}{n} - \frac{4}{n^3} \leq n  
        \]
        Ziehen wir die n.te Wurzel folgt:
        \[
            \sqrt[n]{3} \leq \sqrt[n]{3 + \frac{4}{n} - \frac{4}{n^3}} \leq \sqrt[n]{n}
        \]
        Wir wenden nun das Sandwichkriterium an. Nach Aufgabenblatt 5 wissen wir das \(\sqrt[n]{a}\) mit \(a \in \mathbb{R}\) gegen 1 konvergiert.
        Ebenso konvergiert \(\sqrt[n]{n}\) gegen 1. Es ergibt sich:
        \[
            1 \leq \sqrt[n]{3 + \frac{4}{n} - \frac{4}{n^3}} \leq 1
        \]
        \(\sqrt[n]{3 + \frac{4}{n} - \frac{4}{n^3}}\) konvergiert also gegen 1. \\
        Nun zurück zum Konvergenzradius:
        \[
            \frac{1}{ \limsup {\sqrt[n]{n}}^3 \sqrt[n]{3 + \frac{4}{n^2} - \frac{4}{n^3}} } = \frac{1}{1 \cdot 1} = 1
        \]
        Der Konvergenzradius der Reihe ist damit \(R = 1\). \\
        Setzen wir \((1)\) ein, erhalten wir
        \[ \sum_{n=0}^{\infty} (3n^3 + 4n^2 -4) \cdot 1^n  = \sum_{n=0}^{\infty} (3n^3 + 4n^2 -4) \]
        Die Folge \((3n^3 + 4n^2 -4)\) divergiert offensichtlich gegen \( \infty \). Die Reihe divergiert nach Satz 2.34 (c) nicht, da ihre Folge keine Nullfolge ist. \\
        Setzen wir \(-1\) ein, erhalten wir
        \[ \sum_{n=0}^{\infty} (3n^3 + 4n^2 -4) \cdot {-1}^n \]
        Diese Reihe divergiert ebenfalls, weil \(3n^3 + 4n^2 -4\) gegen unendlich divergiert 
        und die Folgenglieder wegen des Terms \({(-1)}^n\),
        für alle geraden \(n\) gegen \( \infty \) und für alle ungeraden gegen \( -\infty \) divergieren. \\
        Die Reihe divergiert nach Satz 2.34 (c) ebenfalls nicht, da ihre Folge auch keine Nullfolge ist. \\
        Die Reihe \( \sum_{n=0}^{\infty} (3n^3 + 4n^2 -4)x^n \) konvergiert demnach für alle \(x\) im Intervall \(x \in (-1, 1) \)
    \end{enumerate}

    \section*{A 7.2}
    Es sei \(f: \mathbb{R} \to \mathbb{R}, f(x) := x^2\) und \(x_0 \in \mathbb{R}\).
    Finden Sie zu gegebenem \(\varepsilon > 0\) ein \(\delta = \delta(\varepsilon, x_0) > 0\), sodass
    \[ |x-x_0| < \delta \Rightarrow |f(x) - f(x_0)| < \varepsilon \]
    Sei \(\varepsilon > 0\) und \(\delta := -|x_0| + \sqrt{ {|x_0|}^2 + \varepsilon } \) \\
    Zu zeigen \(\delta > 0\): \\
    \[ -|x_0| + \sqrt{ {|x_0|}^2 + \varepsilon} > 0 \Leftrightarrow \sqrt{ {|x_0|}^2 + \varepsilon} > |x_0| \Leftrightarrow^{\text{Quadrieren}}  {|x_0|}^2 + \varepsilon > {|x_0|}^2 \]
    Was eine wahre Aussage ist, da \(\varepsilon > 0\) \\
    Nun zur eigentlichen Stetigkeit:
    \begin{gather*}
        |f(x) - f(x_0)| = |x^2 - x_0^2| =^{\text{Dritte Binomische Formel}} \\
        |( x + x_0 ) ( x - x_0 )| = |x + x_0| |x - x_0|
    \end{gather*}
    Da \( |x-x_0| < \delta \) schätzen wir nach oben ab:
    \begin{gather*}
        |x + x_0| |x - x_0| < \delta | x + x_0 |
    \end{gather*}
    Wir addieren im Betrag 0 dazu, in der Form \( +(x_0 - x_0) \)
    \begin{gather*}
        \delta | x + x_0 | = \delta | x + x_0 + x_0 - x_0| = \delta | (x - x_0) + (2 x_0)| \\
        = \delta (|x - x_0| + |2 x_0|) = \delta |x - x_0| + \delta |2 x_0| \\
        < \delta^2 + \delta |2x_0|
    \end{gather*}
    Nun setzen wir \( \delta \) ein und vereinfachen:
    \begin{gather*}
        \delta^2 + \delta |2x_0| = {\left( -|x_0| + \sqrt{ {|x_0|}^2 + \varepsilon } \right)}^2 + \left( -|x_0| + \sqrt{ {|x_0|}^2 + \varepsilon } \right) |2x_0| \\
        = {(-|x_0|)}^2 -2|x_0|\sqrt{ {|x_0|}^2 + \varepsilon } + {\left( \sqrt{ {|x_0|}^2 + \varepsilon } \right)}^2 -|x_0||2x_0| + |2x_0|\sqrt{ {|x_0|}^2 + \varepsilon } \\
        = {|x_0|}^2 -2|x_0|\sqrt{ {|x_0|}^2 + \varepsilon } + ({|x_0|}^2 + \varepsilon) -|x_0||2x_0| + 2|x_0|\sqrt{ {|x_0|}^2 + \varepsilon } \\
        = {|x_0|}^2 + ({|x_0|}^2 + \varepsilon) -|x_0||2x_0| \\
        = {|x_0|}^2 + {|x_0|}^2 -|x_0||2x_0| + \varepsilon \\
        = 2 {|x_0|}^2 - 2 {|x_0|}^2 + \varepsilon \\
        = \varepsilon
    \end{gather*}
    Es gilt also insgesamt:
    \[ |f(x) - f(x_0)| < \varepsilon \]
    Was zu beweisen war. Die Funktion \(x^2\) ist damit für alle \(x \in \mathbb{R}\) stetig.

    \section*{A 7.4}
    \begin{enumerate}[label = (\alph*) ]
        \item Zeigen Sie mithilfe der Potenzreihendarstellung, dass
        \[ e^{ix} = \cos(x) + i \ \sin(x), \ \forall \ x \in \mathbb{R} \]
        Die Potenzreihe der e-Funktion ist
        \[e^{x} = \sum_{k=0}^{\infty} \frac{x^k}{k!}\]
        Die Potenzreihen für Sinus und Kosinus
        \begin{gather*}
            \sin(x) = \sum_{k=0}^{\infty} {(-1)}^k \frac{x^{2k+1}}{(2k+1)!} \\
            \cos(x) = \sum_{k=0}^{\infty} {(-1)}^k \frac{x^{2k}}{(2k)!}
        \end{gather*}
        Gegeben ist die komplexe e-Funktion \(e^{ix}\). Ihre Potenzreihendarstellung ist \(\sum_{k=0}^{\infty} \frac{{(ix)}^k}{k!}\).
        Aus der Vorlesung ist bekannt, das die komplexe e-Funktion absolut konvergiert. Damit können wir den Riemannschen Umordnungssatz (Satz 2.42) anwenden.
        Wir fassen nun immer zwei aufeinanderfolgende Terme der Potenzreihe zu einem zusammen, einen Term mit geradem \(k\) und einen Term mit ungeradem \(k\). \\
        Wir beginnen mit \( k=0 \) und \( k=1 \). Weil wir bei der umgeordneten Reihe eine fortlaufende Indexvariable haben möchten,
        ändern wir das \(k\) in allen Termen. Da immer 2 Terme zusammengefasst werden müssen wir aus einem Wert der Indexvariable, zwei neue k berechnen.
        Wir ersetzen für Terme mit geradem \(k\), \(k \to 2k\) und für Terme mit ungeradem \(k\), \(k \to 2k + 1\).
        Für den Wert 0 der neuen Indexvariable ergeben sich also 0 und 1, für den Wert 1: 2 und 3, und so weiter.
        Die umgeordnete Reihe sieht wie folgt aus: 
        \[ \sum_{k=0}^{\infty} \frac{{(ix)}^k}{k!} = \sum_{k=0}^{\infty} \frac{{(ix)}^{2k}}{(2k)!} + \frac{{(ix)}^{2k+1}}{(2k+1)!} \]
        Wir formen diese neue Darstellung nun um:
        \begin{gather*}
            \sum_{k=0}^{\infty} \frac{{(ix)}^{2k}}{(2k)!} + \frac{{(ix)}^{2k+1}}{(2k+1)!} = \sum_{k=0}^{\infty} \frac{ {i}^{2k} {x}^{2k}} {(2k)!} + \frac{ {i}^{2k+1} {x}^{2k+1} }{(2k+1)!} \\
            \sum_{k=0}^{\infty} {(i^2)}^{k} \frac{ {x}^{2k}} {(2k)!} + i{(i^2)}^{k} \frac{ {x}^{2k+1} }{(2k+1)!}
        \end{gather*}
        Da \(i^2 = -1\) gilt, können wir diese Reihe weiter vereinfachen:
        \begin{gather*}
            \sum_{k=0}^{\infty} {(i^2)}^{k} \frac{ {x}^{2k}} {(2k)!} + i{(i^2)}^{k} \frac{ {x}^{2k+1} }{(2k+1)!} = \sum_{k=0}^{\infty} {(-1)}^{k} \frac{ {x}^{2k}} {(2k)!} + i{(-1)}^{k} \frac{ {x}^{2k+1} }{(2k+1)!}
        \end{gather*}
        Wir ziehen die beiden Terme nun in ihre eigenen Summen nach Satz 2.34 (b)
        \begin{gather*}
            \sum_{k=0}^{\infty} {(-1)}^{k} \frac{ {x}^{2k}} {(2k)!} + i{(-1)}^{k} \frac{ {x}^{2k+1} }{(2k+1)!} = \sum_{k=0}^{\infty} {(-1)}^{k} \frac{ {x}^{2k}} {(2k)!} + \sum_{k=0}^{\infty} i{(-1)}^{k} \frac{ {x}^{2k+1} }{(2k+1)!} \\
            = \sum_{k=0}^{\infty} {(-1)}^{k} \frac{ {x}^{2k}} {(2k)!} + i\sum_{k=0}^{\infty} {(-1)}^{k} \frac{ {x}^{2k+1} }{(2k+1)!}
        \end{gather*}
        Wir setzen nun die oben gegebenen Potenzreihendarstellungen von Sinus und Kosinus ein:
        \begin{gather*}
            \sum_{k=0}^{\infty} {(-1)}^{k} \frac{ {x}^{2k}} {(2k)!} + i\sum_{k=0}^{\infty} {(-1)}^{k} \frac{ {x}^{2k+1} }{(2k+1)!} = \cos(x) + i \ \sin(x)
        \end{gather*}
        Damit gilt \(e^{ix} = \cos(x) + i \ \sin(x)\)

        \item Folgern Sie, dass für \(x, y \in \mathbb{R}\) die beiden Formeln
        \begin{align*}
            \sin(x+y) &= \sin(x)\cos(y) + \sin(y)\cos(x) \text{ und } \\
            \cos(x+y) &= \cos(x)\cos(y) + \sin(x)\sin(y)
        \end{align*}
        gelten. \\
        Es gilt:
        \begin{align*}
            e^{i(x+y)} &= \cos(x+y) + i \sin(x+y) \ | \ -\cos(x+y) \ :i \\
            \Leftrightarrow \sin(x+y) &= \frac{ e^{i(x+y)} - \cos(x+y) }{i}
        \end{align*}
        Wir formen nun um:
        \begin{align*}
            \sin(x+y) &= \frac{ e^{i(x+y)} - \cos(x+y) }{i} = \frac{ e^{ix + iy} - \cos(x+y) }{i} = \frac{ e^{ix} e^{iy} - \cos(x+y) }{i} \\
            &e^{ix} = \cos(x) + i \sin(x) \text{ einsetzen} \\
            &= \frac{ (\cos(x) + i \sin(x)) (\cos(y) + i \sin(y)) - \cos(x+y) }{i} \\
            &\text{Ausmultiplizieren} \\
            &= \frac{ (\cos(x)\cos(y) + i\cos(x)\sin(y) + i\sin(x)\cos(y) + i^2\sin(x)\sin(y)) - \cos(x+y) }{i} \\
            &\text{Auf Brüche aufteilen} \\
            &= \frac{\cos(x)\cos(y)}{i} + \frac{i\cos(x)\sin(y)}{i} + \frac{i\sin(x)\cos(y)}{i} + \frac{i^2\sin(x)\sin(y)}{i} - \frac{\cos(x+y)}{i} \\
            &\text{Kürzen} \\
            &= \frac{\cos(x)\cos(y)}{i} + \cos(x)\sin(y) + \sin(x)\cos(y) + i\sin(x)\sin(y) - \frac{\cos(x+y)}{i} \\
        \end{align*}
        Wir nehmen nun den Realteil:
        \begin{align*}
            Re(\sin(x+y)) &= Re(\frac{\cos(x)\cos(y)}{i} + \cos(x)\sin(y) + \sin(x)\cos(y) + i\sin(x)\sin(y) - \frac{\cos(x+y)}{i}) \\
            \Rightarrow \sin(x+y) &= \cos(x)\sin(y) + \sin(x)\cos(y)
        \end{align*}
        Nun für die zweite Gleichung:
        \begin{align*}
            e^{i(x+y)} &= \cos(x+y) + i \sin(x+y) \ | \ -\sin(x+y) \\
            \Leftrightarrow \cos(x+y) &= e^{i(x+y)} - i\sin(x+y)
        \end{align*}
        Wir formen nun um:
        \begin{align*}
            \cos(x+y) &= e^{i(x+y)} - i\sin(x+y) = e^{ix+iy} - i\sin(x+y) = e^{ix} e^{iy} - i\sin(x+y) \\
            &e^{ix} = \cos(x) + i \sin(x) \text{ einsetzen} \\
            &= (\cos(x) + i \sin(x))(\cos(y) + i \sin(y)) - i\sin(x+y) \\
            &\text{Ausmultiplizieren} \\
            &= \cos(x)\cos(y) + i\cos(x)\sin(y) + i\sin(x)\cos(y) + i^2 \sin(x)\sin(y) - i\sin(x+y) \\
            &\text{Vereinfachen} \\
            &= \cos(x)\cos(y) + i\cos(x)\sin(y) + i\sin(x)\cos(y) - \sin(x)\sin(y) - i\sin(x+y)
        \end{align*}
        Wir nehmen nun den Realteil:
        \begin{align*}
            Re(\cos(x+y)) &= Re(\cos(x)\cos(y) + i\cos(x)\sin(y) + i\sin(x)\cos(y) - \sin(x)\sin(y) - i\sin(x+y)) \\
            \Rightarrow \cos(x+y) &= \cos(x)\cos(y) - \sin(x)\sin(y)
        \end{align*}
    \end{enumerate}
\end{document}