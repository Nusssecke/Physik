\documentclass{article}

\usepackage[a4paper, top=2cm, bottom=3cm]{geometry}
\usepackage{babel}[german]
\usepackage{booktabs}
\usepackage{mathtools}
\usepackage{amssymb}
\usepackage{enumitem}
\usepackage{amsmath}
\newcommand{\proofeq}{\overset{!}{=}}
\newcommand{\proofeqv}{\overset{!}{\Leftrightarrow}}
\newcommand{\equivalent}{\overset{\scriptscriptstyle\wedge}{=}}
\DeclarePairedDelimiter\ceil{\lceil}{\rceil}
\DeclarePairedDelimiter\floor{\lfloor}{\rfloor}

\date{15.1.2022}
\title{Aufgabenblatt 10, Mathematik für Physiker 1}
\author{Florian Adamczyk, Finn Wagner}

\begin{document}
    \maketitle

    \section*{A 10.1}
    Zeigen Sie, dass die Funktion
    \[G: [-1, 1] \rightarrow \mathbb{R}, \quad G(x) := 
    \begin{cases}
        x \sqrt{|x|} \sin(\frac{1}{x}), &x \neq 0 \\
        0, &x = 0
    \end{cases}\]
    differenzierbar ist. Zeigen Sie dann, dass \(g := G'\) nicht stetig ist. \\ \\
    Die Funktion \(G\) ist nach Satz 5.6 für alle \(x \neq 0\) differenzierbar als Komposition von differenzierbaren Funktionen.
    Wir überprüfen nun die Differenzierbarkeit bei 0:
    \begin{gather*}
        G'(0) = \lim_{x \to 0} \frac{f(x) - f(0)}{x - 0} = \lim_{x \to 0} \frac{ x \sqrt{|x|} \sin(\frac{1}{x}) - 0 }{x - 0} =
        \lim_{x \to 0} \sqrt{|x|} \sin(\frac{1}{x}) = 0
    \end{gather*}
    Analog zum Grenzwert in Aufgabe 9.2 konvergiert der Funktionswert von beiden Seiten gegen 0. \\
    Wir leiten nun die Funktion G ab, da die Differenzierbarkeit eine lokale Eigenschaft ist, leiten wir in Teilen ab.
    \begin{gather*}
        g:= G' \\
        \text{Ableitung für } x = 0: \\
        \Rightarrow \frac{d}{dx} 0 = 0 \\
        \text{Ableitung für } x \neq 0: \\
        \Rightarrow 1 \cdot \sqrt{|x|} \sin(\frac{1}{x}) + x \frac{d}{dx} (\sqrt{|x|}) \cdot \sin(\frac{1}{x}) + x \sqrt{|x|} \cdot \left(-\frac{1}{x^2}\right) \cos(\frac{1}{x}) \\
        = \sqrt{|x|} \sin(\frac{1}{x}) + x \frac{d}{dx} \left(|x|\right) \cdot \frac{1}{2} \frac{1}{\sqrt{|x|}} \cdot \sin(\frac{1}{x}) - \frac{ \sqrt{|x|} }{x} \cos(\frac{1}{x})
    \end{gather*}
    Die Ableitung von \(|x|\) lässt sich nicht geschlossen darstellen. Sie ist an der Stelle 0 nicht definiert.
    \begin{gather*}
        |x| =
        \begin{cases}
            x, & x \geq 0 \\
            -x, & x < 0
        \end{cases} \quad \Rightarrow \quad \frac{d}{dx} |x| =
        \begin{cases}
            1, & x > 0 \\
            -1, & x < 0
        \end{cases}
    \end{gather*}
    Zur besseren Übersicht teilen wir die Ableitung für \(x \neq 0\) in \(x > 0\) und \(x < 0\) auf. Insgesamt ergibt sich:
    \begin{gather*}
        g(x) = 
        \begin{cases}
            \sqrt{|x|} \sin(\frac{1}{x}) + \frac{x}{2} \frac{1}{\sqrt{|x|}} \cdot \sin(\frac{1}{x}) - \frac{ \sqrt{|x|} }{x} \cos(\frac{1}{x}), & x > 0 \\
            \sqrt{|x|} \sin(\frac{1}{x}) - \frac{x}{2} \frac{1}{\sqrt{|x|}} \cdot \sin(\frac{1}{x}) - \frac{ \sqrt{|x|} }{x} \cos(\frac{1}{x}), & x < 0 \\
            0, & x = 0
        \end{cases}
    \end{gather*}
    Wir können weiter vereinfachen, indem wir die Betragsfunktion auflösen.
    \begin{gather*}
        g(x) = 
        \begin{cases}
            \sqrt{x} \sin(\frac{1}{x}) + \frac{1}{2} \frac{x}{\sqrt{x}} \cdot \sin(\frac{1}{x}) - \frac{ \sqrt{x} }{x} \cos(\frac{1}{x}), & x > 0 \\
            \sqrt{-x} \sin(\frac{1}{x}) - \frac{1}{2} \frac{x}{\sqrt{-x}} \cdot \sin(\frac{1}{x}) - \frac{ \sqrt{-x} }{x} \cos(\frac{1}{x}), & x < 0 \\
            0, & x = 0
        \end{cases}
    \end{gather*}
    Wir wollen nun überprüfen, ob \(g\) stetig ist. \(g\) ist für alle positiven und für alle negativen \(x\) stetig als Komposition von stetigen Funktionen nach Satz 4.13.
    Wir müssen also noch überprüfen, ob \(g\) an der Stelle 0 stetig ist. Dafür muss gelten
    \[ \lim_{x \to 0 -} g(x) = g(0) = \lim_{x \to 0 +} g(x) \]
    Wir bilden nun den rechtsseitigen Grenzwert
    \begin{gather*}
        \lim_{x \to 0 +} g(x) = \sqrt{x} \sin(\frac{1}{x}) + \frac{1}{2} \frac{x}{\sqrt{x}} \cdot \sin(\frac{1}{x}) - \frac{ \sqrt{x} }{x} \cos(\frac{1}{x}) \\
        = \lim_{x \to 0 +} \sqrt{x} \sin(\frac{1}{x}) + \frac{1}{2} \sqrt{x} \cdot \sin(\frac{1}{x}) - \frac{1}{\sqrt{x}} \cos(\frac{1}{x}) \\
        = \lim_{x \to 0 +} \frac{3}{2} \sqrt{x} \cdot \sin(\frac{1}{x}) - \frac{1}{\sqrt{x}} \cos(\frac{1}{x}) \\
    \end{gather*}
    Nach Definition 4.19 muss der Grenzwert für alle Folgen \(f(x_n)\) existieren und zum selben Wert konvergieren, dann konvergiert die Funktion. \\
    Wir definieren nun zwei Folgen
    \begin{gather*}
        {(x_n)}_{n \in \mathbb{N}} = \frac{1}{2 \pi n + \frac{\pi}{4}} \\
        {(y_n)}_{n \in \mathbb{N}} = \frac{1}{2 \pi n + \frac{5 \pi}{4}}
    \end{gather*}
    Beide Folgen konvergieren für \(n \to \infty \) von oben gegen 0
    Wir können also den Grenzwert mit \(x_n\) umformulieren:
    \begin{gather*}
        \lim_{x \to 0 +} g(x) = \lim_{n \to \infty} g(x_n) = \lim_{n \to \infty} \frac{3}{2} \sqrt{x_n} \cdot \sin(\frac{1}{x_n}) - \frac{1}{\sqrt{x_n}} \cos(\frac{1}{x_n}) \\
        = \lim_{n \to \infty} \frac{3}{2} \sqrt{\frac{1}{2 \pi n + \frac{\pi}{4}}} \cdot \sin(\frac{1}{\frac{1}{2 \pi n + \frac{\pi}{4}}}) - \frac{1}{\sqrt{\frac{1}{2 \pi n + \frac{\pi}{4}}}} \cos(\frac{1}{\frac{1}{2 \pi n + \frac{\pi}{4}}}) \\
        = \lim_{n \to \infty} \frac{3}{2} \sqrt{\frac{1}{2 \pi n + \frac{\pi}{4}}} \cdot \sin(2 \pi n + \frac{\pi}{4}) - \sqrt{2 \pi n + \frac{\pi}{4}} \cos(2 \pi n + \frac{\pi}{4}) \\
        = \frac{3}{2} \cdot 0 \cdot \frac{1}{\sqrt{2}} - \infty \cdot \frac{1}{\sqrt{2}} \\
        = - \infty
    \end{gather*}
    Ebenso mit \(y_n\)
    \begin{gather*}
        \lim_{x \to 0 +} g(x) = \lim_{n \to \infty} g(y_n) = \lim_{n \to \infty} \frac{3}{2} \sqrt{y_n} \cdot \sin(\frac{1}{y_n}) - \frac{1}{\sqrt{y_n}} \cos(\frac{1}{y_n}) \\
        = \lim_{n \to \infty} \frac{3}{2} \sqrt{\frac{1}{2 \pi n + \frac{5 \pi}{4}}} \cdot \sin(\frac{1}{\frac{1}{2 \pi n + \frac{5 \pi}{4}}}) - \frac{1}{\sqrt{\frac{1}{2 \pi n + \frac{5 \pi}{4}}}} \cos(\frac{1}{\frac{1}{2 \pi n + \frac{5 \pi}{4}}}) \\
        = \lim_{n \to \infty} \frac{3}{2} \sqrt{\frac{1}{2 \pi n + \frac{5 \pi}{4}}} \cdot \sin(2 \pi n + \frac{5 \pi}{4}) - \sqrt{2 \pi n + \frac{5 \pi}{4}} \cos(2 \pi n + \frac{5 \pi}{4}) \\
        = \frac{3}{2} \cdot 0 \cdot \frac{-1}{\sqrt{2}} - \infty \cdot \frac{-1}{\sqrt{2}} \\
        = \infty
    \end{gather*}
    Da der Grenzwert der beiden Folgen nicht der selbe ist existiert der Grenzwert von \(\lim_{x \to 0 +} g(x)\) nicht. Er ist somit insbesondere nicht gleich Null.
    Die Funktion ist also im Punkt Null nicht stetig, da \(\lim_{x \to 0 -} g(x) \neq g(0)\) ist, damit auch nicht auf ihrem gesamten Definitionsbereich.

    \section*{A 10.2}
    Beweisen Sie Satz 6.11. Zeigen Sie also, dass für stetige Funktionen \(f, g : I \to \mathbb{C}\) und \(a, b \in I\) die folgenden Eigenschaften gelten:
    \begin{enumerate}[ label = (\alph*) ]
        \item \( \int_a^b (f+g)(t) dt = \int_a^b f(t) dt + \int_a^b g(t) dt \) \\
        Wir betrachten zunächt die uneigentlichen Integrale
        \[ \int (f+g)(t) dt = \int f(t) dt + \int g(t) dt \]
        Wir leiten nun auf beiden Seiten nach \(t\) ab
        \[ \frac{d}{dt} \left( \int (f+g)(t) dt \right) =  \frac{d}{dt} \left( \int f(t) dt + \int g(t) dt \right) \]
        Nach Satz 5.6 (a) können wir die Ableitung auf der rechten Seite auf die Summanden aufteilen
        \[ \frac{d}{dt} \left( \int (f+g)(t) dt \right) =  \frac{d}{dt} \left( \int f(t) dt \right) + \frac{d}{dt} \left( \int g(t) dt \right) \]
        Es ergibt sich
        \[(f+g)(t) = f(t) + g(t)\]
        Was eine wahre Aussage ist. \\
        Nun betrachten wir noch die Grenzen. Da \(f, g\) stetig sind besitzen sie die Stammfunktionen \(F, G\).
        Die Stammfunktion von \((f+g)(t)\) ist nach dem oben gezeigten also
        \[ \int_a^b (f+g)(t) dt = F(t) + G(t) |_a^b\]
        Wir setzen nun die Grenzen ein und formen um
        \begin{gather*}
            \int_a^b (f+g)(t) dt = (F(t) + G(t)) |_a^b = (F(b) + G(b)) - (F(a) + G(a)) \\
            = (F(b) - F(a)) + (G(b) - G(a)) = \int_a^b f(t) dt + \int_a^b g(t) dt
        \end{gather*}

        \item \( \int_a^b \lambda f(t) dt = \lambda \int_a^b f(t) dt \) für \(\lambda \in \mathbb{C} \) \\
        Wir leiten hier wieder beide Seiten nach \(t\) ab und lassen zuerst die Grenzen weg
        \[ \frac{d}{dt} \left( \int \lambda f(t) dt \right) = \frac{d}{dt} \left( \lambda \int f(t) dt \right) \]
        Auf der rechten Seite leiten wir nach Produktregel ab, \( p(t) = \lambda \) und \(q(t) = \int f(t) dt  \)
        \begin{gather*}
            \lambda f(t) = 0 \cdot \int f(t) dt + \lambda f(t) \\
            \Leftrightarrow \lambda f(t) = \lambda f(t) 
        \end{gather*}
        Was eine wahre Aussage ist.
        Nun betrachten wir noch die Grenzen. Da \(f, g\) stetig sind besitzen sie die Stammfunktionen \(F, G\).
        \begin{gather*}
            \int_a^b \lambda f(t) dt = \lambda F(t) |_a^b = \lambda F(b) - \lambda F(a) = \lambda (F(b) - F(a)) = \lambda (F(t) |_a^b) = \lambda \int_a^b f(t) dt
        \end{gather*}

        \item \( \int_a^b f(t) dt = \int_a^c f(t) dt + \int_c^b f(t) dt \) für alle \(a < b < c\)
        Da \(f\) stetig ist besitzt sie die Stammfunktion \(F\). Wir setzen die Grenzen ein und formen um
        \begin{align*}
            \int_a^b f(t) dt &= \int_a^c f(t) dt + \int_c^b f(t) dt \\
            \Leftrightarrow F(t) |_a^b &= F(t) |_a^c + F(t) |_c^b \\
            \Leftrightarrow F(t) |_a^b &= (F(c) - F(a)) + (F(b) - F(c)) \\
            \Leftrightarrow F(t) |_a^b &= (F(c) - F(c)) + (F(b) - F(a)) \\
            \Leftrightarrow F(t) |_a^b &= F(b) - F(a) \\
            \Leftrightarrow F(t) |_a^b &= F(t) |_a^b
        \end{align*}

        \item \(\int_a^b f(t) dt = - \int_b^a f(t) dt\) \\
        Da \(f\) stetig ist besitzt sie die Stammfunktion \(F\). Wir setzen die Grenzen ein und formen um
        \[\int_a^b f(t) dt = (F(b) - F(a)) = -(F(b) - F(a)) = (F(a) - F(b)) = \int_b^a f(t) dt\]

        \item \(\int_a^b f(t) dt = \int_a^b \Re(f)(t) dt + i \int_a^b \Im(f)(t) dt\) \\
        \begin{align*}
            &\int_a^b \Re(f)(t) dt + i \int_a^b \Im(f)(t) dt \\
            =^{\text{mit (b)}} &\int_a^b \Re(f)(t) dt + \int_a^b i \Im(f)(t) dt \\
            =^{\text{mit (a)}} &\int_a^b \Re(f)(t) + i \Im(f)(t) dt \\
            = &\int_a^b f(t) dt
        \end{align*}

        \section*{A 10.3}
        Auf handgeschriebenem Beiblatt

        \section*{A 10.4}
        Für \(\mathbb{R} \ni a < b \in \mathbb{R} \cup \{\infty\}\) und eine Funktion \(f: [a,b) \to \mathbb{R}\), sodass für alle \(a < r < b\) die Funktion \(f|_{[a,r]}\) Riemann integrierbar ist,
        ist das unegentliche Riemann-Integral definiert durch
        \[ \int_a^b f(x) dx := \lim_{r \to b} \int_a^r f(x) dx \]
        falls der Grenzwert existiert. Exsistiert der Grenzwert, so heißt f (uneigentlich) Riemann-integrierbar. \\
        Sei nun \(f: [1, \infty) \to [0, \infty) \) stetig und monoton fallend.
        \begin{enumerate}[ label = (\alph*)]
            \item Zeigen Sie, dass \(f\) genau dann unegentlich integrierbar ist, falls die Reihe
            \( \sum_{n=1}^\infty f(n)\)
            konvergent ist. \\

            1. Die Reihe konvergiert \( \Rightarrow \) \(f\) ist uneigentlich integrierbar \\
            Da \(f\) stetig ist, ist \(f\) integrierbar und besitzt eine Stammfunktion \(F\). \\
            Seien \(c, d \in \mathbb{R} \subset [a, b) \). Dann existiert das Integral
            \[ \int_c^d f(x) dx\]
            und ist endlich.
            Wir wollen nun diesen endlichen Integral nach oben abschätzen. Da \(f\) monoton fallend ist, ist der erste Wert in einem Intervall immer das Maximum des Intervalls
            und der letzte Wert das Minimum. Auf einem Intervall \([n, n+1], \text{ mit } n \in \mathbb{N} \). Ist also \(f(n)\) das Maximum.
            Die Summe 
            \[ \int_c^d f(x) dx \leq \int_{\floor{c}}^{\floor{d+1}} f(x) dx  = \sum_{n = \floor{c}}^{\floor{d}} \int_n^{n+1} f(x) dx \leq \sum_{n = \floor{c}}^{\floor{d}} f(n) \cdot ((n+1)-n) \]
            ist also immer größer gleich dem Integral. 
            Da \(f\) nur auf die 0 und positive Zahlen abbildet ist das Integral über \(f\) monoton steigend und größer gleich 0.
            Wir bilden nun den Grenzwert von \(d \to \infty \)
            \[ \lim_{d \to \infty} \int_c^d f(x) dx \leq \lim_{d \to \infty} \sum_{n = \floor{c}}^{\floor{d}} f(n) \]
            Da der Wert des Integrals monoton steigend ist mit größerem \(d\) und nach oben durch die Reihe beschränkt ist, konvergiert er. \\

            2. \(f\) ist uneigentlich integrierbar \(\Rightarrow \) die Reihe konvergiert \\
            Da der Wertebereich von \(f\), \([0, \infty)\) ist, ist jedes einzelne Element der Reihe größer gleich Null. Deswegen ist die Reihe monoton steigend.
            \[ 0 \leq \sum_{n=\ceil{2}}^{\ceil{d}} \]
            Analog zu oben ist diese Reihe eine Abschätzung nach unten für das Integral
            \[ \int_2^{d} f(x) dx \]
            Zusammen
            \[0 \leq \sum_{n=\ceil{2}}^{\ceil{d}} \leq \int_2^{d} f(x) dx \]
            Wir bilden nun den Grenzwert von \(d \to \infty \)
            \[ \lim_{d \to \infty} 0 \leq \lim_{d \to \infty} \sum_{n=\ceil{2}}^{\ceil{d}} \leq \lim_{d \to \infty} \int_2^{d} f(x) dx \]
            \[ 0 \leq \sum_{n=2}^{\infty} \leq \int_2^{\infty} f(x) dx \]
            Da wir gegeben haben, dass \(f\) uneigentlich integrierbar ist, ist die Reihe nach oben und unten gegen einen reellen Wert begrenzt.
            Sie muss also konvergieren. \(f(1) \in \mathbb{R}\)
            \[ f(1) + \sum_{n=2}^{\infty} = \sum_{n=1}^{\infty} \]
            Die Reihe \( \sum_{n=1}^{\infty} \) konvergiert.
        \end{enumerate}

        \item Untersuche Sie für welche \(\alpha \in \mathbb{R}\) die Reihe \(\sum_{n=1}^\infty n^{\alpha} \) konvergent ist. \\
        Nach (a) muss, damit die Reihe konvergiert das uneigentliche Integral exestieren.
        \begin{align*}
            &\lim_{x \to \infty} \int_1^{x} f(t) dt \\
            = &\lim_{x \to \infty} \int_1^{x} t^{\alpha} dt \\
            = &\lim_{x \to \infty} (F(x) - F(1)) \\
            = &\lim_{x \to \infty} (\frac{1}{\alpha + 1} x^{\alpha + 1} - \frac{1}{\alpha + 1} 1^{\alpha + 1}) \\
            = &\lim_{x \to \infty} \frac{1}{\alpha + 1} (x^{\alpha + 1} - 1) \\
        \end{align*}
        Wir überprüfen nun die Konvergenz von \(\lim_{x \to \infty} \frac{1}{\alpha + 1} (x^{\alpha + 1} - 1) \) \\
        1. Fall \(\alpha > -1\) \\
        \(\alpha + 1\) ist in diesem Fall positiv. \(x\) hoch eine positive Zahl konvergiert also gegen undendlich.
        Da das uneigentliche Integral aber nicht unendlich werden kann sind diese \(\alpha\) Werte nicht erlaubt. \\
        2. Fall \(\alpha = -1\) \\
        In diesem Fall ist die Reihe \(\sum_{n=1}^{\infty} n^{\alpha} = \sum_{n=1}^{\infty} \frac{1}{n}\) gerade die Harmonische Reihe.
        Diese Reihe konvergiert nicht. \\
        3. Fall \(\alpha < -1\) \\
        \(\alpha + 1\) ist in diesem Fall negativ. \(x\) hoch eine negative Zahl konvergiert also gegen 0.
        \(\lim_{x \to \infty} \frac{1}{\alpha + 1} (x^{\alpha + 1} - 1) \) = 0 \\\\
        Die Reihe konvergiert also für alle \(\alpha < -1 \). Sonst divergiert sie. \\
        Aus dieser Rechnung lässt sich auch das uneigentliche Integral von 1 bis \(\infty\) einer Funktion der Form \(x^{\alpha}\) als Formel von \(\alpha\) (für \(\alpha < -1\)) darstellen.
        Das Integral existiert nur, wenn \(x^{\alpha + 1}\) im Grenzwertprozess 0 wird.
        \(\int_1^{\infty} f(t) dt = \lim_{x \to \infty} \frac{1}{\alpha + 1} (x^{\alpha + 1} - 1) = \frac{1}{\alpha + 1} (0 - 1) = \frac{-1}{\alpha + 1}\)

    \end{enumerate}
\end{document}